\documentclass[11pt]{article}

    \usepackage[breakable]{tcolorbox}
    \usepackage{parskip} % Stop auto-indenting (to mimic markdown behaviour)
    
    \usepackage{iftex}
    \ifPDFTeX
    	\usepackage[T1]{fontenc}
    	\usepackage{mathpazo}
    \else
    	\usepackage{fontspec}
    \fi

    % Basic figure setup, for now with no caption control since it's done
    % automatically by Pandoc (which extracts ![](path) syntax from Markdown).
    \usepackage{graphicx}
    % Maintain compatibility with old templates. Remove in nbconvert 6.0
    \let\Oldincludegraphics\includegraphics
    % Ensure that by default, figures have no caption (until we provide a
    % proper Figure object with a Caption API and a way to capture that
    % in the conversion process - todo).
    \usepackage{caption}
    \DeclareCaptionFormat{nocaption}{}
    \captionsetup{format=nocaption,aboveskip=0pt,belowskip=0pt}

    \usepackage[Export]{adjustbox} % Used to constrain images to a maximum size
    \adjustboxset{max size={0.9\linewidth}{0.9\paperheight}}
    \usepackage{float}
    \floatplacement{figure}{H} % forces figures to be placed at the correct location
    \usepackage{xcolor} % Allow colors to be defined
    \usepackage{enumerate} % Needed for markdown enumerations to work
    \usepackage{geometry} % Used to adjust the document margins
    \usepackage{amsmath} % Equations
    \usepackage{amssymb} % Equations
    \usepackage{textcomp} % defines textquotesingle
    % Hack from http://tex.stackexchange.com/a/47451/13684:
    \AtBeginDocument{%
        \def\PYZsq{\textquotesingle}% Upright quotes in Pygmentized code
    }
    \usepackage{upquote} % Upright quotes for verbatim code
    \usepackage{eurosym} % defines \euro
    \usepackage[mathletters]{ucs} % Extended unicode (utf-8) support
    \usepackage{fancyvrb} % verbatim replacement that allows latex
    \usepackage{grffile} % extends the file name processing of package graphics 
                         % to support a larger range
    \makeatletter % fix for grffile with XeLaTeX
    \def\Gread@@xetex#1{%
      \IfFileExists{"\Gin@base".bb}%
      {\Gread@eps{\Gin@base.bb}}%
      {\Gread@@xetex@aux#1}%
    }
    \makeatother

    % The hyperref package gives us a pdf with properly built
    % internal navigation ('pdf bookmarks' for the table of contents,
    % internal cross-reference links, web links for URLs, etc.)
    \usepackage{hyperref}
    % The default LaTeX title has an obnoxious amount of whitespace. By default,
    % titling removes some of it. It also provides customization options.
    \usepackage{titling}
    \usepackage{longtable} % longtable support required by pandoc >1.10
    \usepackage{booktabs}  % table support for pandoc > 1.12.2
    \usepackage[inline]{enumitem} % IRkernel/repr support (it uses the enumerate* environment)
    \usepackage[normalem]{ulem} % ulem is needed to support strikethroughs (\sout)
                                % normalem makes italics be italics, not underlines
    \usepackage{mathrsfs}
    

    
    % Colors for the hyperref package
    \definecolor{urlcolor}{rgb}{0,.145,.698}
    \definecolor{linkcolor}{rgb}{.71,0.21,0.01}
    \definecolor{citecolor}{rgb}{.12,.54,.11}

    % ANSI colors
    \definecolor{ansi-black}{HTML}{3E424D}
    \definecolor{ansi-black-intense}{HTML}{282C36}
    \definecolor{ansi-red}{HTML}{E75C58}
    \definecolor{ansi-red-intense}{HTML}{B22B31}
    \definecolor{ansi-green}{HTML}{00A250}
    \definecolor{ansi-green-intense}{HTML}{007427}
    \definecolor{ansi-yellow}{HTML}{DDB62B}
    \definecolor{ansi-yellow-intense}{HTML}{B27D12}
    \definecolor{ansi-blue}{HTML}{208FFB}
    \definecolor{ansi-blue-intense}{HTML}{0065CA}
    \definecolor{ansi-magenta}{HTML}{D160C4}
    \definecolor{ansi-magenta-intense}{HTML}{A03196}
    \definecolor{ansi-cyan}{HTML}{60C6C8}
    \definecolor{ansi-cyan-intense}{HTML}{258F8F}
    \definecolor{ansi-white}{HTML}{C5C1B4}
    \definecolor{ansi-white-intense}{HTML}{A1A6B2}
    \definecolor{ansi-default-inverse-fg}{HTML}{FFFFFF}
    \definecolor{ansi-default-inverse-bg}{HTML}{000000}

    % commands and environments needed by pandoc snippets
    % extracted from the output of `pandoc -s`
    \providecommand{\tightlist}{%
      \setlength{\itemsep}{0pt}\setlength{\parskip}{0pt}}
    \DefineVerbatimEnvironment{Highlighting}{Verbatim}{commandchars=\\\{\}}
    % Add ',fontsize=\small' for more characters per line
    \newenvironment{Shaded}{}{}
    \newcommand{\KeywordTok}[1]{\textcolor[rgb]{0.00,0.44,0.13}{\textbf{{#1}}}}
    \newcommand{\DataTypeTok}[1]{\textcolor[rgb]{0.56,0.13,0.00}{{#1}}}
    \newcommand{\DecValTok}[1]{\textcolor[rgb]{0.25,0.63,0.44}{{#1}}}
    \newcommand{\BaseNTok}[1]{\textcolor[rgb]{0.25,0.63,0.44}{{#1}}}
    \newcommand{\FloatTok}[1]{\textcolor[rgb]{0.25,0.63,0.44}{{#1}}}
    \newcommand{\CharTok}[1]{\textcolor[rgb]{0.25,0.44,0.63}{{#1}}}
    \newcommand{\StringTok}[1]{\textcolor[rgb]{0.25,0.44,0.63}{{#1}}}
    \newcommand{\CommentTok}[1]{\textcolor[rgb]{0.38,0.63,0.69}{\textit{{#1}}}}
    \newcommand{\OtherTok}[1]{\textcolor[rgb]{0.00,0.44,0.13}{{#1}}}
    \newcommand{\AlertTok}[1]{\textcolor[rgb]{1.00,0.00,0.00}{\textbf{{#1}}}}
    \newcommand{\FunctionTok}[1]{\textcolor[rgb]{0.02,0.16,0.49}{{#1}}}
    \newcommand{\RegionMarkerTok}[1]{{#1}}
    \newcommand{\ErrorTok}[1]{\textcolor[rgb]{1.00,0.00,0.00}{\textbf{{#1}}}}
    \newcommand{\NormalTok}[1]{{#1}}
    
    % Additional commands for more recent versions of Pandoc
    \newcommand{\ConstantTok}[1]{\textcolor[rgb]{0.53,0.00,0.00}{{#1}}}
    \newcommand{\SpecialCharTok}[1]{\textcolor[rgb]{0.25,0.44,0.63}{{#1}}}
    \newcommand{\VerbatimStringTok}[1]{\textcolor[rgb]{0.25,0.44,0.63}{{#1}}}
    \newcommand{\SpecialStringTok}[1]{\textcolor[rgb]{0.73,0.40,0.53}{{#1}}}
    \newcommand{\ImportTok}[1]{{#1}}
    \newcommand{\DocumentationTok}[1]{\textcolor[rgb]{0.73,0.13,0.13}{\textit{{#1}}}}
    \newcommand{\AnnotationTok}[1]{\textcolor[rgb]{0.38,0.63,0.69}{\textbf{\textit{{#1}}}}}
    \newcommand{\CommentVarTok}[1]{\textcolor[rgb]{0.38,0.63,0.69}{\textbf{\textit{{#1}}}}}
    \newcommand{\VariableTok}[1]{\textcolor[rgb]{0.10,0.09,0.49}{{#1}}}
    \newcommand{\ControlFlowTok}[1]{\textcolor[rgb]{0.00,0.44,0.13}{\textbf{{#1}}}}
    \newcommand{\OperatorTok}[1]{\textcolor[rgb]{0.40,0.40,0.40}{{#1}}}
    \newcommand{\BuiltInTok}[1]{{#1}}
    \newcommand{\ExtensionTok}[1]{{#1}}
    \newcommand{\PreprocessorTok}[1]{\textcolor[rgb]{0.74,0.48,0.00}{{#1}}}
    \newcommand{\AttributeTok}[1]{\textcolor[rgb]{0.49,0.56,0.16}{{#1}}}
    \newcommand{\InformationTok}[1]{\textcolor[rgb]{0.38,0.63,0.69}{\textbf{\textit{{#1}}}}}
    \newcommand{\WarningTok}[1]{\textcolor[rgb]{0.38,0.63,0.69}{\textbf{\textit{{#1}}}}}
    
    
    % Define a nice break command that doesn't care if a line doesn't already
    % exist.
    \def\br{\hspace*{\fill} \\* }
    % Math Jax compatibility definitions
    \def\gt{>}
    \def\lt{<}
    \let\Oldtex\TeX
    \let\Oldlatex\LaTeX
    \renewcommand{\TeX}{\textrm{\Oldtex}}
    \renewcommand{\LaTeX}{\textrm{\Oldlatex}}
    % Document parameters
    % Document title
    \title{SageMaker Project}
    
    
    
    
    
% Pygments definitions
\makeatletter
\def\PY@reset{\let\PY@it=\relax \let\PY@bf=\relax%
    \let\PY@ul=\relax \let\PY@tc=\relax%
    \let\PY@bc=\relax \let\PY@ff=\relax}
\def\PY@tok#1{\csname PY@tok@#1\endcsname}
\def\PY@toks#1+{\ifx\relax#1\empty\else%
    \PY@tok{#1}\expandafter\PY@toks\fi}
\def\PY@do#1{\PY@bc{\PY@tc{\PY@ul{%
    \PY@it{\PY@bf{\PY@ff{#1}}}}}}}
\def\PY#1#2{\PY@reset\PY@toks#1+\relax+\PY@do{#2}}

\expandafter\def\csname PY@tok@w\endcsname{\def\PY@tc##1{\textcolor[rgb]{0.73,0.73,0.73}{##1}}}
\expandafter\def\csname PY@tok@c\endcsname{\let\PY@it=\textit\def\PY@tc##1{\textcolor[rgb]{0.25,0.50,0.50}{##1}}}
\expandafter\def\csname PY@tok@cp\endcsname{\def\PY@tc##1{\textcolor[rgb]{0.74,0.48,0.00}{##1}}}
\expandafter\def\csname PY@tok@k\endcsname{\let\PY@bf=\textbf\def\PY@tc##1{\textcolor[rgb]{0.00,0.50,0.00}{##1}}}
\expandafter\def\csname PY@tok@kp\endcsname{\def\PY@tc##1{\textcolor[rgb]{0.00,0.50,0.00}{##1}}}
\expandafter\def\csname PY@tok@kt\endcsname{\def\PY@tc##1{\textcolor[rgb]{0.69,0.00,0.25}{##1}}}
\expandafter\def\csname PY@tok@o\endcsname{\def\PY@tc##1{\textcolor[rgb]{0.40,0.40,0.40}{##1}}}
\expandafter\def\csname PY@tok@ow\endcsname{\let\PY@bf=\textbf\def\PY@tc##1{\textcolor[rgb]{0.67,0.13,1.00}{##1}}}
\expandafter\def\csname PY@tok@nb\endcsname{\def\PY@tc##1{\textcolor[rgb]{0.00,0.50,0.00}{##1}}}
\expandafter\def\csname PY@tok@nf\endcsname{\def\PY@tc##1{\textcolor[rgb]{0.00,0.00,1.00}{##1}}}
\expandafter\def\csname PY@tok@nc\endcsname{\let\PY@bf=\textbf\def\PY@tc##1{\textcolor[rgb]{0.00,0.00,1.00}{##1}}}
\expandafter\def\csname PY@tok@nn\endcsname{\let\PY@bf=\textbf\def\PY@tc##1{\textcolor[rgb]{0.00,0.00,1.00}{##1}}}
\expandafter\def\csname PY@tok@ne\endcsname{\let\PY@bf=\textbf\def\PY@tc##1{\textcolor[rgb]{0.82,0.25,0.23}{##1}}}
\expandafter\def\csname PY@tok@nv\endcsname{\def\PY@tc##1{\textcolor[rgb]{0.10,0.09,0.49}{##1}}}
\expandafter\def\csname PY@tok@no\endcsname{\def\PY@tc##1{\textcolor[rgb]{0.53,0.00,0.00}{##1}}}
\expandafter\def\csname PY@tok@nl\endcsname{\def\PY@tc##1{\textcolor[rgb]{0.63,0.63,0.00}{##1}}}
\expandafter\def\csname PY@tok@ni\endcsname{\let\PY@bf=\textbf\def\PY@tc##1{\textcolor[rgb]{0.60,0.60,0.60}{##1}}}
\expandafter\def\csname PY@tok@na\endcsname{\def\PY@tc##1{\textcolor[rgb]{0.49,0.56,0.16}{##1}}}
\expandafter\def\csname PY@tok@nt\endcsname{\let\PY@bf=\textbf\def\PY@tc##1{\textcolor[rgb]{0.00,0.50,0.00}{##1}}}
\expandafter\def\csname PY@tok@nd\endcsname{\def\PY@tc##1{\textcolor[rgb]{0.67,0.13,1.00}{##1}}}
\expandafter\def\csname PY@tok@s\endcsname{\def\PY@tc##1{\textcolor[rgb]{0.73,0.13,0.13}{##1}}}
\expandafter\def\csname PY@tok@sd\endcsname{\let\PY@it=\textit\def\PY@tc##1{\textcolor[rgb]{0.73,0.13,0.13}{##1}}}
\expandafter\def\csname PY@tok@si\endcsname{\let\PY@bf=\textbf\def\PY@tc##1{\textcolor[rgb]{0.73,0.40,0.53}{##1}}}
\expandafter\def\csname PY@tok@se\endcsname{\let\PY@bf=\textbf\def\PY@tc##1{\textcolor[rgb]{0.73,0.40,0.13}{##1}}}
\expandafter\def\csname PY@tok@sr\endcsname{\def\PY@tc##1{\textcolor[rgb]{0.73,0.40,0.53}{##1}}}
\expandafter\def\csname PY@tok@ss\endcsname{\def\PY@tc##1{\textcolor[rgb]{0.10,0.09,0.49}{##1}}}
\expandafter\def\csname PY@tok@sx\endcsname{\def\PY@tc##1{\textcolor[rgb]{0.00,0.50,0.00}{##1}}}
\expandafter\def\csname PY@tok@m\endcsname{\def\PY@tc##1{\textcolor[rgb]{0.40,0.40,0.40}{##1}}}
\expandafter\def\csname PY@tok@gh\endcsname{\let\PY@bf=\textbf\def\PY@tc##1{\textcolor[rgb]{0.00,0.00,0.50}{##1}}}
\expandafter\def\csname PY@tok@gu\endcsname{\let\PY@bf=\textbf\def\PY@tc##1{\textcolor[rgb]{0.50,0.00,0.50}{##1}}}
\expandafter\def\csname PY@tok@gd\endcsname{\def\PY@tc##1{\textcolor[rgb]{0.63,0.00,0.00}{##1}}}
\expandafter\def\csname PY@tok@gi\endcsname{\def\PY@tc##1{\textcolor[rgb]{0.00,0.63,0.00}{##1}}}
\expandafter\def\csname PY@tok@gr\endcsname{\def\PY@tc##1{\textcolor[rgb]{1.00,0.00,0.00}{##1}}}
\expandafter\def\csname PY@tok@ge\endcsname{\let\PY@it=\textit}
\expandafter\def\csname PY@tok@gs\endcsname{\let\PY@bf=\textbf}
\expandafter\def\csname PY@tok@gp\endcsname{\let\PY@bf=\textbf\def\PY@tc##1{\textcolor[rgb]{0.00,0.00,0.50}{##1}}}
\expandafter\def\csname PY@tok@go\endcsname{\def\PY@tc##1{\textcolor[rgb]{0.53,0.53,0.53}{##1}}}
\expandafter\def\csname PY@tok@gt\endcsname{\def\PY@tc##1{\textcolor[rgb]{0.00,0.27,0.87}{##1}}}
\expandafter\def\csname PY@tok@err\endcsname{\def\PY@bc##1{\setlength{\fboxsep}{0pt}\fcolorbox[rgb]{1.00,0.00,0.00}{1,1,1}{\strut ##1}}}
\expandafter\def\csname PY@tok@kc\endcsname{\let\PY@bf=\textbf\def\PY@tc##1{\textcolor[rgb]{0.00,0.50,0.00}{##1}}}
\expandafter\def\csname PY@tok@kd\endcsname{\let\PY@bf=\textbf\def\PY@tc##1{\textcolor[rgb]{0.00,0.50,0.00}{##1}}}
\expandafter\def\csname PY@tok@kn\endcsname{\let\PY@bf=\textbf\def\PY@tc##1{\textcolor[rgb]{0.00,0.50,0.00}{##1}}}
\expandafter\def\csname PY@tok@kr\endcsname{\let\PY@bf=\textbf\def\PY@tc##1{\textcolor[rgb]{0.00,0.50,0.00}{##1}}}
\expandafter\def\csname PY@tok@bp\endcsname{\def\PY@tc##1{\textcolor[rgb]{0.00,0.50,0.00}{##1}}}
\expandafter\def\csname PY@tok@fm\endcsname{\def\PY@tc##1{\textcolor[rgb]{0.00,0.00,1.00}{##1}}}
\expandafter\def\csname PY@tok@vc\endcsname{\def\PY@tc##1{\textcolor[rgb]{0.10,0.09,0.49}{##1}}}
\expandafter\def\csname PY@tok@vg\endcsname{\def\PY@tc##1{\textcolor[rgb]{0.10,0.09,0.49}{##1}}}
\expandafter\def\csname PY@tok@vi\endcsname{\def\PY@tc##1{\textcolor[rgb]{0.10,0.09,0.49}{##1}}}
\expandafter\def\csname PY@tok@vm\endcsname{\def\PY@tc##1{\textcolor[rgb]{0.10,0.09,0.49}{##1}}}
\expandafter\def\csname PY@tok@sa\endcsname{\def\PY@tc##1{\textcolor[rgb]{0.73,0.13,0.13}{##1}}}
\expandafter\def\csname PY@tok@sb\endcsname{\def\PY@tc##1{\textcolor[rgb]{0.73,0.13,0.13}{##1}}}
\expandafter\def\csname PY@tok@sc\endcsname{\def\PY@tc##1{\textcolor[rgb]{0.73,0.13,0.13}{##1}}}
\expandafter\def\csname PY@tok@dl\endcsname{\def\PY@tc##1{\textcolor[rgb]{0.73,0.13,0.13}{##1}}}
\expandafter\def\csname PY@tok@s2\endcsname{\def\PY@tc##1{\textcolor[rgb]{0.73,0.13,0.13}{##1}}}
\expandafter\def\csname PY@tok@sh\endcsname{\def\PY@tc##1{\textcolor[rgb]{0.73,0.13,0.13}{##1}}}
\expandafter\def\csname PY@tok@s1\endcsname{\def\PY@tc##1{\textcolor[rgb]{0.73,0.13,0.13}{##1}}}
\expandafter\def\csname PY@tok@mb\endcsname{\def\PY@tc##1{\textcolor[rgb]{0.40,0.40,0.40}{##1}}}
\expandafter\def\csname PY@tok@mf\endcsname{\def\PY@tc##1{\textcolor[rgb]{0.40,0.40,0.40}{##1}}}
\expandafter\def\csname PY@tok@mh\endcsname{\def\PY@tc##1{\textcolor[rgb]{0.40,0.40,0.40}{##1}}}
\expandafter\def\csname PY@tok@mi\endcsname{\def\PY@tc##1{\textcolor[rgb]{0.40,0.40,0.40}{##1}}}
\expandafter\def\csname PY@tok@il\endcsname{\def\PY@tc##1{\textcolor[rgb]{0.40,0.40,0.40}{##1}}}
\expandafter\def\csname PY@tok@mo\endcsname{\def\PY@tc##1{\textcolor[rgb]{0.40,0.40,0.40}{##1}}}
\expandafter\def\csname PY@tok@ch\endcsname{\let\PY@it=\textit\def\PY@tc##1{\textcolor[rgb]{0.25,0.50,0.50}{##1}}}
\expandafter\def\csname PY@tok@cm\endcsname{\let\PY@it=\textit\def\PY@tc##1{\textcolor[rgb]{0.25,0.50,0.50}{##1}}}
\expandafter\def\csname PY@tok@cpf\endcsname{\let\PY@it=\textit\def\PY@tc##1{\textcolor[rgb]{0.25,0.50,0.50}{##1}}}
\expandafter\def\csname PY@tok@c1\endcsname{\let\PY@it=\textit\def\PY@tc##1{\textcolor[rgb]{0.25,0.50,0.50}{##1}}}
\expandafter\def\csname PY@tok@cs\endcsname{\let\PY@it=\textit\def\PY@tc##1{\textcolor[rgb]{0.25,0.50,0.50}{##1}}}

\def\PYZbs{\char`\\}
\def\PYZus{\char`\_}
\def\PYZob{\char`\{}
\def\PYZcb{\char`\}}
\def\PYZca{\char`\^}
\def\PYZam{\char`\&}
\def\PYZlt{\char`\<}
\def\PYZgt{\char`\>}
\def\PYZsh{\char`\#}
\def\PYZpc{\char`\%}
\def\PYZdl{\char`\$}
\def\PYZhy{\char`\-}
\def\PYZsq{\char`\'}
\def\PYZdq{\char`\"}
\def\PYZti{\char`\~}
% for compatibility with earlier versions
\def\PYZat{@}
\def\PYZlb{[}
\def\PYZrb{]}
\makeatother


    % For linebreaks inside Verbatim environment from package fancyvrb. 
    \makeatletter
        \newbox\Wrappedcontinuationbox 
        \newbox\Wrappedvisiblespacebox 
        \newcommand*\Wrappedvisiblespace {\textcolor{red}{\textvisiblespace}} 
        \newcommand*\Wrappedcontinuationsymbol {\textcolor{red}{\llap{\tiny$\m@th\hookrightarrow$}}} 
        \newcommand*\Wrappedcontinuationindent {3ex } 
        \newcommand*\Wrappedafterbreak {\kern\Wrappedcontinuationindent\copy\Wrappedcontinuationbox} 
        % Take advantage of the already applied Pygments mark-up to insert 
        % potential linebreaks for TeX processing. 
        %        {, <, #, %, $, ' and ": go to next line. 
        %        _, }, ^, &, >, - and ~: stay at end of broken line. 
        % Use of \textquotesingle for straight quote. 
        \newcommand*\Wrappedbreaksatspecials {% 
            \def\PYGZus{\discretionary{\char`\_}{\Wrappedafterbreak}{\char`\_}}% 
            \def\PYGZob{\discretionary{}{\Wrappedafterbreak\char`\{}{\char`\{}}% 
            \def\PYGZcb{\discretionary{\char`\}}{\Wrappedafterbreak}{\char`\}}}% 
            \def\PYGZca{\discretionary{\char`\^}{\Wrappedafterbreak}{\char`\^}}% 
            \def\PYGZam{\discretionary{\char`\&}{\Wrappedafterbreak}{\char`\&}}% 
            \def\PYGZlt{\discretionary{}{\Wrappedafterbreak\char`\<}{\char`\<}}% 
            \def\PYGZgt{\discretionary{\char`\>}{\Wrappedafterbreak}{\char`\>}}% 
            \def\PYGZsh{\discretionary{}{\Wrappedafterbreak\char`\#}{\char`\#}}% 
            \def\PYGZpc{\discretionary{}{\Wrappedafterbreak\char`\%}{\char`\%}}% 
            \def\PYGZdl{\discretionary{}{\Wrappedafterbreak\char`\$}{\char`\$}}% 
            \def\PYGZhy{\discretionary{\char`\-}{\Wrappedafterbreak}{\char`\-}}% 
            \def\PYGZsq{\discretionary{}{\Wrappedafterbreak\textquotesingle}{\textquotesingle}}% 
            \def\PYGZdq{\discretionary{}{\Wrappedafterbreak\char`\"}{\char`\"}}% 
            \def\PYGZti{\discretionary{\char`\~}{\Wrappedafterbreak}{\char`\~}}% 
        } 
        % Some characters . , ; ? ! / are not pygmentized. 
        % This macro makes them "active" and they will insert potential linebreaks 
        \newcommand*\Wrappedbreaksatpunct {% 
            \lccode`\~`\.\lowercase{\def~}{\discretionary{\hbox{\char`\.}}{\Wrappedafterbreak}{\hbox{\char`\.}}}% 
            \lccode`\~`\,\lowercase{\def~}{\discretionary{\hbox{\char`\,}}{\Wrappedafterbreak}{\hbox{\char`\,}}}% 
            \lccode`\~`\;\lowercase{\def~}{\discretionary{\hbox{\char`\;}}{\Wrappedafterbreak}{\hbox{\char`\;}}}% 
            \lccode`\~`\:\lowercase{\def~}{\discretionary{\hbox{\char`\:}}{\Wrappedafterbreak}{\hbox{\char`\:}}}% 
            \lccode`\~`\?\lowercase{\def~}{\discretionary{\hbox{\char`\?}}{\Wrappedafterbreak}{\hbox{\char`\?}}}% 
            \lccode`\~`\!\lowercase{\def~}{\discretionary{\hbox{\char`\!}}{\Wrappedafterbreak}{\hbox{\char`\!}}}% 
            \lccode`\~`\/\lowercase{\def~}{\discretionary{\hbox{\char`\/}}{\Wrappedafterbreak}{\hbox{\char`\/}}}% 
            \catcode`\.\active
            \catcode`\,\active 
            \catcode`\;\active
            \catcode`\:\active
            \catcode`\?\active
            \catcode`\!\active
            \catcode`\/\active 
            \lccode`\~`\~ 	
        }
    \makeatother

    \let\OriginalVerbatim=\Verbatim
    \makeatletter
    \renewcommand{\Verbatim}[1][1]{%
        %\parskip\z@skip
        \sbox\Wrappedcontinuationbox {\Wrappedcontinuationsymbol}%
        \sbox\Wrappedvisiblespacebox {\FV@SetupFont\Wrappedvisiblespace}%
        \def\FancyVerbFormatLine ##1{\hsize\linewidth
            \vtop{\raggedright\hyphenpenalty\z@\exhyphenpenalty\z@
                \doublehyphendemerits\z@\finalhyphendemerits\z@
                \strut ##1\strut}%
        }%
        % If the linebreak is at a space, the latter will be displayed as visible
        % space at end of first line, and a continuation symbol starts next line.
        % Stretch/shrink are however usually zero for typewriter font.
        \def\FV@Space {%
            \nobreak\hskip\z@ plus\fontdimen3\font minus\fontdimen4\font
            \discretionary{\copy\Wrappedvisiblespacebox}{\Wrappedafterbreak}
            {\kern\fontdimen2\font}%
        }%
        
        % Allow breaks at special characters using \PYG... macros.
        \Wrappedbreaksatspecials
        % Breaks at punctuation characters . , ; ? ! and / need catcode=\active 	
        \OriginalVerbatim[#1,codes*=\Wrappedbreaksatpunct]%
    }
    \makeatother

    % Exact colors from NB
    \definecolor{incolor}{HTML}{303F9F}
    \definecolor{outcolor}{HTML}{D84315}
    \definecolor{cellborder}{HTML}{CFCFCF}
    \definecolor{cellbackground}{HTML}{F7F7F7}
    
    % prompt
    \makeatletter
    \newcommand{\boxspacing}{\kern\kvtcb@left@rule\kern\kvtcb@boxsep}
    \makeatother
    \newcommand{\prompt}[4]{
        \ttfamily\llap{{\color{#2}[#3]:\hspace{3pt}#4}}\vspace{-\baselineskip}
    }
    

    
    % Prevent overflowing lines due to hard-to-break entities
    \sloppy 
    % Setup hyperref package
    \hypersetup{
      breaklinks=true,  % so long urls are correctly broken across lines
      colorlinks=true,
      urlcolor=urlcolor,
      linkcolor=linkcolor,
      citecolor=citecolor,
      }
    % Slightly bigger margins than the latex defaults
    
    \geometry{verbose,tmargin=1in,bmargin=1in,lmargin=1in,rmargin=1in}
    
    

\begin{document}
    
    \maketitle
    
    

    
    \hypertarget{creating-a-sentiment-analysis-web-app}{%
\section{Creating a Sentiment Analysis Web
App}\label{creating-a-sentiment-analysis-web-app}}

\hypertarget{using-pytorch-and-sagemaker}{%
\subsection{Using PyTorch and
SageMaker}\label{using-pytorch-and-sagemaker}}

\emph{Deep Learning Nanodegree Program \textbar{} Deployment}

\begin{center}\rule{0.5\linewidth}{0.5pt}\end{center}

Now that we have a basic understanding of how SageMaker works we will
try to use it to construct a complete project from end to end. Our goal
will be to have a simple web page which a user can use to enter a movie
review. The web page will then send the review off to our deployed model
which will predict the sentiment of the entered review.

\hypertarget{instructions}{%
\subsection{Instructions}\label{instructions}}

Some template code has already been provided for you, and you will need
to implement additional functionality to successfully complete this
notebook. You will not need to modify the included code beyond what is
requested. Sections that begin with `\textbf{TODO}' in the header
indicate that you need to complete or implement some portion within
them. Instructions will be provided for each section and the specifics
of the implementation are marked in the code block with a
\texttt{\#\ TODO:\ ...} comment. Please be sure to read the instructions
carefully!

In addition to implementing code, there will be questions for you to
answer which relate to the task and your implementation. Each section
where you will answer a question is preceded by a `\textbf{Question:}'
header. Carefully read each question and provide your answer below the
`\textbf{Answer:}' header by editing the Markdown cell.

\begin{quote}
\textbf{Note}: Code and Markdown cells can be executed using the
\textbf{Shift+Enter} keyboard shortcut. In addition, a cell can be
edited by typically clicking it (double-click for Markdown cells) or by
pressing \textbf{Enter} while it is highlighted.
\end{quote}

\hypertarget{general-outline}{%
\subsection{General Outline}\label{general-outline}}

Recall the general outline for SageMaker projects using a notebook
instance.

\begin{enumerate}
\def\labelenumi{\arabic{enumi}.}
\tightlist
\item
  Download or otherwise retrieve the data.
\item
  Process / Prepare the data.
\item
  Upload the processed data to S3.
\item
  Train a chosen model.
\item
  Test the trained model (typically using a batch transform job).
\item
  Deploy the trained model.
\item
  Use the deployed model.
\end{enumerate}

For this project, you will be following the steps in the general outline
with some modifications.

First, you will not be testing the model in its own step. You will still
be testing the model, however, you will do it by deploying your model
and then using the deployed model by sending the test data to it. One of
the reasons for doing this is so that you can make sure that your
deployed model is working correctly before moving forward.

In addition, you will deploy and use your trained model a second time.
In the second iteration you will customize the way that your trained
model is deployed by including some of your own code. In addition, your
newly deployed model will be used in the sentiment analysis web app.

    \hypertarget{step-1-downloading-the-data}{%
\subsection{Step 1: Downloading the
data}\label{step-1-downloading-the-data}}

As in the XGBoost in SageMaker notebook, we will be using the
\href{http://ai.stanford.edu/~amaas/data/sentiment/}{IMDb dataset}

\begin{quote}
Maas, Andrew L., et
al.~\href{http://ai.stanford.edu/~amaas/data/sentiment/}{Learning Word
Vectors for Sentiment Analysis}. In \emph{Proceedings of the 49th Annual
Meeting of the Association for Computational Linguistics: Human Language
Technologies}. Association for Computational Linguistics, 2011.
\end{quote}

    \begin{tcolorbox}[breakable, size=fbox, boxrule=1pt, pad at break*=1mm,colback=cellbackground, colframe=cellborder]
\prompt{In}{incolor}{2}{\boxspacing}
\begin{Verbatim}[commandchars=\\\{\}]
\PY{o}{\PYZpc{}}\PY{k}{mkdir} ../data
\PY{o}{!}wget \PYZhy{}O ../data/aclImdb\PYZus{}v1.tar.gz http://ai.stanford.edu/\PYZti{}amaas/data/sentiment/aclImdb\PYZus{}v1.tar.gz
\PY{o}{!}tar \PYZhy{}zxf ../data/aclImdb\PYZus{}v1.tar.gz \PYZhy{}C ../data
\end{Verbatim}
\end{tcolorbox}

    \begin{Verbatim}[commandchars=\\\{\}]
mkdir: cannot create directory ‘../data’: File exists
--2020-09-10 18:25:34--
http://ai.stanford.edu/\textasciitilde{}amaas/data/sentiment/aclImdb\_v1.tar.gz
Resolving ai.stanford.edu (ai.stanford.edu){\ldots} 171.64.68.10
Connecting to ai.stanford.edu (ai.stanford.edu)|171.64.68.10|:80{\ldots} connected.
HTTP request sent, awaiting response{\ldots} 200 OK
Length: 84125825 (80M) [application/x-gzip]
Saving to: ‘../data/aclImdb\_v1.tar.gz’

../data/aclImdb\_v1. 100\%[===================>]  80.23M  25.3MB/s    in 3.6s

2020-09-10 18:25:37 (22.2 MB/s) - ‘../data/aclImdb\_v1.tar.gz’ saved
[84125825/84125825]

    \end{Verbatim}

    \hypertarget{step-2-preparing-and-processing-the-data}{%
\subsection{Step 2: Preparing and Processing the
data}\label{step-2-preparing-and-processing-the-data}}

Also, as in the XGBoost notebook, we will be doing some initial data
processing. The first few steps are the same as in the XGBoost example.
To begin with, we will read in each of the reviews and combine them into
a single input structure. Then, we will split the dataset into a
training set and a testing set.

    \begin{tcolorbox}[breakable, size=fbox, boxrule=1pt, pad at break*=1mm,colback=cellbackground, colframe=cellborder]
\prompt{In}{incolor}{3}{\boxspacing}
\begin{Verbatim}[commandchars=\\\{\}]
\PY{k+kn}{import} \PY{n+nn}{os}
\PY{k+kn}{import} \PY{n+nn}{glob}

\PY{k}{def} \PY{n+nf}{read\PYZus{}imdb\PYZus{}data}\PY{p}{(}\PY{n}{data\PYZus{}dir}\PY{o}{=}\PY{l+s+s1}{\PYZsq{}}\PY{l+s+s1}{../data/aclImdb}\PY{l+s+s1}{\PYZsq{}}\PY{p}{)}\PY{p}{:}
    \PY{n}{data} \PY{o}{=} \PY{p}{\PYZob{}}\PY{p}{\PYZcb{}}
    \PY{n}{labels} \PY{o}{=} \PY{p}{\PYZob{}}\PY{p}{\PYZcb{}}
    
    \PY{k}{for} \PY{n}{data\PYZus{}type} \PY{o+ow}{in} \PY{p}{[}\PY{l+s+s1}{\PYZsq{}}\PY{l+s+s1}{train}\PY{l+s+s1}{\PYZsq{}}\PY{p}{,} \PY{l+s+s1}{\PYZsq{}}\PY{l+s+s1}{test}\PY{l+s+s1}{\PYZsq{}}\PY{p}{]}\PY{p}{:}
        \PY{n}{data}\PY{p}{[}\PY{n}{data\PYZus{}type}\PY{p}{]} \PY{o}{=} \PY{p}{\PYZob{}}\PY{p}{\PYZcb{}}
        \PY{n}{labels}\PY{p}{[}\PY{n}{data\PYZus{}type}\PY{p}{]} \PY{o}{=} \PY{p}{\PYZob{}}\PY{p}{\PYZcb{}}
        
        \PY{k}{for} \PY{n}{sentiment} \PY{o+ow}{in} \PY{p}{[}\PY{l+s+s1}{\PYZsq{}}\PY{l+s+s1}{pos}\PY{l+s+s1}{\PYZsq{}}\PY{p}{,} \PY{l+s+s1}{\PYZsq{}}\PY{l+s+s1}{neg}\PY{l+s+s1}{\PYZsq{}}\PY{p}{]}\PY{p}{:}
            \PY{n}{data}\PY{p}{[}\PY{n}{data\PYZus{}type}\PY{p}{]}\PY{p}{[}\PY{n}{sentiment}\PY{p}{]} \PY{o}{=} \PY{p}{[}\PY{p}{]}
            \PY{n}{labels}\PY{p}{[}\PY{n}{data\PYZus{}type}\PY{p}{]}\PY{p}{[}\PY{n}{sentiment}\PY{p}{]} \PY{o}{=} \PY{p}{[}\PY{p}{]}
            
            \PY{n}{path} \PY{o}{=} \PY{n}{os}\PY{o}{.}\PY{n}{path}\PY{o}{.}\PY{n}{join}\PY{p}{(}\PY{n}{data\PYZus{}dir}\PY{p}{,} \PY{n}{data\PYZus{}type}\PY{p}{,} \PY{n}{sentiment}\PY{p}{,} \PY{l+s+s1}{\PYZsq{}}\PY{l+s+s1}{*.txt}\PY{l+s+s1}{\PYZsq{}}\PY{p}{)}
            \PY{n}{files} \PY{o}{=} \PY{n}{glob}\PY{o}{.}\PY{n}{glob}\PY{p}{(}\PY{n}{path}\PY{p}{)}
            
            \PY{k}{for} \PY{n}{f} \PY{o+ow}{in} \PY{n}{files}\PY{p}{:}
                \PY{k}{with} \PY{n+nb}{open}\PY{p}{(}\PY{n}{f}\PY{p}{)} \PY{k}{as} \PY{n}{review}\PY{p}{:}
                    \PY{n}{data}\PY{p}{[}\PY{n}{data\PYZus{}type}\PY{p}{]}\PY{p}{[}\PY{n}{sentiment}\PY{p}{]}\PY{o}{.}\PY{n}{append}\PY{p}{(}\PY{n}{review}\PY{o}{.}\PY{n}{read}\PY{p}{(}\PY{p}{)}\PY{p}{)}
                    \PY{c+c1}{\PYZsh{} Here we represent a positive review by \PYZsq{}1\PYZsq{} and a negative review by \PYZsq{}0\PYZsq{}}
                    \PY{n}{labels}\PY{p}{[}\PY{n}{data\PYZus{}type}\PY{p}{]}\PY{p}{[}\PY{n}{sentiment}\PY{p}{]}\PY{o}{.}\PY{n}{append}\PY{p}{(}\PY{l+m+mi}{1} \PY{k}{if} \PY{n}{sentiment} \PY{o}{==} \PY{l+s+s1}{\PYZsq{}}\PY{l+s+s1}{pos}\PY{l+s+s1}{\PYZsq{}} \PY{k}{else} \PY{l+m+mi}{0}\PY{p}{)}
                    
            \PY{k}{assert} \PY{n+nb}{len}\PY{p}{(}\PY{n}{data}\PY{p}{[}\PY{n}{data\PYZus{}type}\PY{p}{]}\PY{p}{[}\PY{n}{sentiment}\PY{p}{]}\PY{p}{)} \PY{o}{==} \PY{n+nb}{len}\PY{p}{(}\PY{n}{labels}\PY{p}{[}\PY{n}{data\PYZus{}type}\PY{p}{]}\PY{p}{[}\PY{n}{sentiment}\PY{p}{]}\PY{p}{)}\PY{p}{,} \PYZbs{}
                    \PY{l+s+s2}{\PYZdq{}}\PY{l+s+si}{\PYZob{}\PYZcb{}}\PY{l+s+s2}{/}\PY{l+s+si}{\PYZob{}\PYZcb{}}\PY{l+s+s2}{ data size does not match labels size}\PY{l+s+s2}{\PYZdq{}}\PY{o}{.}\PY{n}{format}\PY{p}{(}\PY{n}{data\PYZus{}type}\PY{p}{,} \PY{n}{sentiment}\PY{p}{)}
                
    \PY{k}{return} \PY{n}{data}\PY{p}{,} \PY{n}{labels}
\end{Verbatim}
\end{tcolorbox}

    \begin{tcolorbox}[breakable, size=fbox, boxrule=1pt, pad at break*=1mm,colback=cellbackground, colframe=cellborder]
\prompt{In}{incolor}{4}{\boxspacing}
\begin{Verbatim}[commandchars=\\\{\}]
\PY{n}{data}\PY{p}{,} \PY{n}{labels} \PY{o}{=} \PY{n}{read\PYZus{}imdb\PYZus{}data}\PY{p}{(}\PY{p}{)}
\PY{n+nb}{print}\PY{p}{(}\PY{l+s+s2}{\PYZdq{}}\PY{l+s+s2}{IMDB reviews: train = }\PY{l+s+si}{\PYZob{}\PYZcb{}}\PY{l+s+s2}{ pos / }\PY{l+s+si}{\PYZob{}\PYZcb{}}\PY{l+s+s2}{ neg, test = }\PY{l+s+si}{\PYZob{}\PYZcb{}}\PY{l+s+s2}{ pos / }\PY{l+s+si}{\PYZob{}\PYZcb{}}\PY{l+s+s2}{ neg}\PY{l+s+s2}{\PYZdq{}}\PY{o}{.}\PY{n}{format}\PY{p}{(}
            \PY{n+nb}{len}\PY{p}{(}\PY{n}{data}\PY{p}{[}\PY{l+s+s1}{\PYZsq{}}\PY{l+s+s1}{train}\PY{l+s+s1}{\PYZsq{}}\PY{p}{]}\PY{p}{[}\PY{l+s+s1}{\PYZsq{}}\PY{l+s+s1}{pos}\PY{l+s+s1}{\PYZsq{}}\PY{p}{]}\PY{p}{)}\PY{p}{,} \PY{n+nb}{len}\PY{p}{(}\PY{n}{data}\PY{p}{[}\PY{l+s+s1}{\PYZsq{}}\PY{l+s+s1}{train}\PY{l+s+s1}{\PYZsq{}}\PY{p}{]}\PY{p}{[}\PY{l+s+s1}{\PYZsq{}}\PY{l+s+s1}{neg}\PY{l+s+s1}{\PYZsq{}}\PY{p}{]}\PY{p}{)}\PY{p}{,}
            \PY{n+nb}{len}\PY{p}{(}\PY{n}{data}\PY{p}{[}\PY{l+s+s1}{\PYZsq{}}\PY{l+s+s1}{test}\PY{l+s+s1}{\PYZsq{}}\PY{p}{]}\PY{p}{[}\PY{l+s+s1}{\PYZsq{}}\PY{l+s+s1}{pos}\PY{l+s+s1}{\PYZsq{}}\PY{p}{]}\PY{p}{)}\PY{p}{,} \PY{n+nb}{len}\PY{p}{(}\PY{n}{data}\PY{p}{[}\PY{l+s+s1}{\PYZsq{}}\PY{l+s+s1}{test}\PY{l+s+s1}{\PYZsq{}}\PY{p}{]}\PY{p}{[}\PY{l+s+s1}{\PYZsq{}}\PY{l+s+s1}{neg}\PY{l+s+s1}{\PYZsq{}}\PY{p}{]}\PY{p}{)}\PY{p}{)}\PY{p}{)}
\end{Verbatim}
\end{tcolorbox}

    \begin{Verbatim}[commandchars=\\\{\}]
IMDB reviews: train = 12500 pos / 12500 neg, test = 12500 pos / 12500 neg
    \end{Verbatim}

    Now that we've read the raw training and testing data from the
downloaded dataset, we will combine the positive and negative reviews
and shuffle the resulting records.

    \begin{tcolorbox}[breakable, size=fbox, boxrule=1pt, pad at break*=1mm,colback=cellbackground, colframe=cellborder]
\prompt{In}{incolor}{5}{\boxspacing}
\begin{Verbatim}[commandchars=\\\{\}]
\PY{k+kn}{from} \PY{n+nn}{sklearn}\PY{n+nn}{.}\PY{n+nn}{utils} \PY{k+kn}{import} \PY{n}{shuffle}

\PY{k}{def} \PY{n+nf}{prepare\PYZus{}imdb\PYZus{}data}\PY{p}{(}\PY{n}{data}\PY{p}{,} \PY{n}{labels}\PY{p}{)}\PY{p}{:}
    \PY{l+s+sd}{\PYZdq{}\PYZdq{}\PYZdq{}Prepare training and test sets from IMDb movie reviews.\PYZdq{}\PYZdq{}\PYZdq{}}
    
    \PY{c+c1}{\PYZsh{}Combine positive and negative reviews and labels}
    \PY{n}{data\PYZus{}train} \PY{o}{=} \PY{n}{data}\PY{p}{[}\PY{l+s+s1}{\PYZsq{}}\PY{l+s+s1}{train}\PY{l+s+s1}{\PYZsq{}}\PY{p}{]}\PY{p}{[}\PY{l+s+s1}{\PYZsq{}}\PY{l+s+s1}{pos}\PY{l+s+s1}{\PYZsq{}}\PY{p}{]} \PY{o}{+} \PY{n}{data}\PY{p}{[}\PY{l+s+s1}{\PYZsq{}}\PY{l+s+s1}{train}\PY{l+s+s1}{\PYZsq{}}\PY{p}{]}\PY{p}{[}\PY{l+s+s1}{\PYZsq{}}\PY{l+s+s1}{neg}\PY{l+s+s1}{\PYZsq{}}\PY{p}{]}
    \PY{n}{data\PYZus{}test} \PY{o}{=} \PY{n}{data}\PY{p}{[}\PY{l+s+s1}{\PYZsq{}}\PY{l+s+s1}{test}\PY{l+s+s1}{\PYZsq{}}\PY{p}{]}\PY{p}{[}\PY{l+s+s1}{\PYZsq{}}\PY{l+s+s1}{pos}\PY{l+s+s1}{\PYZsq{}}\PY{p}{]} \PY{o}{+} \PY{n}{data}\PY{p}{[}\PY{l+s+s1}{\PYZsq{}}\PY{l+s+s1}{test}\PY{l+s+s1}{\PYZsq{}}\PY{p}{]}\PY{p}{[}\PY{l+s+s1}{\PYZsq{}}\PY{l+s+s1}{neg}\PY{l+s+s1}{\PYZsq{}}\PY{p}{]}
    \PY{n}{labels\PYZus{}train} \PY{o}{=} \PY{n}{labels}\PY{p}{[}\PY{l+s+s1}{\PYZsq{}}\PY{l+s+s1}{train}\PY{l+s+s1}{\PYZsq{}}\PY{p}{]}\PY{p}{[}\PY{l+s+s1}{\PYZsq{}}\PY{l+s+s1}{pos}\PY{l+s+s1}{\PYZsq{}}\PY{p}{]} \PY{o}{+} \PY{n}{labels}\PY{p}{[}\PY{l+s+s1}{\PYZsq{}}\PY{l+s+s1}{train}\PY{l+s+s1}{\PYZsq{}}\PY{p}{]}\PY{p}{[}\PY{l+s+s1}{\PYZsq{}}\PY{l+s+s1}{neg}\PY{l+s+s1}{\PYZsq{}}\PY{p}{]}
    \PY{n}{labels\PYZus{}test} \PY{o}{=} \PY{n}{labels}\PY{p}{[}\PY{l+s+s1}{\PYZsq{}}\PY{l+s+s1}{test}\PY{l+s+s1}{\PYZsq{}}\PY{p}{]}\PY{p}{[}\PY{l+s+s1}{\PYZsq{}}\PY{l+s+s1}{pos}\PY{l+s+s1}{\PYZsq{}}\PY{p}{]} \PY{o}{+} \PY{n}{labels}\PY{p}{[}\PY{l+s+s1}{\PYZsq{}}\PY{l+s+s1}{test}\PY{l+s+s1}{\PYZsq{}}\PY{p}{]}\PY{p}{[}\PY{l+s+s1}{\PYZsq{}}\PY{l+s+s1}{neg}\PY{l+s+s1}{\PYZsq{}}\PY{p}{]}
    
    \PY{c+c1}{\PYZsh{}Shuffle reviews and corresponding labels within training and test sets}
    \PY{n}{data\PYZus{}train}\PY{p}{,} \PY{n}{labels\PYZus{}train} \PY{o}{=} \PY{n}{shuffle}\PY{p}{(}\PY{n}{data\PYZus{}train}\PY{p}{,} \PY{n}{labels\PYZus{}train}\PY{p}{)}
    \PY{n}{data\PYZus{}test}\PY{p}{,} \PY{n}{labels\PYZus{}test} \PY{o}{=} \PY{n}{shuffle}\PY{p}{(}\PY{n}{data\PYZus{}test}\PY{p}{,} \PY{n}{labels\PYZus{}test}\PY{p}{)}
    
    \PY{c+c1}{\PYZsh{} Return a unified training data, test data, training labels, test labets}
    \PY{k}{return} \PY{n}{data\PYZus{}train}\PY{p}{,} \PY{n}{data\PYZus{}test}\PY{p}{,} \PY{n}{labels\PYZus{}train}\PY{p}{,} \PY{n}{labels\PYZus{}test}
\end{Verbatim}
\end{tcolorbox}

    \begin{tcolorbox}[breakable, size=fbox, boxrule=1pt, pad at break*=1mm,colback=cellbackground, colframe=cellborder]
\prompt{In}{incolor}{6}{\boxspacing}
\begin{Verbatim}[commandchars=\\\{\}]
\PY{n}{train\PYZus{}X}\PY{p}{,} \PY{n}{test\PYZus{}X}\PY{p}{,} \PY{n}{train\PYZus{}y}\PY{p}{,} \PY{n}{test\PYZus{}y} \PY{o}{=} \PY{n}{prepare\PYZus{}imdb\PYZus{}data}\PY{p}{(}\PY{n}{data}\PY{p}{,} \PY{n}{labels}\PY{p}{)}
\PY{n+nb}{print}\PY{p}{(}\PY{l+s+s2}{\PYZdq{}}\PY{l+s+s2}{IMDb reviews (combined): train = }\PY{l+s+si}{\PYZob{}\PYZcb{}}\PY{l+s+s2}{, test = }\PY{l+s+si}{\PYZob{}\PYZcb{}}\PY{l+s+s2}{\PYZdq{}}\PY{o}{.}\PY{n}{format}\PY{p}{(}\PY{n+nb}{len}\PY{p}{(}\PY{n}{train\PYZus{}X}\PY{p}{)}\PY{p}{,} \PY{n+nb}{len}\PY{p}{(}\PY{n}{test\PYZus{}X}\PY{p}{)}\PY{p}{)}\PY{p}{)}
\end{Verbatim}
\end{tcolorbox}

    \begin{Verbatim}[commandchars=\\\{\}]
IMDb reviews (combined): train = 25000, test = 25000
    \end{Verbatim}

    Now that we have our training and testing sets unified and prepared, we
should do a quick check and see an example of the data our model will be
trained on. This is generally a good idea as it allows you to see how
each of the further processing steps affects the reviews and it also
ensures that the data has been loaded correctly.

    \begin{tcolorbox}[breakable, size=fbox, boxrule=1pt, pad at break*=1mm,colback=cellbackground, colframe=cellborder]
\prompt{In}{incolor}{7}{\boxspacing}
\begin{Verbatim}[commandchars=\\\{\}]
\PY{n+nb}{print}\PY{p}{(}\PY{n}{train\PYZus{}X}\PY{p}{[}\PY{l+m+mi}{100}\PY{p}{]}\PY{p}{)}
\PY{n+nb}{print}\PY{p}{(}\PY{n}{train\PYZus{}y}\PY{p}{[}\PY{l+m+mi}{100}\PY{p}{]}\PY{p}{)}
\end{Verbatim}
\end{tcolorbox}

    \begin{Verbatim}[commandchars=\\\{\}]
I read John Everingham's story years ago in Reader's Digest, and I remember
thinking what a great movie it would make. And it probably would have been had
Michael Landon never got his hands on it. As far as I'm concerned, Landon was
one of the worst actors on earth, and his artistic license went way over the
top, similar to his massacre of the "Little House" book series is proof. The
acting, for lack of a better word, is atrocious, the screenplay sloppy, and
there are more close-ups of Landon's puss than should be allowed.<br /><br
/>This movie reflects Everingham's story as much as "Little House On The
Prairie" reflects the books is was "based" on. It's just another vehicle to show
off Landons horrendous hair.
0
    \end{Verbatim}

    The first step in processing the reviews is to make sure that any html
tags that appear should be removed. In addition we wish to tokenize our
input, that way words such as \emph{entertained} and \emph{entertaining}
are considered the same with regard to sentiment analysis.

    \begin{tcolorbox}[breakable, size=fbox, boxrule=1pt, pad at break*=1mm,colback=cellbackground, colframe=cellborder]
\prompt{In}{incolor}{8}{\boxspacing}
\begin{Verbatim}[commandchars=\\\{\}]
\PY{k+kn}{import} \PY{n+nn}{nltk}
\PY{k+kn}{from} \PY{n+nn}{nltk}\PY{n+nn}{.}\PY{n+nn}{corpus} \PY{k+kn}{import} \PY{n}{stopwords}
\PY{k+kn}{from} \PY{n+nn}{nltk}\PY{n+nn}{.}\PY{n+nn}{stem}\PY{n+nn}{.}\PY{n+nn}{porter} \PY{k+kn}{import} \PY{o}{*}

\PY{k+kn}{import} \PY{n+nn}{re}
\PY{k+kn}{from} \PY{n+nn}{bs4} \PY{k+kn}{import} \PY{n}{BeautifulSoup}

\PY{k}{def} \PY{n+nf}{review\PYZus{}to\PYZus{}words}\PY{p}{(}\PY{n}{review}\PY{p}{)}\PY{p}{:}
    \PY{n}{nltk}\PY{o}{.}\PY{n}{download}\PY{p}{(}\PY{l+s+s2}{\PYZdq{}}\PY{l+s+s2}{stopwords}\PY{l+s+s2}{\PYZdq{}}\PY{p}{,} \PY{n}{quiet}\PY{o}{=}\PY{k+kc}{True}\PY{p}{)}
    \PY{n}{stemmer} \PY{o}{=} \PY{n}{PorterStemmer}\PY{p}{(}\PY{p}{)}
    
    \PY{n}{text} \PY{o}{=} \PY{n}{BeautifulSoup}\PY{p}{(}\PY{n}{review}\PY{p}{,} \PY{l+s+s2}{\PYZdq{}}\PY{l+s+s2}{html.parser}\PY{l+s+s2}{\PYZdq{}}\PY{p}{)}\PY{o}{.}\PY{n}{get\PYZus{}text}\PY{p}{(}\PY{p}{)} \PY{c+c1}{\PYZsh{} Remove HTML tags}
    \PY{n}{text} \PY{o}{=} \PY{n}{re}\PY{o}{.}\PY{n}{sub}\PY{p}{(}\PY{l+s+sa}{r}\PY{l+s+s2}{\PYZdq{}}\PY{l+s+s2}{[\PYZca{}a\PYZhy{}zA\PYZhy{}Z0\PYZhy{}9]}\PY{l+s+s2}{\PYZdq{}}\PY{p}{,} \PY{l+s+s2}{\PYZdq{}}\PY{l+s+s2}{ }\PY{l+s+s2}{\PYZdq{}}\PY{p}{,} \PY{n}{text}\PY{o}{.}\PY{n}{lower}\PY{p}{(}\PY{p}{)}\PY{p}{)} \PY{c+c1}{\PYZsh{} Convert to lower case}
    \PY{n}{words} \PY{o}{=} \PY{n}{text}\PY{o}{.}\PY{n}{split}\PY{p}{(}\PY{p}{)} \PY{c+c1}{\PYZsh{} Split string into words}
    \PY{n}{words} \PY{o}{=} \PY{p}{[}\PY{n}{w} \PY{k}{for} \PY{n}{w} \PY{o+ow}{in} \PY{n}{words} \PY{k}{if} \PY{n}{w} \PY{o+ow}{not} \PY{o+ow}{in} \PY{n}{stopwords}\PY{o}{.}\PY{n}{words}\PY{p}{(}\PY{l+s+s2}{\PYZdq{}}\PY{l+s+s2}{english}\PY{l+s+s2}{\PYZdq{}}\PY{p}{)}\PY{p}{]} \PY{c+c1}{\PYZsh{} Remove stopwords}
    \PY{n}{words} \PY{o}{=} \PY{p}{[}\PY{n}{PorterStemmer}\PY{p}{(}\PY{p}{)}\PY{o}{.}\PY{n}{stem}\PY{p}{(}\PY{n}{w}\PY{p}{)} \PY{k}{for} \PY{n}{w} \PY{o+ow}{in} \PY{n}{words}\PY{p}{]} \PY{c+c1}{\PYZsh{} stem}
    
    \PY{k}{return} \PY{n}{words}
\end{Verbatim}
\end{tcolorbox}

    The \texttt{review\_to\_words} method defined above uses
\texttt{BeautifulSoup} to remove any html tags that appear and uses the
\texttt{nltk} package to tokenize the reviews. As a check to ensure we
know how everything is working, try applying \texttt{review\_to\_words}
to one of the reviews in the training set.

    \begin{tcolorbox}[breakable, size=fbox, boxrule=1pt, pad at break*=1mm,colback=cellbackground, colframe=cellborder]
\prompt{In}{incolor}{9}{\boxspacing}
\begin{Verbatim}[commandchars=\\\{\}]
\PY{c+c1}{\PYZsh{} TODO: Apply review\PYZus{}to\PYZus{}words to a review (train\PYZus{}X[100] or any other review)}
\PY{n}{review\PYZus{}to\PYZus{}words}\PY{p}{(}\PY{n}{train\PYZus{}X}\PY{p}{[}\PY{l+m+mi}{100}\PY{p}{]}\PY{p}{)}
\end{Verbatim}
\end{tcolorbox}

            \begin{tcolorbox}[breakable, size=fbox, boxrule=.5pt, pad at break*=1mm, opacityfill=0]
\prompt{Out}{outcolor}{9}{\boxspacing}
\begin{Verbatim}[commandchars=\\\{\}]
['read',
 'john',
 'everingham',
 'stori',
 'year',
 'ago',
 'reader',
 'digest',
 'rememb',
 'think',
 'great',
 'movi',
 'would',
 'make',
 'probabl',
 'would',
 'michael',
 'landon',
 'never',
 'got',
 'hand',
 'far',
 'concern',
 'landon',
 'one',
 'worst',
 'actor',
 'earth',
 'artist',
 'licens',
 'went',
 'way',
 'top',
 'similar',
 'massacr',
 'littl',
 'hous',
 'book',
 'seri',
 'proof',
 'act',
 'lack',
 'better',
 'word',
 'atroci',
 'screenplay',
 'sloppi',
 'close',
 'up',
 'landon',
 'puss',
 'allow',
 'movi',
 'reflect',
 'everingham',
 'stori',
 'much',
 'littl',
 'hous',
 'prairi',
 'reflect',
 'book',
 'base',
 'anoth',
 'vehicl',
 'show',
 'landon',
 'horrend',
 'hair']
\end{Verbatim}
\end{tcolorbox}
        
    \textbf{Question:} Above we mentioned that \texttt{review\_to\_words}
method removes html formatting and allows us to tokenize the words found
in a review, for example, converting \emph{entertained} and
\emph{entertaining} into \emph{entertain} so that they are treated as
though they are the same word. What else, if anything, does this method
do to the input?

    \textbf{Answer:} The method allows the content to be broken down into a
series of bag of words by tokenizing, lemminizing and splitting the
words and sentences. The algorithms also remove the stopwords to make
the data more refined and eliminate any unnecessary terms from the
entire content dataset by finally stemming the words and key phrases.

    The method below applies the \texttt{review\_to\_words} method to each
of the reviews in the training and testing datasets. In addition it
caches the results. This is because performing this processing step can
take a long time. This way if you are unable to complete the notebook in
the current session, you can come back without needing to process the
data a second time.

    \begin{tcolorbox}[breakable, size=fbox, boxrule=1pt, pad at break*=1mm,colback=cellbackground, colframe=cellborder]
\prompt{In}{incolor}{10}{\boxspacing}
\begin{Verbatim}[commandchars=\\\{\}]
\PY{k+kn}{import} \PY{n+nn}{pickle}

\PY{n}{cache\PYZus{}dir} \PY{o}{=} \PY{n}{os}\PY{o}{.}\PY{n}{path}\PY{o}{.}\PY{n}{join}\PY{p}{(}\PY{l+s+s2}{\PYZdq{}}\PY{l+s+s2}{../cache}\PY{l+s+s2}{\PYZdq{}}\PY{p}{,} \PY{l+s+s2}{\PYZdq{}}\PY{l+s+s2}{sentiment\PYZus{}analysis}\PY{l+s+s2}{\PYZdq{}}\PY{p}{)}  \PY{c+c1}{\PYZsh{} where to store cache files}
\PY{n}{os}\PY{o}{.}\PY{n}{makedirs}\PY{p}{(}\PY{n}{cache\PYZus{}dir}\PY{p}{,} \PY{n}{exist\PYZus{}ok}\PY{o}{=}\PY{k+kc}{True}\PY{p}{)}  \PY{c+c1}{\PYZsh{} ensure cache directory exists}

\PY{k}{def} \PY{n+nf}{preprocess\PYZus{}data}\PY{p}{(}\PY{n}{data\PYZus{}train}\PY{p}{,} \PY{n}{data\PYZus{}test}\PY{p}{,} \PY{n}{labels\PYZus{}train}\PY{p}{,} \PY{n}{labels\PYZus{}test}\PY{p}{,}
                    \PY{n}{cache\PYZus{}dir}\PY{o}{=}\PY{n}{cache\PYZus{}dir}\PY{p}{,} \PY{n}{cache\PYZus{}file}\PY{o}{=}\PY{l+s+s2}{\PYZdq{}}\PY{l+s+s2}{preprocessed\PYZus{}data.pkl}\PY{l+s+s2}{\PYZdq{}}\PY{p}{)}\PY{p}{:}
    \PY{l+s+sd}{\PYZdq{}\PYZdq{}\PYZdq{}Convert each review to words; read from cache if available.\PYZdq{}\PYZdq{}\PYZdq{}}

    \PY{c+c1}{\PYZsh{} If cache\PYZus{}file is not None, try to read from it first}
    \PY{n}{cache\PYZus{}data} \PY{o}{=} \PY{k+kc}{None}
    \PY{k}{if} \PY{n}{cache\PYZus{}file} \PY{o+ow}{is} \PY{o+ow}{not} \PY{k+kc}{None}\PY{p}{:}
        \PY{k}{try}\PY{p}{:}
            \PY{k}{with} \PY{n+nb}{open}\PY{p}{(}\PY{n}{os}\PY{o}{.}\PY{n}{path}\PY{o}{.}\PY{n}{join}\PY{p}{(}\PY{n}{cache\PYZus{}dir}\PY{p}{,} \PY{n}{cache\PYZus{}file}\PY{p}{)}\PY{p}{,} \PY{l+s+s2}{\PYZdq{}}\PY{l+s+s2}{rb}\PY{l+s+s2}{\PYZdq{}}\PY{p}{)} \PY{k}{as} \PY{n}{f}\PY{p}{:}
                \PY{n}{cache\PYZus{}data} \PY{o}{=} \PY{n}{pickle}\PY{o}{.}\PY{n}{load}\PY{p}{(}\PY{n}{f}\PY{p}{)}
            \PY{n+nb}{print}\PY{p}{(}\PY{l+s+s2}{\PYZdq{}}\PY{l+s+s2}{Read preprocessed data from cache file:}\PY{l+s+s2}{\PYZdq{}}\PY{p}{,} \PY{n}{cache\PYZus{}file}\PY{p}{)}
        \PY{k}{except}\PY{p}{:}
            \PY{k}{pass}  \PY{c+c1}{\PYZsh{} unable to read from cache, but that\PYZsq{}s okay}
    
    \PY{c+c1}{\PYZsh{} If cache is missing, then do the heavy lifting}
    \PY{k}{if} \PY{n}{cache\PYZus{}data} \PY{o+ow}{is} \PY{k+kc}{None}\PY{p}{:}
        \PY{c+c1}{\PYZsh{} Preprocess training and test data to obtain words for each review}
        \PY{c+c1}{\PYZsh{}words\PYZus{}train = list(map(review\PYZus{}to\PYZus{}words, data\PYZus{}train))}
        \PY{c+c1}{\PYZsh{}words\PYZus{}test = list(map(review\PYZus{}to\PYZus{}words, data\PYZus{}test))}
        \PY{n}{words\PYZus{}train} \PY{o}{=} \PY{p}{[}\PY{n}{review\PYZus{}to\PYZus{}words}\PY{p}{(}\PY{n}{review}\PY{p}{)} \PY{k}{for} \PY{n}{review} \PY{o+ow}{in} \PY{n}{data\PYZus{}train}\PY{p}{]}
        \PY{n}{words\PYZus{}test} \PY{o}{=} \PY{p}{[}\PY{n}{review\PYZus{}to\PYZus{}words}\PY{p}{(}\PY{n}{review}\PY{p}{)} \PY{k}{for} \PY{n}{review} \PY{o+ow}{in} \PY{n}{data\PYZus{}test}\PY{p}{]}
        
        \PY{c+c1}{\PYZsh{} Write to cache file for future runs}
        \PY{k}{if} \PY{n}{cache\PYZus{}file} \PY{o+ow}{is} \PY{o+ow}{not} \PY{k+kc}{None}\PY{p}{:}
            \PY{n}{cache\PYZus{}data} \PY{o}{=} \PY{n+nb}{dict}\PY{p}{(}\PY{n}{words\PYZus{}train}\PY{o}{=}\PY{n}{words\PYZus{}train}\PY{p}{,} \PY{n}{words\PYZus{}test}\PY{o}{=}\PY{n}{words\PYZus{}test}\PY{p}{,}
                              \PY{n}{labels\PYZus{}train}\PY{o}{=}\PY{n}{labels\PYZus{}train}\PY{p}{,} \PY{n}{labels\PYZus{}test}\PY{o}{=}\PY{n}{labels\PYZus{}test}\PY{p}{)}
            \PY{k}{with} \PY{n+nb}{open}\PY{p}{(}\PY{n}{os}\PY{o}{.}\PY{n}{path}\PY{o}{.}\PY{n}{join}\PY{p}{(}\PY{n}{cache\PYZus{}dir}\PY{p}{,} \PY{n}{cache\PYZus{}file}\PY{p}{)}\PY{p}{,} \PY{l+s+s2}{\PYZdq{}}\PY{l+s+s2}{wb}\PY{l+s+s2}{\PYZdq{}}\PY{p}{)} \PY{k}{as} \PY{n}{f}\PY{p}{:}
                \PY{n}{pickle}\PY{o}{.}\PY{n}{dump}\PY{p}{(}\PY{n}{cache\PYZus{}data}\PY{p}{,} \PY{n}{f}\PY{p}{)}
            \PY{n+nb}{print}\PY{p}{(}\PY{l+s+s2}{\PYZdq{}}\PY{l+s+s2}{Wrote preprocessed data to cache file:}\PY{l+s+s2}{\PYZdq{}}\PY{p}{,} \PY{n}{cache\PYZus{}file}\PY{p}{)}
    \PY{k}{else}\PY{p}{:}
        \PY{c+c1}{\PYZsh{} Unpack data loaded from cache file}
        \PY{n}{words\PYZus{}train}\PY{p}{,} \PY{n}{words\PYZus{}test}\PY{p}{,} \PY{n}{labels\PYZus{}train}\PY{p}{,} \PY{n}{labels\PYZus{}test} \PY{o}{=} \PY{p}{(}\PY{n}{cache\PYZus{}data}\PY{p}{[}\PY{l+s+s1}{\PYZsq{}}\PY{l+s+s1}{words\PYZus{}train}\PY{l+s+s1}{\PYZsq{}}\PY{p}{]}\PY{p}{,}
                \PY{n}{cache\PYZus{}data}\PY{p}{[}\PY{l+s+s1}{\PYZsq{}}\PY{l+s+s1}{words\PYZus{}test}\PY{l+s+s1}{\PYZsq{}}\PY{p}{]}\PY{p}{,} \PY{n}{cache\PYZus{}data}\PY{p}{[}\PY{l+s+s1}{\PYZsq{}}\PY{l+s+s1}{labels\PYZus{}train}\PY{l+s+s1}{\PYZsq{}}\PY{p}{]}\PY{p}{,} \PY{n}{cache\PYZus{}data}\PY{p}{[}\PY{l+s+s1}{\PYZsq{}}\PY{l+s+s1}{labels\PYZus{}test}\PY{l+s+s1}{\PYZsq{}}\PY{p}{]}\PY{p}{)}
    
    \PY{k}{return} \PY{n}{words\PYZus{}train}\PY{p}{,} \PY{n}{words\PYZus{}test}\PY{p}{,} \PY{n}{labels\PYZus{}train}\PY{p}{,} \PY{n}{labels\PYZus{}test}
\end{Verbatim}
\end{tcolorbox}

    \begin{tcolorbox}[breakable, size=fbox, boxrule=1pt, pad at break*=1mm,colback=cellbackground, colframe=cellborder]
\prompt{In}{incolor}{11}{\boxspacing}
\begin{Verbatim}[commandchars=\\\{\}]
\PY{c+c1}{\PYZsh{} Preprocess data}
\PY{n}{train\PYZus{}X}\PY{p}{,} \PY{n}{test\PYZus{}X}\PY{p}{,} \PY{n}{train\PYZus{}y}\PY{p}{,} \PY{n}{test\PYZus{}y} \PY{o}{=} \PY{n}{preprocess\PYZus{}data}\PY{p}{(}\PY{n}{train\PYZus{}X}\PY{p}{,} \PY{n}{test\PYZus{}X}\PY{p}{,} \PY{n}{train\PYZus{}y}\PY{p}{,} \PY{n}{test\PYZus{}y}\PY{p}{)}
\end{Verbatim}
\end{tcolorbox}

    \begin{Verbatim}[commandchars=\\\{\}]
Read preprocessed data from cache file: preprocessed\_data.pkl
    \end{Verbatim}

    \hypertarget{transform-the-data}{%
\subsection{Transform the data}\label{transform-the-data}}

In the XGBoost notebook we transformed the data from its word
representation to a bag-of-words feature representation. For the model
we are going to construct in this notebook we will construct a feature
representation which is very similar. To start, we will represent each
word as an integer. Of course, some of the words that appear in the
reviews occur very infrequently and so likely don't contain much
information for the purposes of sentiment analysis. The way we will deal
with this problem is that we will fix the size of our working vocabulary
and we will only include the words that appear most frequently. We will
then combine all of the infrequent words into a single category and, in
our case, we will label it as \texttt{1}.

Since we will be using a recurrent neural network, it will be convenient
if the length of each review is the same. To do this, we will fix a size
for our reviews and then pad short reviews with the category `no word'
(which we will label \texttt{0}) and truncate long reviews.

    \hypertarget{todo-create-a-word-dictionary}{%
\subsubsection{(TODO) Create a word
dictionary}\label{todo-create-a-word-dictionary}}

To begin with, we need to construct a way to map words that appear in
the reviews to integers. Here we fix the size of our vocabulary
(including the `no word' and `infrequent' categories) to be
\texttt{5000} but you may wish to change this to see how it affects the
model.

\begin{quote}
\textbf{TODO:} Complete the implementation for the
\texttt{build\_dict()} method below. Note that even though the
vocab\_size is set to \texttt{5000}, we only want to construct a mapping
for the most frequently appearing \texttt{4998} words. This is because
we want to reserve the special labels \texttt{0} for `no word' and
\texttt{1} for `infrequent word'.
\end{quote}

    \begin{tcolorbox}[breakable, size=fbox, boxrule=1pt, pad at break*=1mm,colback=cellbackground, colframe=cellborder]
\prompt{In}{incolor}{12}{\boxspacing}
\begin{Verbatim}[commandchars=\\\{\}]
\PY{k+kn}{import} \PY{n+nn}{numpy} \PY{k}{as} \PY{n+nn}{np}

\PY{k}{def} \PY{n+nf}{build\PYZus{}dict}\PY{p}{(}\PY{n}{data}\PY{p}{,} \PY{n}{vocab\PYZus{}size} \PY{o}{=} \PY{l+m+mi}{5000}\PY{p}{)}\PY{p}{:}
    \PY{l+s+sd}{\PYZdq{}\PYZdq{}\PYZdq{}Construct and return a dictionary mapping each of the most frequently appearing words to a unique integer.\PYZdq{}\PYZdq{}\PYZdq{}}
    
    \PY{c+c1}{\PYZsh{} TODO: Determine how often each word appears in `data`. Note that `data` is a list of sentences and that a}
    \PY{c+c1}{\PYZsh{}       sentence is a list of words.}
    
    \PY{n}{word\PYZus{}count} \PY{o}{=} \PY{p}{\PYZob{}}\PY{p}{\PYZcb{}} \PY{c+c1}{\PYZsh{} A dict storing the words that appear in the reviews along with how often they occur}
    
    \PY{k}{for} \PY{n}{review} \PY{o+ow}{in} \PY{n}{data}\PY{p}{:}
        \PY{k}{for} \PY{n}{word} \PY{o+ow}{in} \PY{n}{review}\PY{p}{:}
            \PY{k}{if} \PY{n}{word} \PY{o+ow}{in} \PY{n}{word\PYZus{}count}\PY{p}{:}
                \PY{n}{word\PYZus{}count}\PY{p}{[}\PY{n}{word}\PY{p}{]} \PY{o}{+}\PY{o}{=} \PY{l+m+mi}{1}
            \PY{k}{else}\PY{p}{:}
                \PY{n}{word\PYZus{}count}\PY{p}{[}\PY{n}{word}\PY{p}{]} \PY{o}{=} \PY{l+m+mi}{1}
                
                
    \PY{c+c1}{\PYZsh{} TODO: Sort the words found in `data` so that sorted\PYZus{}words[0] is the most frequently appearing word and}
    \PY{c+c1}{\PYZsh{}       sorted\PYZus{}words[\PYZhy{}1] is the least frequently appearing word.}
    
    \PY{n}{sorted\PYZus{}words} \PY{o}{=} \PY{p}{[}\PY{n}{item}\PY{p}{[}\PY{l+m+mi}{0}\PY{p}{]} \PY{k}{for} \PY{n}{item} \PY{o+ow}{in} \PY{n+nb}{sorted}\PY{p}{(}\PY{n}{word\PYZus{}count}\PY{o}{.}\PY{n}{items}\PY{p}{(}\PY{p}{)}\PY{p}{,} \PY{n}{key} \PY{o}{=} \PY{k}{lambda} \PY{n}{x}\PY{p}{:} \PY{n}{x}\PY{p}{[}\PY{l+m+mi}{1}\PY{p}{]}\PY{p}{,} \PY{n}{reverse} \PY{o}{=} \PY{k+kc}{True}\PY{p}{)}\PY{p}{]}
    

    
    \PY{n}{word\PYZus{}dict} \PY{o}{=} \PY{p}{\PYZob{}}\PY{p}{\PYZcb{}} \PY{c+c1}{\PYZsh{} This is what we are building, a dictionary that translates words into integers}
    \PY{k}{for} \PY{n}{idx}\PY{p}{,} \PY{n}{word} \PY{o+ow}{in} \PY{n+nb}{enumerate}\PY{p}{(}\PY{n}{sorted\PYZus{}words}\PY{p}{[}\PY{p}{:}\PY{n}{vocab\PYZus{}size} \PY{o}{\PYZhy{}} \PY{l+m+mi}{2}\PY{p}{]}\PY{p}{)}\PY{p}{:} \PY{c+c1}{\PYZsh{} The \PYZhy{}2 is so that we save room for the \PYZsq{}no word\PYZsq{}}
        \PY{n}{word\PYZus{}dict}\PY{p}{[}\PY{n}{word}\PY{p}{]} \PY{o}{=} \PY{n}{idx} \PY{o}{+} \PY{l+m+mi}{2}                              \PY{c+c1}{\PYZsh{} \PYZsq{}infrequent\PYZsq{} labels}
        
    \PY{k}{return} \PY{n}{word\PYZus{}dict}
\end{Verbatim}
\end{tcolorbox}

    \begin{tcolorbox}[breakable, size=fbox, boxrule=1pt, pad at break*=1mm,colback=cellbackground, colframe=cellborder]
\prompt{In}{incolor}{13}{\boxspacing}
\begin{Verbatim}[commandchars=\\\{\}]
\PY{n}{word\PYZus{}dict} \PY{o}{=} \PY{n}{build\PYZus{}dict}\PY{p}{(}\PY{n}{train\PYZus{}X}\PY{p}{)}
\end{Verbatim}
\end{tcolorbox}

    \textbf{Question:} What are the five most frequently appearing
(tokenized) words in the training set? Does it makes sense that these
words appear frequently in the training set?

    \textbf{Answer:}

    \begin{tcolorbox}[breakable, size=fbox, boxrule=1pt, pad at break*=1mm,colback=cellbackground, colframe=cellborder]
\prompt{In}{incolor}{14}{\boxspacing}
\begin{Verbatim}[commandchars=\\\{\}]
\PY{c+c1}{\PYZsh{} TODO: Use this space to determine the five most frequently appearing words in the training set.}
\PY{n+nb}{print}\PY{p}{(}\PY{l+s+s1}{\PYZsq{}}\PY{l+s+s1}{Five most frequently appearing words in the training set are:}\PY{l+s+se}{\PYZbs{}n}\PY{l+s+s1}{\PYZsq{}}\PY{p}{)}
\PY{n+nb}{print}\PY{p}{(}\PY{n+nb}{sorted}\PY{p}{(}\PY{n}{word\PYZus{}dict}\PY{p}{,} \PY{n}{key} \PY{o}{=} \PY{n}{word\PYZus{}dict}\PY{o}{.}\PY{n}{get}\PY{p}{,} \PY{n}{reverse} \PY{o}{=} \PY{k+kc}{False}\PY{p}{)}\PY{p}{[}\PY{p}{:}\PY{l+m+mi}{5}\PY{p}{]}\PY{p}{)}
\end{Verbatim}
\end{tcolorbox}

    \begin{Verbatim}[commandchars=\\\{\}]
Five most frequently appearing words in the training set are:

['movi', 'film', 'one', 'like', 'time']
    \end{Verbatim}

    \hypertarget{save-word_dict}{%
\subsubsection{\texorpdfstring{Save
\texttt{word\_dict}}{Save word\_dict}}\label{save-word_dict}}

Later on when we construct an endpoint which processes a submitted
review we will need to make use of the \texttt{word\_dict} which we have
created. As such, we will save it to a file now for future use.

    \begin{tcolorbox}[breakable, size=fbox, boxrule=1pt, pad at break*=1mm,colback=cellbackground, colframe=cellborder]
\prompt{In}{incolor}{15}{\boxspacing}
\begin{Verbatim}[commandchars=\\\{\}]
\PY{n}{data\PYZus{}dir} \PY{o}{=} \PY{l+s+s1}{\PYZsq{}}\PY{l+s+s1}{../data/pytorch}\PY{l+s+s1}{\PYZsq{}} \PY{c+c1}{\PYZsh{} The folder we will use for storing data}
\PY{k}{if} \PY{o+ow}{not} \PY{n}{os}\PY{o}{.}\PY{n}{path}\PY{o}{.}\PY{n}{exists}\PY{p}{(}\PY{n}{data\PYZus{}dir}\PY{p}{)}\PY{p}{:} \PY{c+c1}{\PYZsh{} Make sure that the folder exists}
    \PY{n}{os}\PY{o}{.}\PY{n}{makedirs}\PY{p}{(}\PY{n}{data\PYZus{}dir}\PY{p}{)}
\end{Verbatim}
\end{tcolorbox}

    \begin{tcolorbox}[breakable, size=fbox, boxrule=1pt, pad at break*=1mm,colback=cellbackground, colframe=cellborder]
\prompt{In}{incolor}{16}{\boxspacing}
\begin{Verbatim}[commandchars=\\\{\}]
\PY{k}{with} \PY{n+nb}{open}\PY{p}{(}\PY{n}{os}\PY{o}{.}\PY{n}{path}\PY{o}{.}\PY{n}{join}\PY{p}{(}\PY{n}{data\PYZus{}dir}\PY{p}{,} \PY{l+s+s1}{\PYZsq{}}\PY{l+s+s1}{word\PYZus{}dict.pkl}\PY{l+s+s1}{\PYZsq{}}\PY{p}{)}\PY{p}{,} \PY{l+s+s2}{\PYZdq{}}\PY{l+s+s2}{wb}\PY{l+s+s2}{\PYZdq{}}\PY{p}{)} \PY{k}{as} \PY{n}{f}\PY{p}{:}
    \PY{n}{pickle}\PY{o}{.}\PY{n}{dump}\PY{p}{(}\PY{n}{word\PYZus{}dict}\PY{p}{,} \PY{n}{f}\PY{p}{)}
\end{Verbatim}
\end{tcolorbox}

    \hypertarget{transform-the-reviews}{%
\subsubsection{Transform the reviews}\label{transform-the-reviews}}

Now that we have our word dictionary which allows us to transform the
words appearing in the reviews into integers, it is time to make use of
it and convert our reviews to their integer sequence representation,
making sure to pad or truncate to a fixed length, which in our case is
\texttt{500}.

    \begin{tcolorbox}[breakable, size=fbox, boxrule=1pt, pad at break*=1mm,colback=cellbackground, colframe=cellborder]
\prompt{In}{incolor}{17}{\boxspacing}
\begin{Verbatim}[commandchars=\\\{\}]
\PY{k}{def} \PY{n+nf}{convert\PYZus{}and\PYZus{}pad}\PY{p}{(}\PY{n}{word\PYZus{}dict}\PY{p}{,} \PY{n}{sentence}\PY{p}{,} \PY{n}{pad}\PY{o}{=}\PY{l+m+mi}{500}\PY{p}{)}\PY{p}{:}
    \PY{n}{NOWORD} \PY{o}{=} \PY{l+m+mi}{0} \PY{c+c1}{\PYZsh{} We will use 0 to represent the \PYZsq{}no word\PYZsq{} category}
    \PY{n}{INFREQ} \PY{o}{=} \PY{l+m+mi}{1} \PY{c+c1}{\PYZsh{} and we use 1 to represent the infrequent words, i.e., words not appearing in word\PYZus{}dict}
    
    \PY{n}{working\PYZus{}sentence} \PY{o}{=} \PY{p}{[}\PY{n}{NOWORD}\PY{p}{]} \PY{o}{*} \PY{n}{pad}
    
    \PY{k}{for} \PY{n}{word\PYZus{}index}\PY{p}{,} \PY{n}{word} \PY{o+ow}{in} \PY{n+nb}{enumerate}\PY{p}{(}\PY{n}{sentence}\PY{p}{[}\PY{p}{:}\PY{n}{pad}\PY{p}{]}\PY{p}{)}\PY{p}{:}
        \PY{k}{if} \PY{n}{word} \PY{o+ow}{in} \PY{n}{word\PYZus{}dict}\PY{p}{:}
            \PY{n}{working\PYZus{}sentence}\PY{p}{[}\PY{n}{word\PYZus{}index}\PY{p}{]} \PY{o}{=} \PY{n}{word\PYZus{}dict}\PY{p}{[}\PY{n}{word}\PY{p}{]}
        \PY{k}{else}\PY{p}{:}
            \PY{n}{working\PYZus{}sentence}\PY{p}{[}\PY{n}{word\PYZus{}index}\PY{p}{]} \PY{o}{=} \PY{n}{INFREQ}
            
    \PY{k}{return} \PY{n}{working\PYZus{}sentence}\PY{p}{,} \PY{n+nb}{min}\PY{p}{(}\PY{n+nb}{len}\PY{p}{(}\PY{n}{sentence}\PY{p}{)}\PY{p}{,} \PY{n}{pad}\PY{p}{)}

\PY{k}{def} \PY{n+nf}{convert\PYZus{}and\PYZus{}pad\PYZus{}data}\PY{p}{(}\PY{n}{word\PYZus{}dict}\PY{p}{,} \PY{n}{data}\PY{p}{,} \PY{n}{pad}\PY{o}{=}\PY{l+m+mi}{500}\PY{p}{)}\PY{p}{:}
    \PY{n}{result} \PY{o}{=} \PY{p}{[}\PY{p}{]}
    \PY{n}{lengths} \PY{o}{=} \PY{p}{[}\PY{p}{]}
    
    \PY{k}{for} \PY{n}{sentence} \PY{o+ow}{in} \PY{n}{data}\PY{p}{:}
        \PY{n}{converted}\PY{p}{,} \PY{n}{leng} \PY{o}{=} \PY{n}{convert\PYZus{}and\PYZus{}pad}\PY{p}{(}\PY{n}{word\PYZus{}dict}\PY{p}{,} \PY{n}{sentence}\PY{p}{,} \PY{n}{pad}\PY{p}{)}
        \PY{n}{result}\PY{o}{.}\PY{n}{append}\PY{p}{(}\PY{n}{converted}\PY{p}{)}
        \PY{n}{lengths}\PY{o}{.}\PY{n}{append}\PY{p}{(}\PY{n}{leng}\PY{p}{)}
        
    \PY{k}{return} \PY{n}{np}\PY{o}{.}\PY{n}{array}\PY{p}{(}\PY{n}{result}\PY{p}{)}\PY{p}{,} \PY{n}{np}\PY{o}{.}\PY{n}{array}\PY{p}{(}\PY{n}{lengths}\PY{p}{)}
\end{Verbatim}
\end{tcolorbox}

    \begin{tcolorbox}[breakable, size=fbox, boxrule=1pt, pad at break*=1mm,colback=cellbackground, colframe=cellborder]
\prompt{In}{incolor}{18}{\boxspacing}
\begin{Verbatim}[commandchars=\\\{\}]
\PY{n}{train\PYZus{}X}\PY{p}{,} \PY{n}{train\PYZus{}X\PYZus{}len} \PY{o}{=} \PY{n}{convert\PYZus{}and\PYZus{}pad\PYZus{}data}\PY{p}{(}\PY{n}{word\PYZus{}dict}\PY{p}{,} \PY{n}{train\PYZus{}X}\PY{p}{)}
\PY{n}{test\PYZus{}X}\PY{p}{,} \PY{n}{test\PYZus{}X\PYZus{}len} \PY{o}{=} \PY{n}{convert\PYZus{}and\PYZus{}pad\PYZus{}data}\PY{p}{(}\PY{n}{word\PYZus{}dict}\PY{p}{,} \PY{n}{test\PYZus{}X}\PY{p}{)}
\end{Verbatim}
\end{tcolorbox}

    As a quick check to make sure that things are working as intended, check
to see what one of the reviews in the training set looks like after
having been processeed. Does this look reasonable? What is the length of
a review in the training set?

    \begin{tcolorbox}[breakable, size=fbox, boxrule=1pt, pad at break*=1mm,colback=cellbackground, colframe=cellborder]
\prompt{In}{incolor}{19}{\boxspacing}
\begin{Verbatim}[commandchars=\\\{\}]
\PY{c+c1}{\PYZsh{} Use this cell to examine one of the processed reviews to make sure everything is working as intended.}
\PY{n+nb}{print}\PY{p}{(}\PY{l+s+s1}{\PYZsq{}}\PY{l+s+s1}{length of a review }\PY{l+s+s1}{\PYZsq{}}\PY{p}{,} \PY{n+nb}{len}\PY{p}{(}\PY{n}{train\PYZus{}X}\PY{p}{[}\PY{l+m+mi}{500}\PY{p}{]}\PY{p}{)}\PY{p}{)}
\PY{n+nb}{print}\PY{p}{(}\PY{n}{train\PYZus{}X}\PY{p}{[}\PY{l+m+mi}{500}\PY{p}{]}\PY{p}{)}
\end{Verbatim}
\end{tcolorbox}

    \begin{Verbatim}[commandchars=\\\{\}]
length of a review  500
[   1 1120  578  238 1598  362  340    6 1011 1008   61    1 1771  477
 3028  104  373   97 2477  315    1  749  152  459  445   65 1385 2265
  149  291 1294  112  316 1729   27 1837  394  386 1655 1063 4009    1
 4128  315   97  570 1977    1  222   46   18 3658  358  296  495  196
 1154  627 4408    1 1142 2070 1310 4302  597 2657   10  851  801  910
    1  107  162  157 1650 1358  659   80  690    1   65 1125 1209 3010
  597    1    8 2313    4 2203 3028    1 1576  592   98 1124  755  945
    1   42 2694 1161  214 1650  760    1    4 1161  472 3679   64 3010
  561    1 2028    1 1058  489 1473 1620 3386    1    1    1  531 2694
  214 1650 1241   63 3738    1 3028  477  471 2144    1 3386  167    1
  167 4848    1 2907  303 2683 2106 3866 2151   40    8  705 2038  284
 1030 3386   11 1310  157  705  997  548   55  122  917 1195   42  970
 2115 1310  765 3386  574    1  133    1 3386  517 1027 1577  863 1733
 1170 1310    1 2768  629 1338 3386 2046 1599    1    1 3010  471 1650
   61 3277    1 4798 1650  609 4848 1310 2368 1310  469  416 3386   61
 1650 1033  273 1578    1  910  200 3087 1310 1646    1 2180 1616    1
  133 3010 1310 1646 3010 1126   57  648  471 3087    1  214   14 1310
 1650  750    1  350  419  152 3255  442 2258  593  489  451  128  450
  115 2970  406 3028 1132  939   97  104  840 1310 1650 4848    9 2424
 4733   92  370  213 1970 2083 2503 2481  271 1080 4848   40  489  911
 1852 1599   23  979   27   23   37    4   36 1224  981    6  185    1
    1   27 2969  383  773   72 3018  175  320  200  442  466    1 4158
 4820  114  915 1713   93 1592  152  177 1260  911 3070  930 1553 3481
  172    4 2390 1453  598 2016    1 2907   60  391 2189  553 1239 1964
   81 1310  972 1011  888  687  404 1978  211    6  421   75 1655 2747
 2330   49 1978  411   59  472 1011 1771  486 1003 2823 4848 2324    1
  231   14 3629 4848 1018  399    1  555  478 1650 1125 4553  201  568
 1771 1255  140 4038  435    1 4472    1  383 2897  128 1219 2657  659
  313   81    5 2907  286  149    1  127   26  315 1978    0    0    0
    0    0    0    0    0    0    0    0    0    0    0    0    0    0
    0    0    0    0    0    0    0    0    0    0    0    0    0    0
    0    0    0    0    0    0    0    0    0    0    0    0    0    0
    0    0    0    0    0    0    0    0    0    0    0    0    0    0
    0    0    0    0    0    0    0    0    0    0    0    0    0    0
    0    0    0    0    0    0    0    0    0    0    0    0    0    0
    0    0    0    0    0    0    0    0    0    0]
    \end{Verbatim}

    \textbf{Question:} In the cells above we use the
\texttt{preprocess\_data} and \texttt{convert\_and\_pad\_data} methods
to process both the training and testing set. Why or why not might this
be a problem?

    \textbf{Answer:} The preprocess\_data method helps in saving the outputs
to a file and can reduce the overall time for getting the data back
after the notebook is restarted. But it can take a long time for
preprocessing if the data is highly unclean and requires more stringent
tranformations. It may also be time consuming if the data is highly
unstructured or filled with stopwords, leaving very little information
to be processed and analyzed for sentiments. The convert\_and\_pad\_data
method helps in creating a constant length for all the reviews through
the LSTM classifier which needs a constant batch size. One particular
problem can arise from making all the vectors the same size as there is
a considerable increase in the memory usage for the system, since it may
not be necessary to keep a constant length for the reviews, given their
non-uniform shape and size.

    \hypertarget{step-3-upload-the-data-to-s3}{%
\subsection{Step 3: Upload the data to
S3}\label{step-3-upload-the-data-to-s3}}

As in the XGBoost notebook, we will need to upload the training dataset
to S3 in order for our training code to access it. For now we will save
it locally and we will upload to S3 later on.

\hypertarget{save-the-processed-training-dataset-locally}{%
\subsubsection{Save the processed training dataset
locally}\label{save-the-processed-training-dataset-locally}}

It is important to note the format of the data that we are saving as we
will need to know it when we write the training code. In our case, each
row of the dataset has the form \texttt{label}, \texttt{length},
\texttt{review{[}500{]}} where \texttt{review{[}500{]}} is a sequence of
\texttt{500} integers representing the words in the review.

    \begin{tcolorbox}[breakable, size=fbox, boxrule=1pt, pad at break*=1mm,colback=cellbackground, colframe=cellborder]
\prompt{In}{incolor}{20}{\boxspacing}
\begin{Verbatim}[commandchars=\\\{\}]
\PY{k+kn}{import} \PY{n+nn}{pandas} \PY{k}{as} \PY{n+nn}{pd}
    
\PY{n}{pd}\PY{o}{.}\PY{n}{concat}\PY{p}{(}\PY{p}{[}\PY{n}{pd}\PY{o}{.}\PY{n}{DataFrame}\PY{p}{(}\PY{n}{train\PYZus{}y}\PY{p}{)}\PY{p}{,} \PY{n}{pd}\PY{o}{.}\PY{n}{DataFrame}\PY{p}{(}\PY{n}{train\PYZus{}X\PYZus{}len}\PY{p}{)}\PY{p}{,} \PY{n}{pd}\PY{o}{.}\PY{n}{DataFrame}\PY{p}{(}\PY{n}{train\PYZus{}X}\PY{p}{)}\PY{p}{]}\PY{p}{,} \PY{n}{axis}\PY{o}{=}\PY{l+m+mi}{1}\PY{p}{)} \PYZbs{}
        \PY{o}{.}\PY{n}{to\PYZus{}csv}\PY{p}{(}\PY{n}{os}\PY{o}{.}\PY{n}{path}\PY{o}{.}\PY{n}{join}\PY{p}{(}\PY{n}{data\PYZus{}dir}\PY{p}{,} \PY{l+s+s1}{\PYZsq{}}\PY{l+s+s1}{train.csv}\PY{l+s+s1}{\PYZsq{}}\PY{p}{)}\PY{p}{,} \PY{n}{header}\PY{o}{=}\PY{k+kc}{False}\PY{p}{,} \PY{n}{index}\PY{o}{=}\PY{k+kc}{False}\PY{p}{)}
\end{Verbatim}
\end{tcolorbox}

    \hypertarget{uploading-the-training-data}{%
\subsubsection{Uploading the training
data}\label{uploading-the-training-data}}

Next, we need to upload the training data to the SageMaker default S3
bucket so that we can provide access to it while training our model.

    \begin{tcolorbox}[breakable, size=fbox, boxrule=1pt, pad at break*=1mm,colback=cellbackground, colframe=cellborder]
\prompt{In}{incolor}{21}{\boxspacing}
\begin{Verbatim}[commandchars=\\\{\}]
\PY{k+kn}{import} \PY{n+nn}{sagemaker}

\PY{n}{sagemaker\PYZus{}session} \PY{o}{=} \PY{n}{sagemaker}\PY{o}{.}\PY{n}{Session}\PY{p}{(}\PY{p}{)}

\PY{n}{bucket} \PY{o}{=} \PY{n}{sagemaker\PYZus{}session}\PY{o}{.}\PY{n}{default\PYZus{}bucket}\PY{p}{(}\PY{p}{)}
\PY{n}{prefix} \PY{o}{=} \PY{l+s+s1}{\PYZsq{}}\PY{l+s+s1}{sagemaker/sentiment\PYZus{}rnn}\PY{l+s+s1}{\PYZsq{}}

\PY{n}{role} \PY{o}{=} \PY{n}{sagemaker}\PY{o}{.}\PY{n}{get\PYZus{}execution\PYZus{}role}\PY{p}{(}\PY{p}{)}
\end{Verbatim}
\end{tcolorbox}

    \begin{tcolorbox}[breakable, size=fbox, boxrule=1pt, pad at break*=1mm,colback=cellbackground, colframe=cellborder]
\prompt{In}{incolor}{22}{\boxspacing}
\begin{Verbatim}[commandchars=\\\{\}]
\PY{n}{input\PYZus{}data} \PY{o}{=} \PY{n}{sagemaker\PYZus{}session}\PY{o}{.}\PY{n}{upload\PYZus{}data}\PY{p}{(}\PY{n}{path}\PY{o}{=}\PY{n}{data\PYZus{}dir}\PY{p}{,} \PY{n}{bucket}\PY{o}{=}\PY{n}{bucket}\PY{p}{,} \PY{n}{key\PYZus{}prefix}\PY{o}{=}\PY{n}{prefix}\PY{p}{)}
\end{Verbatim}
\end{tcolorbox}

    \textbf{NOTE:} The cell above uploads the entire contents of our data
directory. This includes the \texttt{word\_dict.pkl} file. This is
fortunate as we will need this later on when we create an endpoint that
accepts an arbitrary review. For now, we will just take note of the fact
that it resides in the data directory (and so also in the S3 training
bucket) and that we will need to make sure it gets saved in the model
directory.

    \hypertarget{step-4-build-and-train-the-pytorch-model}{%
\subsection{Step 4: Build and Train the PyTorch
Model}\label{step-4-build-and-train-the-pytorch-model}}

In the XGBoost notebook we discussed what a model is in the SageMaker
framework. In particular, a model comprises three objects

\begin{itemize}
\tightlist
\item
  Model Artifacts,
\item
  Training Code, and
\item
  Inference Code,
\end{itemize}

each of which interact with one another. In the XGBoost example we used
training and inference code that was provided by Amazon. Here we will
still be using containers provided by Amazon with the added benefit of
being able to include our own custom code.

We will start by implementing our own neural network in PyTorch along
with a training script. For the purposes of this project we have
provided the necessary model object in the \texttt{model.py} file,
inside of the \texttt{train} folder. You can see the provided
implementation by running the cell below.

    \begin{tcolorbox}[breakable, size=fbox, boxrule=1pt, pad at break*=1mm,colback=cellbackground, colframe=cellborder]
\prompt{In}{incolor}{23}{\boxspacing}
\begin{Verbatim}[commandchars=\\\{\}]
\PY{o}{!}pygmentize train/model.py
\end{Verbatim}
\end{tcolorbox}

    \begin{Verbatim}[commandchars=\\\{\}]
\textcolor{ansi-blue}{import}
\textcolor{ansi-cyan}{\underline{torch}}\textcolor{ansi-cyan}{\underline{.}}\textcolor{ansi-cyan}{\underline{nn}}
\textcolor{ansi-blue}{as} \textcolor{ansi-cyan}{\underline{nn}}

\textcolor{ansi-blue}{class} \textcolor{ansi-green}{\underline{LSTMClassifier}}(nn.Module):
    \textcolor{ansi-yellow}{"""}
\textcolor{ansi-yellow}{    This is the simple RNN model we will be using to perform Sentiment
Analysis.}
\textcolor{ansi-yellow}{    """}

    \textcolor{ansi-blue}{def} \textcolor{ansi-green}{\_\_init\_\_}(\textcolor{ansi-cyan}{self},
embedding\_dim, hidden\_dim, vocab\_size):
        \textcolor{ansi-yellow}{"""}
\textcolor{ansi-yellow}{        Initialize the model by settingg up the various layers.}
\textcolor{ansi-yellow}{        """}
        \textcolor{ansi-cyan}{super}(LSTMClassifier,
\textcolor{ansi-cyan}{self}).\textcolor{ansi-green}{\_\_init\_\_}()

        \textcolor{ansi-cyan}{self}.embedding = nn.Embedding(vocab\_size, embedding\_dim,
padding\_idx=\textcolor{ansi-blue}{0})
        \textcolor{ansi-cyan}{self}.lstm = nn.LSTM(embedding\_dim, hidden\_dim)
        \textcolor{ansi-cyan}{self}.dense = nn.Linear(in\_features=hidden\_dim,
out\_features=\textcolor{ansi-blue}{1})
        \textcolor{ansi-cyan}{self}.sig = nn.Sigmoid()

        \textcolor{ansi-cyan}{self}.word\_dict = \textcolor{ansi-blue}{None}

    \textcolor{ansi-blue}{def} \textcolor{ansi-green}{forward}(\textcolor{ansi-cyan}{self}, x):
        \textcolor{ansi-yellow}{"""}
\textcolor{ansi-yellow}{        Perform a forward pass of our model on some input.}
\textcolor{ansi-yellow}{        """}
        x = x.t()
        lengths = x[\textcolor{ansi-blue}{0},:]
        reviews = x[\textcolor{ansi-blue}{1}:,:]
        embeds = \textcolor{ansi-cyan}{self}.embedding(reviews)
        lstm\_out, \_ = \textcolor{ansi-cyan}{self}.lstm(embeds)
        out = \textcolor{ansi-cyan}{self}.dense(lstm\_out)
        out = out[lengths - \textcolor{ansi-blue}{1},
\textcolor{ansi-cyan}{range}(\textcolor{ansi-cyan}{len}(lengths))]
        \textcolor{ansi-blue}{return} \textcolor{ansi-cyan}{self}.sig(out.squeeze())
    \end{Verbatim}

    The important takeaway from the implementation provided is that there
are three parameters that we may wish to tweak to improve the
performance of our model. These are the embedding dimension, the hidden
dimension and the size of the vocabulary. We will likely want to make
these parameters configurable in the training script so that if we wish
to modify them we do not need to modify the script itself. We will see
how to do this later on. To start we will write some of the training
code in the notebook so that we can more easily diagnose any issues that
arise.

First we will load a small portion of the training data set to use as a
sample. It would be very time consuming to try and train the model
completely in the notebook as we do not have access to a gpu and the
compute instance that we are using is not particularly powerful.
However, we can work on a small bit of the data to get a feel for how
our training script is behaving.

    \begin{tcolorbox}[breakable, size=fbox, boxrule=1pt, pad at break*=1mm,colback=cellbackground, colframe=cellborder]
\prompt{In}{incolor}{24}{\boxspacing}
\begin{Verbatim}[commandchars=\\\{\}]
\PY{k+kn}{import} \PY{n+nn}{torch}
\PY{k+kn}{import} \PY{n+nn}{torch}\PY{n+nn}{.}\PY{n+nn}{utils}\PY{n+nn}{.}\PY{n+nn}{data}

\PY{c+c1}{\PYZsh{} Read in only the first 250 rows}
\PY{n}{train\PYZus{}sample} \PY{o}{=} \PY{n}{pd}\PY{o}{.}\PY{n}{read\PYZus{}csv}\PY{p}{(}\PY{n}{os}\PY{o}{.}\PY{n}{path}\PY{o}{.}\PY{n}{join}\PY{p}{(}\PY{n}{data\PYZus{}dir}\PY{p}{,} \PY{l+s+s1}{\PYZsq{}}\PY{l+s+s1}{train.csv}\PY{l+s+s1}{\PYZsq{}}\PY{p}{)}\PY{p}{,} \PY{n}{header}\PY{o}{=}\PY{k+kc}{None}\PY{p}{,} \PY{n}{names}\PY{o}{=}\PY{k+kc}{None}\PY{p}{,} \PY{n}{nrows}\PY{o}{=}\PY{l+m+mi}{250}\PY{p}{)}

\PY{c+c1}{\PYZsh{} Turn the input pandas dataframe into tensors}
\PY{n}{train\PYZus{}sample\PYZus{}y} \PY{o}{=} \PY{n}{torch}\PY{o}{.}\PY{n}{from\PYZus{}numpy}\PY{p}{(}\PY{n}{train\PYZus{}sample}\PY{p}{[}\PY{p}{[}\PY{l+m+mi}{0}\PY{p}{]}\PY{p}{]}\PY{o}{.}\PY{n}{values}\PY{p}{)}\PY{o}{.}\PY{n}{float}\PY{p}{(}\PY{p}{)}\PY{o}{.}\PY{n}{squeeze}\PY{p}{(}\PY{p}{)}
\PY{n}{train\PYZus{}sample\PYZus{}X} \PY{o}{=} \PY{n}{torch}\PY{o}{.}\PY{n}{from\PYZus{}numpy}\PY{p}{(}\PY{n}{train\PYZus{}sample}\PY{o}{.}\PY{n}{drop}\PY{p}{(}\PY{p}{[}\PY{l+m+mi}{0}\PY{p}{]}\PY{p}{,} \PY{n}{axis}\PY{o}{=}\PY{l+m+mi}{1}\PY{p}{)}\PY{o}{.}\PY{n}{values}\PY{p}{)}\PY{o}{.}\PY{n}{long}\PY{p}{(}\PY{p}{)}

\PY{c+c1}{\PYZsh{} Build the dataset}
\PY{n}{train\PYZus{}sample\PYZus{}ds} \PY{o}{=} \PY{n}{torch}\PY{o}{.}\PY{n}{utils}\PY{o}{.}\PY{n}{data}\PY{o}{.}\PY{n}{TensorDataset}\PY{p}{(}\PY{n}{train\PYZus{}sample\PYZus{}X}\PY{p}{,} \PY{n}{train\PYZus{}sample\PYZus{}y}\PY{p}{)}
\PY{c+c1}{\PYZsh{} Build the dataloader}
\PY{n}{train\PYZus{}sample\PYZus{}dl} \PY{o}{=} \PY{n}{torch}\PY{o}{.}\PY{n}{utils}\PY{o}{.}\PY{n}{data}\PY{o}{.}\PY{n}{DataLoader}\PY{p}{(}\PY{n}{train\PYZus{}sample\PYZus{}ds}\PY{p}{,} \PY{n}{batch\PYZus{}size}\PY{o}{=}\PY{l+m+mi}{50}\PY{p}{)}
\end{Verbatim}
\end{tcolorbox}

    \hypertarget{todo-writing-the-training-method}{%
\subsubsection{(TODO) Writing the training
method}\label{todo-writing-the-training-method}}

Next we need to write the training code itself. This should be very
similar to training methods that you have written before to train
PyTorch models. We will leave any difficult aspects such as model saving
/ loading and parameter loading until a little later.

    \begin{tcolorbox}[breakable, size=fbox, boxrule=1pt, pad at break*=1mm,colback=cellbackground, colframe=cellborder]
\prompt{In}{incolor}{25}{\boxspacing}
\begin{Verbatim}[commandchars=\\\{\}]
\PY{k}{def} \PY{n+nf}{train}\PY{p}{(}\PY{n}{model}\PY{p}{,} \PY{n}{train\PYZus{}loader}\PY{p}{,} \PY{n}{epochs}\PY{p}{,} \PY{n}{optimizer}\PY{p}{,} \PY{n}{loss\PYZus{}fn}\PY{p}{,} \PY{n}{device}\PY{p}{)}\PY{p}{:}
    \PY{k}{for} \PY{n}{epoch} \PY{o+ow}{in} \PY{n+nb}{range}\PY{p}{(}\PY{l+m+mi}{1}\PY{p}{,} \PY{n}{epochs} \PY{o}{+} \PY{l+m+mi}{1}\PY{p}{)}\PY{p}{:}
        \PY{n}{model}\PY{o}{.}\PY{n}{train}\PY{p}{(}\PY{p}{)}
        \PY{n}{total\PYZus{}loss} \PY{o}{=} \PY{l+m+mi}{0}
        \PY{k}{for} \PY{n}{batch} \PY{o+ow}{in} \PY{n}{train\PYZus{}loader}\PY{p}{:}         
            \PY{n}{batch\PYZus{}X}\PY{p}{,} \PY{n}{batch\PYZus{}y} \PY{o}{=} \PY{n}{batch}
            
            \PY{n}{batch\PYZus{}X} \PY{o}{=} \PY{n}{batch\PYZus{}X}\PY{o}{.}\PY{n}{to}\PY{p}{(}\PY{n}{device}\PY{p}{)}
            \PY{n}{batch\PYZus{}y} \PY{o}{=} \PY{n}{batch\PYZus{}y}\PY{o}{.}\PY{n}{to}\PY{p}{(}\PY{n}{device}\PY{p}{)}
            
            \PY{c+c1}{\PYZsh{} TODO: Complete this train method to train the model provided.}
            \PY{n}{optimizer}\PY{o}{.}\PY{n}{zero\PYZus{}grad}\PY{p}{(}\PY{p}{)}
            \PY{n}{out} \PY{o}{=} \PY{n}{model}\PY{o}{.}\PY{n}{forward}\PY{p}{(}\PY{n}{batch\PYZus{}X}\PY{p}{)}
            \PY{n}{loss} \PY{o}{=} \PY{n}{loss\PYZus{}fn}\PY{p}{(}\PY{n}{out}\PY{p}{,} \PY{n}{batch\PYZus{}y}\PY{p}{)}
            \PY{n}{loss}\PY{o}{.}\PY{n}{backward}\PY{p}{(}\PY{p}{)}
            \PY{n}{optimizer}\PY{o}{.}\PY{n}{step}\PY{p}{(}\PY{p}{)}
            
            \PY{n}{total\PYZus{}loss} \PY{o}{+}\PY{o}{=} \PY{n}{loss}\PY{o}{.}\PY{n}{data}\PY{o}{.}\PY{n}{item}\PY{p}{(}\PY{p}{)}
        \PY{n+nb}{print}\PY{p}{(}\PY{l+s+s2}{\PYZdq{}}\PY{l+s+s2}{Epoch: }\PY{l+s+si}{\PYZob{}\PYZcb{}}\PY{l+s+s2}{, BCELoss: }\PY{l+s+si}{\PYZob{}\PYZcb{}}\PY{l+s+s2}{\PYZdq{}}\PY{o}{.}\PY{n}{format}\PY{p}{(}\PY{n}{epoch}\PY{p}{,} \PY{n}{total\PYZus{}loss} \PY{o}{/} \PY{n+nb}{len}\PY{p}{(}\PY{n}{train\PYZus{}loader}\PY{p}{)}\PY{p}{)}\PY{p}{)}
\end{Verbatim}
\end{tcolorbox}

    Supposing we have the training method above, we will test that it is
working by writing a bit of code in the notebook that executes our
training method on the small sample training set that we loaded earlier.
The reason for doing this in the notebook is so that we have an
opportunity to fix any errors that arise early when they are easier to
diagnose.

    \begin{tcolorbox}[breakable, size=fbox, boxrule=1pt, pad at break*=1mm,colback=cellbackground, colframe=cellborder]
\prompt{In}{incolor}{26}{\boxspacing}
\begin{Verbatim}[commandchars=\\\{\}]
\PY{k+kn}{import} \PY{n+nn}{torch}\PY{n+nn}{.}\PY{n+nn}{optim} \PY{k}{as} \PY{n+nn}{optim}
\PY{k+kn}{from} \PY{n+nn}{train}\PY{n+nn}{.}\PY{n+nn}{model} \PY{k+kn}{import} \PY{n}{LSTMClassifier}

\PY{n}{device} \PY{o}{=} \PY{n}{torch}\PY{o}{.}\PY{n}{device}\PY{p}{(}\PY{l+s+s2}{\PYZdq{}}\PY{l+s+s2}{cuda}\PY{l+s+s2}{\PYZdq{}} \PY{k}{if} \PY{n}{torch}\PY{o}{.}\PY{n}{cuda}\PY{o}{.}\PY{n}{is\PYZus{}available}\PY{p}{(}\PY{p}{)} \PY{k}{else} \PY{l+s+s2}{\PYZdq{}}\PY{l+s+s2}{cpu}\PY{l+s+s2}{\PYZdq{}}\PY{p}{)}
\PY{n}{model} \PY{o}{=} \PY{n}{LSTMClassifier}\PY{p}{(}\PY{l+m+mi}{32}\PY{p}{,} \PY{l+m+mi}{100}\PY{p}{,} \PY{l+m+mi}{5000}\PY{p}{)}\PY{o}{.}\PY{n}{to}\PY{p}{(}\PY{n}{device}\PY{p}{)}
\PY{n}{optimizer} \PY{o}{=} \PY{n}{optim}\PY{o}{.}\PY{n}{Adam}\PY{p}{(}\PY{n}{model}\PY{o}{.}\PY{n}{parameters}\PY{p}{(}\PY{p}{)}\PY{p}{)}
\PY{n}{loss\PYZus{}fn} \PY{o}{=} \PY{n}{torch}\PY{o}{.}\PY{n}{nn}\PY{o}{.}\PY{n}{BCELoss}\PY{p}{(}\PY{p}{)}

\PY{n}{train}\PY{p}{(}\PY{n}{model}\PY{p}{,} \PY{n}{train\PYZus{}sample\PYZus{}dl}\PY{p}{,} \PY{l+m+mi}{5}\PY{p}{,} \PY{n}{optimizer}\PY{p}{,} \PY{n}{loss\PYZus{}fn}\PY{p}{,} \PY{n}{device}\PY{p}{)}
\end{Verbatim}
\end{tcolorbox}

    \begin{Verbatim}[commandchars=\\\{\}]
Epoch: 1, BCELoss: 0.6949911832809448
Epoch: 2, BCELoss: 0.6822074055671692
Epoch: 3, BCELoss: 0.6717594981193542
Epoch: 4, BCELoss: 0.6617854237556458
Epoch: 5, BCELoss: 0.6516794562339783
    \end{Verbatim}

    In order to construct a PyTorch model using SageMaker we must provide
SageMaker with a training script. We may optionally include a directory
which will be copied to the container and from which our training code
will be run. When the training container is executed it will check the
uploaded directory (if there is one) for a \texttt{requirements.txt}
file and install any required Python libraries, after which the training
script will be run.

    \hypertarget{todo-training-the-model}{%
\subsubsection{(TODO) Training the
model}\label{todo-training-the-model}}

When a PyTorch model is constructed in SageMaker, an entry point must be
specified. This is the Python file which will be executed when the model
is trained. Inside of the \texttt{train} directory is a file called
\texttt{train.py} which has been provided and which contains most of the
necessary code to train our model. The only thing that is missing is the
implementation of the \texttt{train()} method which you wrote earlier in
this notebook.

\textbf{TODO}: Copy the \texttt{train()} method written above and paste
it into the \texttt{train/train.py} file where required.

The way that SageMaker passes hyperparameters to the training script is
by way of arguments. These arguments can then be parsed and used in the
training script. To see how this is done take a look at the provided
\texttt{train/train.py} file.

    \begin{tcolorbox}[breakable, size=fbox, boxrule=1pt, pad at break*=1mm,colback=cellbackground, colframe=cellborder]
\prompt{In}{incolor}{27}{\boxspacing}
\begin{Verbatim}[commandchars=\\\{\}]
\PY{k+kn}{from} \PY{n+nn}{sagemaker}\PY{n+nn}{.}\PY{n+nn}{pytorch} \PY{k+kn}{import} \PY{n}{PyTorch}

\PY{n}{estimator} \PY{o}{=} \PY{n}{PyTorch}\PY{p}{(}\PY{n}{entry\PYZus{}point}\PY{o}{=}\PY{l+s+s2}{\PYZdq{}}\PY{l+s+s2}{train.py}\PY{l+s+s2}{\PYZdq{}}\PY{p}{,}
                    \PY{n}{source\PYZus{}dir}\PY{o}{=}\PY{l+s+s2}{\PYZdq{}}\PY{l+s+s2}{train}\PY{l+s+s2}{\PYZdq{}}\PY{p}{,}
                    \PY{n}{role}\PY{o}{=}\PY{n}{role}\PY{p}{,}
                    \PY{n}{framework\PYZus{}version}\PY{o}{=}\PY{l+s+s1}{\PYZsq{}}\PY{l+s+s1}{0.4.0}\PY{l+s+s1}{\PYZsq{}}\PY{p}{,}
                    \PY{n}{train\PYZus{}instance\PYZus{}count}\PY{o}{=}\PY{l+m+mi}{1}\PY{p}{,}
                    \PY{n}{train\PYZus{}instance\PYZus{}type}\PY{o}{=}\PY{l+s+s1}{\PYZsq{}}\PY{l+s+s1}{ml.p2.xlarge}\PY{l+s+s1}{\PYZsq{}}\PY{p}{,}
                    \PY{n}{hyperparameters}\PY{o}{=}\PY{p}{\PYZob{}}
                        \PY{l+s+s1}{\PYZsq{}}\PY{l+s+s1}{epochs}\PY{l+s+s1}{\PYZsq{}}\PY{p}{:} \PY{l+m+mi}{10}\PY{p}{,}
                        \PY{l+s+s1}{\PYZsq{}}\PY{l+s+s1}{hidden\PYZus{}dim}\PY{l+s+s1}{\PYZsq{}}\PY{p}{:} \PY{l+m+mi}{200}\PY{p}{,}
                    \PY{p}{\PYZcb{}}\PY{p}{)}
\end{Verbatim}
\end{tcolorbox}

    \begin{tcolorbox}[breakable, size=fbox, boxrule=1pt, pad at break*=1mm,colback=cellbackground, colframe=cellborder]
\prompt{In}{incolor}{28}{\boxspacing}
\begin{Verbatim}[commandchars=\\\{\}]
\PY{n}{estimator}\PY{o}{.}\PY{n}{fit}\PY{p}{(}\PY{p}{\PYZob{}}\PY{l+s+s1}{\PYZsq{}}\PY{l+s+s1}{training}\PY{l+s+s1}{\PYZsq{}}\PY{p}{:} \PY{n}{input\PYZus{}data}\PY{p}{\PYZcb{}}\PY{p}{)}
\end{Verbatim}
\end{tcolorbox}

    \begin{Verbatim}[commandchars=\\\{\}]
'create\_image\_uri' will be deprecated in favor of 'ImageURIProvider' class in
SageMaker Python SDK v2.
's3\_input' class will be renamed to 'TrainingInput' in SageMaker Python SDK v2.
'create\_image\_uri' will be deprecated in favor of 'ImageURIProvider' class in
SageMaker Python SDK v2.
    \end{Verbatim}

    \begin{Verbatim}[commandchars=\\\{\}]
2020-09-10 18:31:58 Starting - Starting the training job{\ldots}
2020-09-10 18:32:00 Starting - Launching requested ML instances{\ldots}
2020-09-10 18:33:23 Starting - Preparing the instances for training{\ldots}
2020-09-10 18:35:11 Downloading - Downloading input data{\ldots}
2020-09-10 18:35:40 Training - Downloading the training image..\textcolor{ansi-blue}{bash: cannot
set terminal process group (-1): Inappropriate ioctl for device}
\textcolor{ansi-blue}{bash: no job control in this shell}
\textcolor{ansi-blue}{2020-09-10 18:36:02,046 sagemaker-containers INFO     Imported framework
sagemaker\_pytorch\_container.training}
\textcolor{ansi-blue}{2020-09-10 18:36:02,071 sagemaker\_pytorch\_container.training INFO     Block
until all host DNS lookups succeed.}
\textcolor{ansi-blue}{2020-09-10 18:36:02,075 sagemaker\_pytorch\_container.training INFO
Invoking user training script.}
\textcolor{ansi-blue}{2020-09-10 18:36:02,337 sagemaker-containers INFO     Module train does not
provide a setup.py. }
\textcolor{ansi-blue}{Generating setup.py}
\textcolor{ansi-blue}{2020-09-10 18:36:02,338 sagemaker-containers INFO     Generating
setup.cfg}
\textcolor{ansi-blue}{2020-09-10 18:36:02,338 sagemaker-containers INFO     Generating
MANIFEST.in}
\textcolor{ansi-blue}{2020-09-10 18:36:02,338 sagemaker-containers INFO     Installing module
with the following command:}
\textcolor{ansi-blue}{/usr/bin/python -m pip install -U . -r requirements.txt}
\textcolor{ansi-blue}{Processing /opt/ml/code}
\textcolor{ansi-blue}{Collecting pandas (from -r requirements.txt (line 1))
  Downloading https://files.pythonhosted.org/packages/74/24/0cdbf8907e1e3bc5a8da
03345c23cbed7044330bb8f73bb12e711a640a00/pandas-0.24.2-cp35-cp35m-manylinux1\_x86
\_64.whl (10.0MB)}
\textcolor{ansi-blue}{Collecting numpy (from -r requirements.txt (line 2))}
\textcolor{ansi-blue}{  Downloading https://files.pythonhosted.org/packages/b5/36/88723426b4ff576
809fec7d73594fe17a35c27f8d01f93637637a29ae25b/numpy-1.18.5-cp35-cp35m-manylinux1
\_x86\_64.whl (19.9MB)}
\textcolor{ansi-blue}{Collecting nltk (from -r requirements.txt (line 3))
  Downloading https://files.pythonhosted.org/packages/92/75/ce35194d8e3022203cca
0d2f896dbb88689f9b3fce8e9f9cff942913519d/nltk-3.5.zip (1.4MB)}
\textcolor{ansi-blue}{Collecting beautifulsoup4 (from -r requirements.txt (line 4))
  Downloading https://files.pythonhosted.org/packages/66/25/ff030e2437265616a1e9
b25ccc864e0371a0bc3adb7c5a404fd661c6f4f6/beautifulsoup4-4.9.1-py3-none-any.whl
(115kB)}
\textcolor{ansi-blue}{Collecting html5lib (from -r requirements.txt (line 5))
  Downloading https://files.pythonhosted.org/packages/6c/dd/a834df6482147d48e225
a49515aabc28974ad5a4ca3215c18a882565b028/html5lib-1.1-py2.py3-none-any.whl
(112kB)}
\textcolor{ansi-blue}{Collecting pytz>=2011k (from pandas->-r requirements.txt (line 1))
  Downloading https://files.pythonhosted.org/packages/4f/a4/879454d49688e2fad93e
59d7d4efda580b783c745fd2ec2a3adf87b0808d/pytz-2020.1-py2.py3-none-any.whl
(510kB)}
\textcolor{ansi-blue}{Requirement already satisfied, skipping upgrade: python-dateutil>=2.5.0 in
/usr/local/lib/python3.5/dist-packages (from pandas->-r requirements.txt (line
1)) (2.7.5)}
\textcolor{ansi-blue}{Requirement already satisfied, skipping upgrade: click in
/usr/local/lib/python3.5/dist-packages (from nltk->-r requirements.txt (line 3))
(7.0)}
\textcolor{ansi-blue}{Collecting joblib (from nltk->-r requirements.txt (line 3))
  Downloading https://files.pythonhosted.org/packages/28/5c/cf6a2b65a321c4a209ef
cdf64c2689efae2cb62661f8f6f4bb28547cf1bf/joblib-0.14.1-py2.py3-none-any.whl
(294kB)}
\textcolor{ansi-blue}{Collecting regex (from nltk->-r requirements.txt (line 3))
  Downloading https://files.pythonhosted.org/packages/09/c3/ddaa87500f31ed86290e
3d014c0302a51fde28d7139eda0b5f33733726db/regex-2020.7.14.tar.gz (690kB)}
\textcolor{ansi-blue}{Collecting tqdm (from nltk->-r requirements.txt (line 3))
  Downloading https://files.pythonhosted.org/packages/28/7e/281edb5bc3274dfb894d
90f4dbacfceaca381c2435ec6187a2c6f329aed7/tqdm-4.48.2-py2.py3-none-any.whl
(68kB)}
\textcolor{ansi-blue}{Collecting soupsieve>1.2 (from beautifulsoup4->-r requirements.txt (line
4))
  Downloading https://files.pythonhosted.org/packages/6f/8f/457f4a5390eeae1cc3ae
ab89deb7724c965be841ffca6cfca9197482e470/soupsieve-2.0.1-py3-none-any.whl}
\textcolor{ansi-blue}{Requirement already satisfied, skipping upgrade: six>=1.9 in
/usr/local/lib/python3.5/dist-packages (from html5lib->-r requirements.txt (line
5)) (1.11.0)}
\textcolor{ansi-blue}{Collecting webencodings (from html5lib->-r requirements.txt (line 5))
  Downloading https://files.pythonhosted.org/packages/f4/24/2a3e3df732393fed8b3e
bf2ec078f05546de641fe1b667ee316ec1dcf3b7/webencodings-0.5.1-py2.py3-none-
any.whl}
\textcolor{ansi-blue}{Building wheels for collected packages: nltk, train, regex
  Running setup.py bdist\_wheel for nltk: started}
\textcolor{ansi-blue}{  Running setup.py bdist\_wheel for nltk: finished with status 'done'
  Stored in directory: /root/.cache/pip/wheels/ae/8c/3f/b1fe0ba04555b08b57ab52ab
7f86023639a526d8bc8d384306
  Running setup.py bdist\_wheel for train: started
  Running setup.py bdist\_wheel for train: finished with status 'done'
  Stored in directory: /tmp/pip-ephem-wheel-cache-o34sj5df/wheels/35/24/16/37574
d11bf9bde50616c67372a334f94fa8356bc7164af8ca3
  Running setup.py bdist\_wheel for regex: started}

2020-09-10 18:36:01 Training - Training image download completed. Training in
progress.\textcolor{ansi-blue}{  Running setup.py bdist\_wheel for regex: finished with status
'done'
  Stored in directory: /root/.cache/pip/wheels/53/55/dc/e17fa4568958f4c53be34b65
e474a1327b64641f65df379ec3}
\textcolor{ansi-blue}{Successfully built nltk train regex}
\textcolor{ansi-blue}{Installing collected packages: numpy, pytz, pandas, joblib, regex, tqdm,
nltk, soupsieve, beautifulsoup4, webencodings, html5lib, train
  Found existing installation: numpy 1.15.4
    Uninstalling numpy-1.15.4:
      Successfully uninstalled numpy-1.15.4}
\textcolor{ansi-blue}{Successfully installed beautifulsoup4-4.9.1 html5lib-1.1 joblib-0.14.1
nltk-3.5 numpy-1.18.5 pandas-0.24.2 pytz-2020.1 regex-2020.7.14 soupsieve-2.0.1
tqdm-4.48.2 train-1.0.0 webencodings-0.5.1}
\textcolor{ansi-blue}{You are using pip version 18.1, however version 20.2.3 is available.}
\textcolor{ansi-blue}{You should consider upgrading via the 'pip install --upgrade pip'
command.}
\textcolor{ansi-blue}{2020-09-10 18:36:24,591 sagemaker-containers INFO     Invoking user script
}
\textcolor{ansi-blue}{Training Env:
}
\textcolor{ansi-blue}{\{
    "module\_name": "train",
    "job\_name": "sagemaker-pytorch-2020-09-10-18-31-58-277",
    "num\_cpus": 4,
    "hosts": [
        "algo-1"
    ],
    "channel\_input\_dirs": \{
        "training": "/opt/ml/input/data/training"
    \},
    "network\_interface\_name": "eth0",
    "output\_dir": "/opt/ml/output",
    "current\_host": "algo-1",
    "input\_config\_dir": "/opt/ml/input/config",
    "user\_entry\_point": "train.py",
    "framework\_module": "sagemaker\_pytorch\_container.training:main",
    "resource\_config": \{
        "network\_interface\_name": "eth0",
        "hosts": [
            "algo-1"
        ],
        "current\_host": "algo-1"
    \},
    "log\_level": 20,
    "hyperparameters": \{
        "epochs": 10,
        "hidden\_dim": 200
    \},
    "output\_intermediate\_dir": "/opt/ml/output/intermediate",
    "additional\_framework\_parameters": \{\},
    "input\_data\_config": \{
        "training": \{
            "S3DistributionType": "FullyReplicated",
            "RecordWrapperType": "None",
            "TrainingInputMode": "File"
        \}
    \},
    "model\_dir": "/opt/ml/model",
    "input\_dir": "/opt/ml/input",
    "num\_gpus": 1,
    "output\_data\_dir": "/opt/ml/output/data",
    "module\_dir": "s3://sagemaker-us-east-2-423904393887/sagemaker-
pytorch-2020-09-10-18-31-58-277/source/sourcedir.tar.gz"}
\textcolor{ansi-blue}{\}
}
\textcolor{ansi-blue}{Environment variables:
}
\textcolor{ansi-blue}{SM\_LOG\_LEVEL=20}
\textcolor{ansi-blue}{SM\_USER\_ARGS=["--epochs","10","--hidden\_dim","200"]}
\textcolor{ansi-blue}{SM\_INPUT\_DIR=/opt/ml/input}
\textcolor{ansi-blue}{SM\_HOSTS=["algo-1"]}
\textcolor{ansi-blue}{SM\_HP\_HIDDEN\_DIM=200}
\textcolor{ansi-blue}{SM\_MODULE\_DIR=s3://sagemaker-us-east-2-423904393887/sagemaker-
pytorch-2020-09-10-18-31-58-277/source/sourcedir.tar.gz}
\textcolor{ansi-blue}{SM\_FRAMEWORK\_MODULE=sagemaker\_pytorch\_container.training:main}
\textcolor{ansi-blue}{SM\_FRAMEWORK\_PARAMS=\{\}}
\textcolor{ansi-blue}{SM\_NUM\_GPUS=1}
\textcolor{ansi-blue}{SM\_CHANNEL\_TRAINING=/opt/ml/input/data/training}
\textcolor{ansi-blue}{SM\_OUTPUT\_DATA\_DIR=/opt/ml/output/data}
\textcolor{ansi-blue}{SM\_OUTPUT\_DIR=/opt/ml/output}
\textcolor{ansi-blue}{SM\_NUM\_CPUS=4}
\textcolor{ansi-blue}{SM\_MODULE\_NAME=train}
\textcolor{ansi-blue}{SM\_INPUT\_DATA\_CONFIG=\{"training":\{"RecordWrapperType":"None","S3Distributio
nType":"FullyReplicated","TrainingInputMode":"File"\}\}}
\textcolor{ansi-blue}{SM\_MODEL\_DIR=/opt/ml/model}
\textcolor{ansi-blue}{SM\_CHANNELS=["training"]}
\textcolor{ansi-blue}{SM\_NETWORK\_INTERFACE\_NAME=eth0}
\textcolor{ansi-blue}{PYTHONPATH=/usr/local/bin:/usr/lib/python35.zip:/usr/lib/python3.5:/usr/lib
/python3.5/plat-x86\_64-linux-gnu:/usr/lib/python3.5/lib-
dynload:/usr/local/lib/python3.5/dist-packages:/usr/lib/python3/dist-
packages}
\textcolor{ansi-blue}{SM\_HP\_EPOCHS=10}
\textcolor{ansi-blue}{SM\_USER\_ENTRY\_POINT=train.py}
\textcolor{ansi-blue}{SM\_TRAINING\_ENV=\{"additional\_framework\_parameters":\{\},"channel\_input\_dirs":
\{"training":"/opt/ml/input/data/training"\},"current\_host":"algo-1","framework\_mo
dule":"sagemaker\_pytorch\_container.training:main","hosts":["algo-1"],"hyperparam
eters":\{"epochs":10,"hidden\_dim":200\},"input\_config\_dir":"/opt/ml/input/config",
"input\_data\_config":\{"training":\{"RecordWrapperType":"None","S3DistributionType"
:"FullyReplicated","TrainingInputMode":"File"\}\},"input\_dir":"/opt/ml/input","job
\_name":"sagemaker-pytorch-2020-09-10-18-31-58-277","log\_level":20,"model\_dir":"/
opt/ml/model","module\_dir":"s3://sagemaker-us-east-2-423904393887/sagemaker-pyto
rch-2020-09-10-18-31-58-277/source/sourcedir.tar.gz","module\_name":"train","netw
ork\_interface\_name":"eth0","num\_cpus":4,"num\_gpus":1,"output\_data\_dir":"/opt/ml/
output/data","output\_dir":"/opt/ml/output","output\_intermediate\_dir":"/opt/ml/ou
tput/intermediate","resource\_config":\{"current\_host":"algo-1","hosts":["algo-1"]
,"network\_interface\_name":"eth0"\},"user\_entry\_point":"train.py"\}}
\textcolor{ansi-blue}{SM\_OUTPUT\_INTERMEDIATE\_DIR=/opt/ml/output/intermediate}
\textcolor{ansi-blue}{SM\_CURRENT\_HOST=algo-1}
\textcolor{ansi-blue}{SM\_HPS=\{"epochs":10,"hidden\_dim":200\}}
\textcolor{ansi-blue}{SM\_RESOURCE\_CONFIG=\{"current\_host":"algo-1","hosts":["algo-1"],"network\_int
erface\_name":"eth0"\}}
\textcolor{ansi-blue}{SM\_INPUT\_CONFIG\_DIR=/opt/ml/input/config
}
\textcolor{ansi-blue}{Invoking script with the following command:
}
\textcolor{ansi-blue}{/usr/bin/python -m train --epochs 10 --hidden\_dim 200

}
\textcolor{ansi-blue}{Using device cuda.}
\textcolor{ansi-blue}{Get train data loader.}
\textcolor{ansi-blue}{Model loaded with embedding\_dim 32, hidden\_dim 200, vocab\_size 5000.}
\textcolor{ansi-blue}{2020-09-10 18:36:33,537 sagemaker-containers INFO     Reporting training
SUCCESS}

2020-09-10 18:36:43 Uploading - Uploading generated training model
2020-09-10 18:36:43 Completed - Training job completed
Training seconds: 92
Billable seconds: 92
    \end{Verbatim}

    \hypertarget{step-5-testing-the-model}{%
\subsection{Step 5: Testing the model}\label{step-5-testing-the-model}}

As mentioned at the top of this notebook, we will be testing this model
by first deploying it and then sending the testing data to the deployed
endpoint. We will do this so that we can make sure that the deployed
model is working correctly.

\hypertarget{step-6-deploy-the-model-for-testing}{%
\subsection{Step 6: Deploy the model for
testing}\label{step-6-deploy-the-model-for-testing}}

Now that we have trained our model, we would like to test it to see how
it performs. Currently our model takes input of the form
\texttt{review\_length,\ review{[}500{]}} where \texttt{review{[}500{]}}
is a sequence of \texttt{500} integers which describe the words present
in the review, encoded using \texttt{word\_dict}. Fortunately for us,
SageMaker provides built-in inference code for models with simple inputs
such as this.

There is one thing that we need to provide, however, and that is a
function which loads the saved model. This function must be called
\texttt{model\_fn()} and takes as its only parameter a path to the
directory where the model artifacts are stored. This function must also
be present in the python file which we specified as the entry point. In
our case the model loading function has been provided and so no changes
need to be made.

\textbf{NOTE}: When the built-in inference code is run it must import
the \texttt{model\_fn()} method from the \texttt{train.py} file. This is
why the training code is wrapped in a main guard ( ie,
\texttt{if\ \_\_name\_\_\ ==\ \textquotesingle{}\_\_main\_\_\textquotesingle{}:}
)

Since we don't need to change anything in the code that was uploaded
during training, we can simply deploy the current model as-is.

\textbf{NOTE:} When deploying a model you are asking SageMaker to launch
an compute instance that will wait for data to be sent to it. As a
result, this compute instance will continue to run until \emph{you} shut
it down. This is important to know since the cost of a deployed endpoint
depends on how long it has been running for.

In other words \textbf{If you are no longer using a deployed endpoint,
shut it down!}

\textbf{TODO:} Deploy the trained model.

    \begin{tcolorbox}[breakable, size=fbox, boxrule=1pt, pad at break*=1mm,colback=cellbackground, colframe=cellborder]
\prompt{In}{incolor}{29}{\boxspacing}
\begin{Verbatim}[commandchars=\\\{\}]
\PY{c+c1}{\PYZsh{} TODO: Deploy the trained model}
\PY{n}{predictor} \PY{o}{=} \PY{n}{estimator}\PY{o}{.}\PY{n}{deploy}\PY{p}{(}\PY{n}{initial\PYZus{}instance\PYZus{}count}\PY{o}{=}\PY{l+m+mi}{1}\PY{p}{,} \PY{n}{instance\PYZus{}type}\PY{o}{=}\PY{l+s+s1}{\PYZsq{}}\PY{l+s+s1}{ml.p2.xlarge}\PY{l+s+s1}{\PYZsq{}}\PY{p}{)}
\end{Verbatim}
\end{tcolorbox}

    \begin{Verbatim}[commandchars=\\\{\}]
Parameter image will be renamed to image\_uri in SageMaker Python SDK v2.
'create\_image\_uri' will be deprecated in favor of 'ImageURIProvider' class in
SageMaker Python SDK v2.
    \end{Verbatim}

    \begin{Verbatim}[commandchars=\\\{\}]
-------------------!
    \end{Verbatim}

    \hypertarget{step-7---use-the-model-for-testing}{%
\subsection{Step 7 - Use the model for
testing}\label{step-7---use-the-model-for-testing}}

Once deployed, we can read in the test data and send it off to our
deployed model to get some results. Once we collect all of the results
we can determine how accurate our model is.

    \begin{tcolorbox}[breakable, size=fbox, boxrule=1pt, pad at break*=1mm,colback=cellbackground, colframe=cellborder]
\prompt{In}{incolor}{30}{\boxspacing}
\begin{Verbatim}[commandchars=\\\{\}]
\PY{n}{test\PYZus{}X} \PY{o}{=} \PY{n}{pd}\PY{o}{.}\PY{n}{concat}\PY{p}{(}\PY{p}{[}\PY{n}{pd}\PY{o}{.}\PY{n}{DataFrame}\PY{p}{(}\PY{n}{test\PYZus{}X\PYZus{}len}\PY{p}{)}\PY{p}{,} \PY{n}{pd}\PY{o}{.}\PY{n}{DataFrame}\PY{p}{(}\PY{n}{test\PYZus{}X}\PY{p}{)}\PY{p}{]}\PY{p}{,} \PY{n}{axis}\PY{o}{=}\PY{l+m+mi}{1}\PY{p}{)}
\end{Verbatim}
\end{tcolorbox}

    \begin{tcolorbox}[breakable, size=fbox, boxrule=1pt, pad at break*=1mm,colback=cellbackground, colframe=cellborder]
\prompt{In}{incolor}{31}{\boxspacing}
\begin{Verbatim}[commandchars=\\\{\}]
\PY{c+c1}{\PYZsh{} We split the data into chunks and send each chunk seperately, accumulating the results.}

\PY{k}{def} \PY{n+nf}{predict}\PY{p}{(}\PY{n}{data}\PY{p}{,} \PY{n}{rows}\PY{o}{=}\PY{l+m+mi}{512}\PY{p}{)}\PY{p}{:}
    \PY{n}{split\PYZus{}array} \PY{o}{=} \PY{n}{np}\PY{o}{.}\PY{n}{array\PYZus{}split}\PY{p}{(}\PY{n}{data}\PY{p}{,} \PY{n+nb}{int}\PY{p}{(}\PY{n}{data}\PY{o}{.}\PY{n}{shape}\PY{p}{[}\PY{l+m+mi}{0}\PY{p}{]} \PY{o}{/} \PY{n+nb}{float}\PY{p}{(}\PY{n}{rows}\PY{p}{)} \PY{o}{+} \PY{l+m+mi}{1}\PY{p}{)}\PY{p}{)}
    \PY{n}{predictions} \PY{o}{=} \PY{n}{np}\PY{o}{.}\PY{n}{array}\PY{p}{(}\PY{p}{[}\PY{p}{]}\PY{p}{)}
    \PY{k}{for} \PY{n}{array} \PY{o+ow}{in} \PY{n}{split\PYZus{}array}\PY{p}{:}
        \PY{n}{predictions} \PY{o}{=} \PY{n}{np}\PY{o}{.}\PY{n}{append}\PY{p}{(}\PY{n}{predictions}\PY{p}{,} \PY{n}{predictor}\PY{o}{.}\PY{n}{predict}\PY{p}{(}\PY{n}{array}\PY{p}{)}\PY{p}{)}
    
    \PY{k}{return} \PY{n}{predictions}
\end{Verbatim}
\end{tcolorbox}

    \begin{tcolorbox}[breakable, size=fbox, boxrule=1pt, pad at break*=1mm,colback=cellbackground, colframe=cellborder]
\prompt{In}{incolor}{32}{\boxspacing}
\begin{Verbatim}[commandchars=\\\{\}]
\PY{n}{predictions} \PY{o}{=} \PY{n}{predict}\PY{p}{(}\PY{n}{test\PYZus{}X}\PY{o}{.}\PY{n}{values}\PY{p}{)}
\PY{n}{predictions} \PY{o}{=} \PY{p}{[}\PY{n+nb}{round}\PY{p}{(}\PY{n}{num}\PY{p}{)} \PY{k}{for} \PY{n}{num} \PY{o+ow}{in} \PY{n}{predictions}\PY{p}{]}
\end{Verbatim}
\end{tcolorbox}

    \begin{tcolorbox}[breakable, size=fbox, boxrule=1pt, pad at break*=1mm,colback=cellbackground, colframe=cellborder]
\prompt{In}{incolor}{33}{\boxspacing}
\begin{Verbatim}[commandchars=\\\{\}]
\PY{k+kn}{from} \PY{n+nn}{sklearn}\PY{n+nn}{.}\PY{n+nn}{metrics} \PY{k+kn}{import} \PY{n}{accuracy\PYZus{}score}
\PY{n}{accuracy\PYZus{}score}\PY{p}{(}\PY{n}{test\PYZus{}y}\PY{p}{,} \PY{n}{predictions}\PY{p}{)}
\end{Verbatim}
\end{tcolorbox}

            \begin{tcolorbox}[breakable, size=fbox, boxrule=.5pt, pad at break*=1mm, opacityfill=0]
\prompt{Out}{outcolor}{33}{\boxspacing}
\begin{Verbatim}[commandchars=\\\{\}]
0.50548
\end{Verbatim}
\end{tcolorbox}
        
    \textbf{Question:} How does this model compare to the XGBoost model you
created earlier? Why might these two models perform differently on this
dataset? Which do \emph{you} think is better for sentiment analysis?

    \textbf{Answer:} The previously created XGBoost model performed a little
better than the current model. One key difference to account for is that
XGBoosting is primarily a tree based/gradient boosting model that uses a
matrix enhanced feature table(consisting of counts or tf-idf) which may
be independent from the order of words in the data. The XGBoosting model
would be more accurate in terms of accuracy scores and predictions, but
may not be the best choice is the objective is to have a model that is
highly context drive, making the former more appropriate for smaller
datasets that have a single label approach to them. The Pytorch model
may be improved with advanced hyper-parameter tuning but for now XGBoost
seems like a better approach for sentiment analysis.

    \hypertarget{todo-more-testing}{%
\subsubsection{(TODO) More testing}\label{todo-more-testing}}

We now have a trained model which has been deployed and which we can
send processed reviews to and which returns the predicted sentiment.
However, ultimately we would like to be able to send our model an
unprocessed review. That is, we would like to send the review itself as
a string. For example, suppose we wish to send the following review to
our model.

    \begin{tcolorbox}[breakable, size=fbox, boxrule=1pt, pad at break*=1mm,colback=cellbackground, colframe=cellborder]
\prompt{In}{incolor}{34}{\boxspacing}
\begin{Verbatim}[commandchars=\\\{\}]
\PY{n}{test\PYZus{}review} \PY{o}{=} \PY{l+s+s1}{\PYZsq{}}\PY{l+s+s1}{The simplest pleasures in life are the best, and this film is one of them. Combining a rather basic storyline of love and adventure this movie transcends the usual weekend fair with wit and unmitigated charm.}\PY{l+s+s1}{\PYZsq{}}
\end{Verbatim}
\end{tcolorbox}

    The question we now need to answer is, how do we send this review to our
model?

Recall in the first section of this notebook we did a bunch of data
processing to the IMDb dataset. In particular, we did two specific
things to the provided reviews. - Removed any html tags and stemmed the
input - Encoded the review as a sequence of integers using
\texttt{word\_dict}

In order process the review we will need to repeat these two steps.

\textbf{TODO}: Using the \texttt{review\_to\_words} and
\texttt{convert\_and\_pad} methods from section one, convert
\texttt{test\_review} into a numpy array \texttt{test\_data} suitable to
send to our model. Remember that our model expects input of the form
\texttt{review\_length,\ review{[}500{]}}.

    \begin{tcolorbox}[breakable, size=fbox, boxrule=1pt, pad at break*=1mm,colback=cellbackground, colframe=cellborder]
\prompt{In}{incolor}{35}{\boxspacing}
\begin{Verbatim}[commandchars=\\\{\}]
\PY{c+c1}{\PYZsh{} TODO: Convert test\PYZus{}review into a form usable by the model and save the results in test\PYZus{}data}
\PY{n}{test\PYZus{}data} \PY{o}{=} \PY{n}{review\PYZus{}to\PYZus{}words}\PY{p}{(}\PY{n}{test\PYZus{}review}\PY{p}{)}
\PY{n}{test\PYZus{}data} \PY{o}{=} \PY{p}{[}\PY{n}{np}\PY{o}{.}\PY{n}{array}\PY{p}{(}\PY{n}{convert\PYZus{}and\PYZus{}pad}\PY{p}{(}\PY{n}{word\PYZus{}dict}\PY{p}{,} \PY{n}{test\PYZus{}data}\PY{p}{)}\PY{p}{[}\PY{l+m+mi}{0}\PY{p}{]}\PY{p}{)}\PY{p}{]}
\end{Verbatim}
\end{tcolorbox}

    \begin{tcolorbox}[breakable, size=fbox, boxrule=1pt, pad at break*=1mm,colback=cellbackground, colframe=cellborder]
\prompt{In}{incolor}{36}{\boxspacing}
\begin{Verbatim}[commandchars=\\\{\}]
\PY{n}{test\PYZus{}review\PYZus{}words} \PY{o}{=} \PY{n}{review\PYZus{}to\PYZus{}words}\PY{p}{(}\PY{n}{test\PYZus{}review}\PY{p}{)}     \PY{c+c1}{\PYZsh{} splits reviews to words}
\PY{n}{review\PYZus{}X}\PY{p}{,} \PY{n}{review\PYZus{}len} \PY{o}{=} \PY{n}{convert\PYZus{}and\PYZus{}pad}\PY{p}{(}\PY{n}{word\PYZus{}dict}\PY{p}{,} \PY{n}{test\PYZus{}review\PYZus{}words}\PY{p}{)}   \PY{c+c1}{\PYZsh{} pad review}

\PY{n}{data\PYZus{}pack} \PY{o}{=} \PY{n}{np}\PY{o}{.}\PY{n}{hstack}\PY{p}{(}\PY{p}{(}\PY{n}{review\PYZus{}len}\PY{p}{,} \PY{n}{review\PYZus{}X}\PY{p}{)}\PY{p}{)}
\PY{n}{data\PYZus{}pack} \PY{o}{=} \PY{n}{data\PYZus{}pack}\PY{o}{.}\PY{n}{reshape}\PY{p}{(}\PY{l+m+mi}{1}\PY{p}{,} \PY{o}{\PYZhy{}}\PY{l+m+mi}{1}\PY{p}{)}

\PY{n}{test\PYZus{}data} \PY{o}{=} \PY{n}{torch}\PY{o}{.}\PY{n}{from\PYZus{}numpy}\PY{p}{(}\PY{n}{data\PYZus{}pack}\PY{p}{)}
\PY{n}{test\PYZus{}data} \PY{o}{=} \PY{n}{test\PYZus{}data}\PY{o}{.}\PY{n}{to}\PY{p}{(}\PY{n}{device}\PY{p}{)}
\end{Verbatim}
\end{tcolorbox}

    Now that we have processed the review, we can send the resulting array
to our model to predict the sentiment of the review.

    \begin{tcolorbox}[breakable, size=fbox, boxrule=1pt, pad at break*=1mm,colback=cellbackground, colframe=cellborder]
\prompt{In}{incolor}{39}{\boxspacing}
\begin{Verbatim}[commandchars=\\\{\}]
\PY{n}{predictor}\PY{o}{.}\PY{n}{predict}\PY{p}{(}\PY{n}{test\PYZus{}data}\PY{p}{)}
\end{Verbatim}
\end{tcolorbox}

            \begin{tcolorbox}[breakable, size=fbox, boxrule=.5pt, pad at break*=1mm, opacityfill=0]
\prompt{Out}{outcolor}{39}{\boxspacing}
\begin{Verbatim}[commandchars=\\\{\}]
array(0.5066693, dtype=float32)
\end{Verbatim}
\end{tcolorbox}
        
    Since the return value of our model is close to \texttt{1}, we can be
certain that the review we submitted is positive.

    \hypertarget{delete-the-endpoint}{%
\subsubsection{Delete the endpoint}\label{delete-the-endpoint}}

Of course, just like in the XGBoost notebook, once we've deployed an
endpoint it continues to run until we tell it to shut down. Since we are
done using our endpoint for now, we can delete it.

    \begin{tcolorbox}[breakable, size=fbox, boxrule=1pt, pad at break*=1mm,colback=cellbackground, colframe=cellborder]
\prompt{In}{incolor}{40}{\boxspacing}
\begin{Verbatim}[commandchars=\\\{\}]
\PY{n}{estimator}\PY{o}{.}\PY{n}{delete\PYZus{}endpoint}\PY{p}{(}\PY{p}{)}
\end{Verbatim}
\end{tcolorbox}

    \begin{Verbatim}[commandchars=\\\{\}]
estimator.delete\_endpoint() will be deprecated in SageMaker Python SDK v2.
Please use the delete\_endpoint() function on your predictor instead.
    \end{Verbatim}

    \hypertarget{step-6-again---deploy-the-model-for-the-web-app}{%
\subsection{Step 6 (again) - Deploy the model for the web
app}\label{step-6-again---deploy-the-model-for-the-web-app}}

Now that we know that our model is working, it's time to create some
custom inference code so that we can send the model a review which has
not been processed and have it determine the sentiment of the review.

As we saw above, by default the estimator which we created, when
deployed, will use the entry script and directory which we provided when
creating the model. However, since we now wish to accept a string as
input and our model expects a processed review, we need to write some
custom inference code.

We will store the code that we write in the \texttt{serve} directory.
Provided in this directory is the \texttt{model.py} file that we used to
construct our model, a \texttt{utils.py} file which contains the
\texttt{review\_to\_words} and \texttt{convert\_and\_pad} pre-processing
functions which we used during the initial data processing, and
\texttt{predict.py}, the file which will contain our custom inference
code. Note also that \texttt{requirements.txt} is present which will
tell SageMaker what Python libraries are required by our custom
inference code.

When deploying a PyTorch model in SageMaker, you are expected to provide
four functions which the SageMaker inference container will use. -
\texttt{model\_fn}: This function is the same function that we used in
the training script and it tells SageMaker how to load our model. -
\texttt{input\_fn}: This function receives the raw serialized input that
has been sent to the model's endpoint and its job is to de-serialize and
make the input available for the inference code. - \texttt{output\_fn}:
This function takes the output of the inference code and its job is to
serialize this output and return it to the caller of the model's
endpoint. - \texttt{predict\_fn}: The heart of the inference script,
this is where the actual prediction is done and is the function which
you will need to complete.

For the simple website that we are constructing during this project, the
\texttt{input\_fn} and \texttt{output\_fn} methods are relatively
straightforward. We only require being able to accept a string as input
and we expect to return a single value as output. You might imagine
though that in a more complex application the input or output may be
image data or some other binary data which would require some effort to
serialize.

\hypertarget{todo-writing-inference-code}{%
\subsubsection{(TODO) Writing inference
code}\label{todo-writing-inference-code}}

Before writing our custom inference code, we will begin by taking a look
at the code which has been provided.

    \begin{tcolorbox}[breakable, size=fbox, boxrule=1pt, pad at break*=1mm,colback=cellbackground, colframe=cellborder]
\prompt{In}{incolor}{38}{\boxspacing}
\begin{Verbatim}[commandchars=\\\{\}]
\PY{o}{!}pygmentize serve/predict.py
\end{Verbatim}
\end{tcolorbox}

    \begin{Verbatim}[commandchars=\\\{\}]
\textcolor{ansi-blue}{import} \textcolor{ansi-cyan}{\underline{argparse}}
\textcolor{ansi-blue}{import} \textcolor{ansi-cyan}{\underline{json}}
\textcolor{ansi-blue}{import} \textcolor{ansi-cyan}{\underline{os}}
\textcolor{ansi-blue}{import} \textcolor{ansi-cyan}{\underline{pickle}}
\textcolor{ansi-blue}{import} \textcolor{ansi-cyan}{\underline{sys}}
\textcolor{ansi-blue}{import} \textcolor{ansi-cyan}{\underline{sagemaker\_containers}}
\textcolor{ansi-blue}{import} \textcolor{ansi-cyan}{\underline{pandas}} \textcolor{ansi-blue}{as}
\textcolor{ansi-cyan}{\underline{pd}}
\textcolor{ansi-blue}{import} \textcolor{ansi-cyan}{\underline{numpy}} \textcolor{ansi-blue}{as}
\textcolor{ansi-cyan}{\underline{np}}
\textcolor{ansi-blue}{import} \textcolor{ansi-cyan}{\underline{torch}}
\textcolor{ansi-blue}{import}
\textcolor{ansi-cyan}{\underline{torch}}\textcolor{ansi-cyan}{\underline{.}}\textcolor{ansi-cyan}{\underline{nn}}
\textcolor{ansi-blue}{as} \textcolor{ansi-cyan}{\underline{nn}}
\textcolor{ansi-blue}{import}
\textcolor{ansi-cyan}{\underline{torch}}\textcolor{ansi-cyan}{\underline{.}}\textcolor{ansi-cyan}{\underline{optim}}
\textcolor{ansi-blue}{as} \textcolor{ansi-cyan}{\underline{optim}}
\textcolor{ansi-blue}{import} \textcolor{ansi-cyan}{\underline{torch}}\textcolor{ansi-cyan}{\underline{.}}\underline{[36
mutils}\textcolor{ansi-cyan}{\underline{.}}\textcolor{ansi-cyan}{\underline{data}}

\textcolor{ansi-blue}{from} \textcolor{ansi-cyan}{\underline{model}} \textcolor{ansi-blue}{import}
LSTMClassifier

\textcolor{ansi-blue}{from} \textcolor{ansi-cyan}{\underline{utils}} \textcolor{ansi-blue}{import}
review\_to\_words, convert\_and\_pad

\textcolor{ansi-blue}{def} \textcolor{ansi-green}{model\_fn}(model\_dir):
    \textcolor{ansi-yellow}{"""Load the PyTorch model from the `model\_dir` directory."""}
    \textcolor{ansi-cyan}{print}(\textcolor{ansi-yellow}{"}\textcolor{ansi-yellow}{Loading
model.}\textcolor{ansi-yellow}{"})

    \textcolor{ansi-white}{\# First, load the parameters used to create the model.}
    model\_info = \{\}
    model\_info\_path = os.path.join(model\_dir,
\textcolor{ansi-yellow}{'}\textcolor{ansi-yellow}{model\_info.pth}\textcolor{ansi-yellow}{'})
    \textcolor{ansi-blue}{with} \textcolor{ansi-cyan}{open}(model\_info\_path,
\textcolor{ansi-yellow}{'}\textcolor{ansi-yellow}{rb}\textcolor{ansi-yellow}{'}) \textcolor{ansi-blue}{as} f:
        model\_info = torch.load(f)

    \textcolor{ansi-cyan}{print}(\textcolor{ansi-yellow}{"}\textcolor{ansi-yellow}{model\_info:
}\textcolor{ansi-yellow}{\{\}}\textcolor{ansi-yellow}{"}.format(model\_info))

    \textcolor{ansi-white}{\# Determine the device and construct the model.}
    device = torch.device(\textcolor{ansi-yellow}{"}\textcolor{ansi-yellow}{cuda}\textcolor{ansi-yellow}{"}
\textcolor{ansi-blue}{if} torch.cuda.is\_available() \textcolor{ansi-blue}{else}
\textcolor{ansi-yellow}{"}\textcolor{ansi-yellow}{cpu}\textcolor{ansi-yellow}{"})
    model = LSTMClassifier(model\_info[\textcolor{ansi-yellow}{'}\textcolor{ansi-yellow}{embedding\_dim[39;49
;00m}\textcolor{ansi-yellow}{'}],
model\_info[\textcolor{ansi-yellow}{'}\textcolor{ansi-yellow}{hidden\_dim}\textcolor{ansi-yellow}{'}],
model\_info[\textcolor{ansi-yellow}{'}\textcolor{ansi-yellow}{vocab\_size}\textcolor{ansi-yellow}{'}])

    \textcolor{ansi-white}{\# Load the store model parameters.}
    model\_path = os.path.join(model\_dir,
\textcolor{ansi-yellow}{'}\textcolor{ansi-yellow}{model.pth}\textcolor{ansi-yellow}{'})
    \textcolor{ansi-blue}{with} \textcolor{ansi-cyan}{open}(model\_path,
\textcolor{ansi-yellow}{'}\textcolor{ansi-yellow}{rb}\textcolor{ansi-yellow}{'}) \textcolor{ansi-blue}{as} f:
        model.load\_state\_dict(torch.load(f))

    \textcolor{ansi-white}{\# Load the saved word\_dict.}
    word\_dict\_path = os.path.join(model\_dir,
\textcolor{ansi-yellow}{'}\textcolor{ansi-yellow}{word\_dict.pkl}\textcolor{ansi-yellow}{'})
    \textcolor{ansi-blue}{with} \textcolor{ansi-cyan}{open}(word\_dict\_path,
\textcolor{ansi-yellow}{'}\textcolor{ansi-yellow}{rb}\textcolor{ansi-yellow}{'}) \textcolor{ansi-blue}{as} f:
        model.word\_dict = pickle.load(f)

    model.to(device).eval()

    \textcolor{ansi-cyan}{print}(\textcolor{ansi-yellow}{"}\textcolor{ansi-yellow}{Done loading
model.}\textcolor{ansi-yellow}{"})
    \textcolor{ansi-blue}{return} model

\textcolor{ansi-blue}{def} \textcolor{ansi-green}{input\_fn}(serialized\_input\_data,
content\_type):
    \textcolor{ansi-cyan}{print}(\textcolor{ansi-yellow}{'}\textcolor{ansi-yellow}{Deserializing the input
data.}\textcolor{ansi-yellow}{'})
    \textcolor{ansi-blue}{if} content\_type ==
\textcolor{ansi-yellow}{'}\textcolor{ansi-yellow}{text/plain}\textcolor{ansi-yellow}{'}:
        data = serialized\_input\_data.decode(\textcolor{ansi-yellow}{'}\textcolor{ansi-yellow}{utf-8[39;49;0
0m}\textcolor{ansi-yellow}{'})
        \textcolor{ansi-blue}{return} data
    \textcolor{ansi-blue}{raise}
\textcolor{ansi-cyan}{Exception}(\textcolor{ansi-yellow}{'}\textcolor{ansi-yellow}{Requested unsupported
ContentType in content\_type: }\textcolor{ansi-yellow}{'} + content\_type)

\textcolor{ansi-blue}{def} \textcolor{ansi-green}{output\_fn}(prediction\_output, accept):
    \textcolor{ansi-cyan}{print}(\textcolor{ansi-yellow}{'}\textcolor{ansi-yellow}{Serializing the generated
output.}\textcolor{ansi-yellow}{'})
    \textcolor{ansi-blue}{return} \textcolor{ansi-cyan}{str}(prediction\_output)

\textcolor{ansi-blue}{def} \textcolor{ansi-green}{predict\_fn}(input\_data, model):
    \textcolor{ansi-cyan}{print}(\textcolor{ansi-yellow}{'}\textcolor{ansi-yellow}{Inferring sentiment of input
data.}\textcolor{ansi-yellow}{'})

    device = torch.device(\textcolor{ansi-yellow}{"}\textcolor{ansi-yellow}{cuda}\textcolor{ansi-yellow}{"}
\textcolor{ansi-blue}{if} torch.cuda.is\_available() \textcolor{ansi-blue}{else}
\textcolor{ansi-yellow}{"}\textcolor{ansi-yellow}{cpu}\textcolor{ansi-yellow}{"})

    \textcolor{ansi-blue}{if} model.word\_dict \textcolor{ansi-magenta}{is} \textcolor{ansi-blue}{None}:
        \textcolor{ansi-blue}{raise}
\textcolor{ansi-cyan}{Exception}(\textcolor{ansi-yellow}{'}\textcolor{ansi-yellow}{Model has not been loaded
properly, no word\_dict.}\textcolor{ansi-yellow}{'})

    \textcolor{ansi-white}{\# TODO: Process input\_data so that it is ready to be sent to our
model.}
    \textcolor{ansi-white}{\#       You should produce two variables:}
    \textcolor{ansi-white}{\#         data\_X   - A sequence of length 500 which represents the
converted review}
    \textcolor{ansi-white}{\#         data\_len - The length of the review}

    data\_X = \textcolor{ansi-blue}{None}
    data\_len = \textcolor{ansi-blue}{None}

    \textcolor{ansi-white}{\# Using data\_X and data\_len we construct an appropriate input tensor.
Remember}
    \textcolor{ansi-white}{\# that our model expects input data of the form 'len,
review[500]'.}
    data\_pack = np.hstack((data\_len, data\_X))
    data\_pack = data\_pack.reshape(\textcolor{ansi-blue}{1}, -\textcolor{ansi-blue}{1})

    data = torch.from\_numpy(data\_pack)
    data = data.to(device)

    \textcolor{ansi-white}{\# Make sure to put the model into evaluation mode}
    model.eval()

    \textcolor{ansi-white}{\# TODO: Compute the result of applying the model to the input data. The
variable `result` should}
    \textcolor{ansi-white}{\#       be a numpy array which contains a single integer which is
either 1 or 0}

    result = \textcolor{ansi-blue}{None}

    \textcolor{ansi-blue}{return} result
    \end{Verbatim}

    As mentioned earlier, the \texttt{model\_fn} method is the same as the
one provided in the training code and the \texttt{input\_fn} and
\texttt{output\_fn} methods are very simple and your task will be to
complete the \texttt{predict\_fn} method. Make sure that you save the
completed file as \texttt{predict.py} in the \texttt{serve} directory.

\textbf{TODO}: Complete the \texttt{predict\_fn()} method in the
\texttt{serve/predict.py} file.

    \hypertarget{deploying-the-model}{%
\subsubsection{Deploying the model}\label{deploying-the-model}}

Now that the custom inference code has been written, we will create and
deploy our model. To begin with, we need to construct a new PyTorchModel
object which points to the model artifacts created during training and
also points to the inference code that we wish to use. Then we can call
the deploy method to launch the deployment container.

\textbf{NOTE}: The default behaviour for a deployed PyTorch model is to
assume that any input passed to the predictor is a \texttt{numpy} array.
In our case we want to send a string so we need to construct a simple
wrapper around the \texttt{RealTimePredictor} class to accomodate simple
strings. In a more complicated situation you may want to provide a
serialization object, for example if you wanted to sent image data.

    \begin{tcolorbox}[breakable, size=fbox, boxrule=1pt, pad at break*=1mm,colback=cellbackground, colframe=cellborder]
\prompt{In}{incolor}{39}{\boxspacing}
\begin{Verbatim}[commandchars=\\\{\}]
\PY{k+kn}{from} \PY{n+nn}{sagemaker}\PY{n+nn}{.}\PY{n+nn}{predictor} \PY{k+kn}{import} \PY{n}{RealTimePredictor}
\PY{k+kn}{from} \PY{n+nn}{sagemaker}\PY{n+nn}{.}\PY{n+nn}{pytorch} \PY{k+kn}{import} \PY{n}{PyTorchModel}

\PY{k}{class} \PY{n+nc}{StringPredictor}\PY{p}{(}\PY{n}{RealTimePredictor}\PY{p}{)}\PY{p}{:}
    \PY{k}{def} \PY{n+nf+fm}{\PYZus{}\PYZus{}init\PYZus{}\PYZus{}}\PY{p}{(}\PY{n+nb+bp}{self}\PY{p}{,} \PY{n}{endpoint\PYZus{}name}\PY{p}{,} \PY{n}{sagemaker\PYZus{}session}\PY{p}{)}\PY{p}{:}
        \PY{n+nb}{super}\PY{p}{(}\PY{n}{StringPredictor}\PY{p}{,} \PY{n+nb+bp}{self}\PY{p}{)}\PY{o}{.}\PY{n+nf+fm}{\PYZus{}\PYZus{}init\PYZus{}\PYZus{}}\PY{p}{(}\PY{n}{endpoint\PYZus{}name}\PY{p}{,} \PY{n}{sagemaker\PYZus{}session}\PY{p}{,} \PY{n}{content\PYZus{}type}\PY{o}{=}\PY{l+s+s1}{\PYZsq{}}\PY{l+s+s1}{text/plain}\PY{l+s+s1}{\PYZsq{}}\PY{p}{)}

\PY{n}{model} \PY{o}{=} \PY{n}{PyTorchModel}\PY{p}{(}\PY{n}{model\PYZus{}data}\PY{o}{=}\PY{n}{estimator}\PY{o}{.}\PY{n}{model\PYZus{}data}\PY{p}{,}
                     \PY{n}{role} \PY{o}{=} \PY{n}{role}\PY{p}{,}
                     \PY{n}{framework\PYZus{}version}\PY{o}{=}\PY{l+s+s1}{\PYZsq{}}\PY{l+s+s1}{0.4.0}\PY{l+s+s1}{\PYZsq{}}\PY{p}{,}
                     \PY{n}{entry\PYZus{}point}\PY{o}{=}\PY{l+s+s1}{\PYZsq{}}\PY{l+s+s1}{predict.py}\PY{l+s+s1}{\PYZsq{}}\PY{p}{,}
                     \PY{n}{source\PYZus{}dir}\PY{o}{=}\PY{l+s+s1}{\PYZsq{}}\PY{l+s+s1}{serve}\PY{l+s+s1}{\PYZsq{}}\PY{p}{,}
                     \PY{n}{predictor\PYZus{}cls}\PY{o}{=}\PY{n}{StringPredictor}\PY{p}{)}
\PY{n}{predictor} \PY{o}{=} \PY{n}{model}\PY{o}{.}\PY{n}{deploy}\PY{p}{(}\PY{n}{initial\PYZus{}instance\PYZus{}count}\PY{o}{=}\PY{l+m+mi}{1}\PY{p}{,} \PY{n}{instance\PYZus{}type}\PY{o}{=}\PY{l+s+s1}{\PYZsq{}}\PY{l+s+s1}{ml.m4.xlarge}\PY{l+s+s1}{\PYZsq{}}\PY{p}{)}
\end{Verbatim}
\end{tcolorbox}

    \begin{Verbatim}[commandchars=\\\{\}]
Parameter image will be renamed to image\_uri in SageMaker Python SDK v2.
'create\_image\_uri' will be deprecated in favor of 'ImageURIProvider' class in
SageMaker Python SDK v2.
    \end{Verbatim}

    \begin{Verbatim}[commandchars=\\\{\}]
---------------!
    \end{Verbatim}

    \hypertarget{testing-the-model}{%
\subsubsection{Testing the model}\label{testing-the-model}}

Now that we have deployed our model with the custom inference code, we
should test to see if everything is working. Here we test our model by
loading the first \texttt{250} positive and negative reviews and send
them to the endpoint, then collect the results. The reason for only
sending some of the data is that the amount of time it takes for our
model to process the input and then perform inference is quite long and
so testing the entire data set would be prohibitive.

    \begin{tcolorbox}[breakable, size=fbox, boxrule=1pt, pad at break*=1mm,colback=cellbackground, colframe=cellborder]
\prompt{In}{incolor}{43}{\boxspacing}
\begin{Verbatim}[commandchars=\\\{\}]
\PY{k+kn}{import} \PY{n+nn}{glob}

\PY{k}{def} \PY{n+nf}{test\PYZus{}reviews}\PY{p}{(}\PY{n}{data\PYZus{}dir}\PY{o}{=}\PY{l+s+s1}{\PYZsq{}}\PY{l+s+s1}{../data/aclImdb}\PY{l+s+s1}{\PYZsq{}}\PY{p}{,} \PY{n}{stop}\PY{o}{=}\PY{l+m+mi}{250}\PY{p}{)}\PY{p}{:}
    
    \PY{n}{results} \PY{o}{=} \PY{p}{[}\PY{p}{]}
    \PY{n}{ground} \PY{o}{=} \PY{p}{[}\PY{p}{]}
    
    \PY{c+c1}{\PYZsh{} We make sure to test both positive and negative reviews    }
    \PY{k}{for} \PY{n}{sentiment} \PY{o+ow}{in} \PY{p}{[}\PY{l+s+s1}{\PYZsq{}}\PY{l+s+s1}{pos}\PY{l+s+s1}{\PYZsq{}}\PY{p}{,} \PY{l+s+s1}{\PYZsq{}}\PY{l+s+s1}{neg}\PY{l+s+s1}{\PYZsq{}}\PY{p}{]}\PY{p}{:}
        
        \PY{n}{path} \PY{o}{=} \PY{n}{os}\PY{o}{.}\PY{n}{path}\PY{o}{.}\PY{n}{join}\PY{p}{(}\PY{n}{data\PYZus{}dir}\PY{p}{,} \PY{l+s+s1}{\PYZsq{}}\PY{l+s+s1}{test}\PY{l+s+s1}{\PYZsq{}}\PY{p}{,} \PY{n}{sentiment}\PY{p}{,} \PY{l+s+s1}{\PYZsq{}}\PY{l+s+s1}{*.txt}\PY{l+s+s1}{\PYZsq{}}\PY{p}{)}
        \PY{n}{files} \PY{o}{=} \PY{n}{glob}\PY{o}{.}\PY{n}{glob}\PY{p}{(}\PY{n}{path}\PY{p}{)}
        
        \PY{n}{files\PYZus{}read} \PY{o}{=} \PY{l+m+mi}{0}
        
        \PY{n+nb}{print}\PY{p}{(}\PY{l+s+s1}{\PYZsq{}}\PY{l+s+s1}{Starting }\PY{l+s+s1}{\PYZsq{}}\PY{p}{,} \PY{n}{sentiment}\PY{p}{,} \PY{l+s+s1}{\PYZsq{}}\PY{l+s+s1}{ files}\PY{l+s+s1}{\PYZsq{}}\PY{p}{)}
        
        \PY{c+c1}{\PYZsh{} Iterate through the files and send them to the predictor}
        \PY{k}{for} \PY{n}{f} \PY{o+ow}{in} \PY{n}{files}\PY{p}{:}
            \PY{k}{with} \PY{n+nb}{open}\PY{p}{(}\PY{n}{f}\PY{p}{)} \PY{k}{as} \PY{n}{review}\PY{p}{:}
                \PY{c+c1}{\PYZsh{} First, we store the ground truth (was the review positive or negative)}
                \PY{k}{if} \PY{n}{sentiment} \PY{o}{==} \PY{l+s+s1}{\PYZsq{}}\PY{l+s+s1}{pos}\PY{l+s+s1}{\PYZsq{}}\PY{p}{:}
                    \PY{n}{ground}\PY{o}{.}\PY{n}{append}\PY{p}{(}\PY{l+m+mi}{1}\PY{p}{)}
                \PY{k}{else}\PY{p}{:}
                    \PY{n}{ground}\PY{o}{.}\PY{n}{append}\PY{p}{(}\PY{l+m+mi}{0}\PY{p}{)}
                \PY{c+c1}{\PYZsh{} Read in the review and convert to \PYZsq{}utf\PYZhy{}8\PYZsq{} for transmission via HTTP}
                \PY{n}{review\PYZus{}input} \PY{o}{=} \PY{n}{review}\PY{o}{.}\PY{n}{read}\PY{p}{(}\PY{p}{)}\PY{o}{.}\PY{n}{encode}\PY{p}{(}\PY{l+s+s1}{\PYZsq{}}\PY{l+s+s1}{utf\PYZhy{}8}\PY{l+s+s1}{\PYZsq{}}\PY{p}{)}
                \PY{c+c1}{\PYZsh{} Send the review to the predictor and store the results}
                \PY{n}{results}\PY{o}{.}\PY{n}{append}\PY{p}{(}\PY{n+nb}{int}\PY{p}{(}\PY{n}{predictor}\PY{o}{.}\PY{n}{predict}\PY{p}{(}\PY{n}{review\PYZus{}input}\PY{p}{)}\PY{p}{)}\PY{p}{)}
                
            \PY{c+c1}{\PYZsh{} Sending reviews to our endpoint one at a time takes a while so we}
            \PY{c+c1}{\PYZsh{} only send a small number of reviews}
            \PY{n}{files\PYZus{}read} \PY{o}{+}\PY{o}{=} \PY{l+m+mi}{1}
            \PY{k}{if} \PY{n}{files\PYZus{}read} \PY{o}{==} \PY{n}{stop}\PY{p}{:}
                \PY{k}{break}
            
    \PY{k}{return} \PY{n}{ground}\PY{p}{,} \PY{n}{results}
\end{Verbatim}
\end{tcolorbox}

    \begin{tcolorbox}[breakable, size=fbox, boxrule=1pt, pad at break*=1mm,colback=cellbackground, colframe=cellborder]
\prompt{In}{incolor}{52}{\boxspacing}
\begin{Verbatim}[commandchars=\\\{\}]
\PY{n}{ground}\PY{p}{,} \PY{n}{results} \PY{o}{=} \PY{n}{test\PYZus{}reviews}\PY{p}{(}\PY{p}{)}
\end{Verbatim}
\end{tcolorbox}

    \begin{Verbatim}[commandchars=\\\{\}]
Starting  pos  files
    \end{Verbatim}

    \begin{Verbatim}[commandchars=\\\{\}]

        ---------------------------------------------------------------------------

        ValidationError                           Traceback (most recent call last)

        <ipython-input-52-27d1fd4b7c7b> in <module>
    ----> 1 ground, results = test\_reviews()
    

        <ipython-input-43-4c56d4ff716b> in test\_reviews(data\_dir, stop)
         27                 review\_input = review.read().encode('utf-8')
         28                 \# Send the review to the predictor and store the results
    ---> 29                 results.append(int(predictor.predict(review\_input)))
         30 
         31             \# Sending reviews to our endpoint one at a time takes a while so we


        \textasciitilde{}/anaconda3/envs/pytorch\_p36/lib/python3.6/site-packages/sagemaker/predictor.py in predict(self, data, initial\_args, target\_model, target\_variant)
        111 
        112         request\_args = self.\_create\_request\_args(data, initial\_args, target\_model, target\_variant)
    --> 113         response = self.sagemaker\_session.sagemaker\_runtime\_client.invoke\_endpoint(**request\_args)
        114         return self.\_handle\_response(response)
        115 


        \textasciitilde{}/anaconda3/envs/pytorch\_p36/lib/python3.6/site-packages/botocore/client.py in \_api\_call(self, *args, **kwargs)
        335                     "\%s() only accepts keyword arguments." \% py\_operation\_name)
        336             \# The "self" in this scope is referring to the BaseClient.
    --> 337             return self.\_make\_api\_call(operation\_name, kwargs)
        338 
        339         \_api\_call.\_\_name\_\_ = str(py\_operation\_name)


        \textasciitilde{}/anaconda3/envs/pytorch\_p36/lib/python3.6/site-packages/botocore/client.py in \_make\_api\_call(self, operation\_name, api\_params)
        654             error\_code = parsed\_response.get("Error", \{\}).get("Code")
        655             error\_class = self.exceptions.from\_code(error\_code)
    --> 656             raise error\_class(parsed\_response, operation\_name)
        657         else:
        658             return parsed\_response


        ValidationError: An error occurred (ValidationError) when calling the InvokeEndpoint operation: Endpoint sagemaker-pytorch-2020-09-10-08-46-11-995 of account 423904393887 not found.

    \end{Verbatim}

    \begin{tcolorbox}[breakable, size=fbox, boxrule=1pt, pad at break*=1mm,colback=cellbackground, colframe=cellborder]
\prompt{In}{incolor}{42}{\boxspacing}
\begin{Verbatim}[commandchars=\\\{\}]
\PY{k+kn}{from} \PY{n+nn}{sklearn}\PY{n+nn}{.}\PY{n+nn}{metrics} \PY{k+kn}{import} \PY{n}{accuracy\PYZus{}score}
\PY{n}{accuracy\PYZus{}score}\PY{p}{(}\PY{n}{ground}\PY{p}{,} \PY{n}{results}\PY{p}{)}
\end{Verbatim}
\end{tcolorbox}

    \begin{Verbatim}[commandchars=\\\{\}]

        ---------------------------------------------------------------------------

        NameError                                 Traceback (most recent call last)

        <ipython-input-42-f3e6875633e1> in <module>
          1 from sklearn.metrics import accuracy\_score
    ----> 2 accuracy\_score(ground, results)
    

        NameError: name 'ground' is not defined

    \end{Verbatim}

    As an additional test, we can try sending the \texttt{test\_review} that
we looked at earlier.

    \begin{tcolorbox}[breakable, size=fbox, boxrule=1pt, pad at break*=1mm,colback=cellbackground, colframe=cellborder]
\prompt{In}{incolor}{47}{\boxspacing}
\begin{Verbatim}[commandchars=\\\{\}]
\PY{n}{predictor}\PY{o}{.}\PY{n}{predict}\PY{p}{(}\PY{n}{test\PYZus{}review}\PY{p}{)}
\end{Verbatim}
\end{tcolorbox}

    \begin{Verbatim}[commandchars=\\\{\}]

        ---------------------------------------------------------------------------

        ModelError                                Traceback (most recent call last)

        <ipython-input-47-ccef8a3a2cb2> in <module>
    ----> 1 predictor.predict(test\_review)
    

        \textasciitilde{}/anaconda3/envs/pytorch\_p36/lib/python3.6/site-packages/sagemaker/predictor.py in predict(self, data, initial\_args, target\_model, target\_variant)
        111 
        112         request\_args = self.\_create\_request\_args(data, initial\_args, target\_model, target\_variant)
    --> 113         response = self.sagemaker\_session.sagemaker\_runtime\_client.invoke\_endpoint(**request\_args)
        114         return self.\_handle\_response(response)
        115 


        \textasciitilde{}/anaconda3/envs/pytorch\_p36/lib/python3.6/site-packages/botocore/client.py in \_api\_call(self, *args, **kwargs)
        335                     "\%s() only accepts keyword arguments." \% py\_operation\_name)
        336             \# The "self" in this scope is referring to the BaseClient.
    --> 337             return self.\_make\_api\_call(operation\_name, kwargs)
        338 
        339         \_api\_call.\_\_name\_\_ = str(py\_operation\_name)


        \textasciitilde{}/anaconda3/envs/pytorch\_p36/lib/python3.6/site-packages/botocore/client.py in \_make\_api\_call(self, operation\_name, api\_params)
        654             error\_code = parsed\_response.get("Error", \{\}).get("Code")
        655             error\_class = self.exceptions.from\_code(error\_code)
    --> 656             raise error\_class(parsed\_response, operation\_name)
        657         else:
        658             return parsed\_response


        ModelError: An error occurred (ModelError) when calling the InvokeEndpoint operation: Received server error (500) from model with message "<!DOCTYPE HTML PUBLIC "-//W3C//DTD HTML 3.2 Final//EN">
    <title>500 Internal Server Error</title>
    <h1>Internal Server Error</h1>
    <p>The server encountered an internal error and was unable to complete your request.  Either the server is overloaded or there is an error in the application.</p>
    ". See https://us-east-2.console.aws.amazon.com/cloudwatch/home?region=us-east-2\#logEventViewer:group=/aws/sagemaker/Endpoints/sagemaker-pytorch-2020-09-10-08-46-11-995 in account 423904393887 for more information.

    \end{Verbatim}

    Now that we know our endpoint is working as expected, we can set up the
web page that will interact with it. If you don't have time to finish
the project now, make sure to skip down to the end of this notebook and
shut down your endpoint. You can deploy it again when you come back.

    \hypertarget{step-7-again-use-the-model-for-the-web-app}{%
\subsection{Step 7 (again): Use the model for the web
app}\label{step-7-again-use-the-model-for-the-web-app}}

\begin{quote}
\textbf{TODO:} This entire section and the next contain tasks for you to
complete, mostly using the AWS console.
\end{quote}

So far we have been accessing our model endpoint by constructing a
predictor object which uses the endpoint and then just using the
predictor object to perform inference. What if we wanted to create a web
app which accessed our model? The way things are set up currently makes
that not possible since in order to access a SageMaker endpoint the app
would first have to authenticate with AWS using an IAM role which
included access to SageMaker endpoints. However, there is an easier way!
We just need to use some additional AWS services.

The diagram above gives an overview of how the various services will
work together. On the far right is the model which we trained above and
which is deployed using SageMaker. On the far left is our web app that
collects a user's movie review, sends it off and expects a positive or
negative sentiment in return.

In the middle is where some of the magic happens. We will construct a
Lambda function, which you can think of as a straightforward Python
function that can be executed whenever a specified event occurs. We will
give this function permission to send and recieve data from a SageMaker
endpoint.

Lastly, the method we will use to execute the Lambda function is a new
endpoint that we will create using API Gateway. This endpoint will be a
url that listens for data to be sent to it. Once it gets some data it
will pass that data on to the Lambda function and then return whatever
the Lambda function returns. Essentially it will act as an interface
that lets our web app communicate with the Lambda function.

\hypertarget{setting-up-a-lambda-function}{%
\subsubsection{Setting up a Lambda
function}\label{setting-up-a-lambda-function}}

The first thing we are going to do is set up a Lambda function. This
Lambda function will be executed whenever our public API has data sent
to it. When it is executed it will receive the data, perform any sort of
processing that is required, send the data (the review) to the SageMaker
endpoint we've created and then return the result.

\hypertarget{part-a-create-an-iam-role-for-the-lambda-function}{%
\paragraph{Part A: Create an IAM Role for the Lambda
function}\label{part-a-create-an-iam-role-for-the-lambda-function}}

Since we want the Lambda function to call a SageMaker endpoint, we need
to make sure that it has permission to do so. To do this, we will
construct a role that we can later give the Lambda function.

Using the AWS Console, navigate to the \textbf{IAM} page and click on
\textbf{Roles}. Then, click on \textbf{Create role}. Make sure that the
\textbf{AWS service} is the type of trusted entity selected and choose
\textbf{Lambda} as the service that will use this role, then click
\textbf{Next: Permissions}.

In the search box type \texttt{sagemaker} and select the check box next
to the \textbf{AmazonSageMakerFullAccess} policy. Then, click on
\textbf{Next: Review}.

Lastly, give this role a name. Make sure you use a name that you will
remember later on, for example \texttt{LambdaSageMakerRole}. Then, click
on \textbf{Create role}.

\hypertarget{part-b-create-a-lambda-function}{%
\paragraph{Part B: Create a Lambda
function}\label{part-b-create-a-lambda-function}}

Now it is time to actually create the Lambda function.

Using the AWS Console, navigate to the AWS Lambda page and click on
\textbf{Create a function}. When you get to the next page, make sure
that \textbf{Author from scratch} is selected. Now, name your Lambda
function, using a name that you will remember later on, for example
\texttt{sentiment\_analysis\_func}. Make sure that the \textbf{Python
3.6} runtime is selected and then choose the role that you created in
the previous part. Then, click on \textbf{Create Function}.

On the next page you will see some information about the Lambda function
you've just created. If you scroll down you should see an editor in
which you can write the code that will be executed when your Lambda
function is triggered. In our example, we will use the code below.

\begin{Shaded}
\begin{Highlighting}[]
\CommentTok{\# We need to use the low{-}level library to interact with SageMaker since the SageMaker API}
\CommentTok{\# is not available natively through Lambda.}
\ImportTok{import}\NormalTok{ boto3}

\KeywordTok{def}\NormalTok{ lambda\_handler(event, context):}

    \CommentTok{\# The SageMaker runtime is what allows us to invoke the endpoint that we\textquotesingle{}ve created.}
\NormalTok{    runtime }\OperatorTok{=}\NormalTok{ boto3.Session().client(}\StringTok{\textquotesingle{}sagemaker{-}runtime\textquotesingle{}}\NormalTok{)}

    \CommentTok{\# Now we use the SageMaker runtime to invoke our endpoint, sending the review we were given}
\NormalTok{    response }\OperatorTok{=}\NormalTok{ runtime.invoke\_endpoint(EndpointName }\OperatorTok{=} \StringTok{\textquotesingle{}**ENDPOINT NAME HERE**\textquotesingle{}}\NormalTok{,    }\CommentTok{\# The name of the endpoint we created}
\NormalTok{                                       ContentType }\OperatorTok{=} \StringTok{\textquotesingle{}text/plain\textquotesingle{}}\NormalTok{,                 }\CommentTok{\# The data format that is expected}
\NormalTok{                                       Body }\OperatorTok{=}\NormalTok{ event[}\StringTok{\textquotesingle{}body\textquotesingle{}}\NormalTok{])                       }\CommentTok{\# The actual review}

    \CommentTok{\# The response is an HTTP response whose body contains the result of our inference}
\NormalTok{    result }\OperatorTok{=}\NormalTok{ response[}\StringTok{\textquotesingle{}Body\textquotesingle{}}\NormalTok{].read().decode(}\StringTok{\textquotesingle{}utf{-}8\textquotesingle{}}\NormalTok{)}

    \ControlFlowTok{return}\NormalTok{ \{}
        \StringTok{\textquotesingle{}statusCode\textquotesingle{}}\NormalTok{ : }\DecValTok{200}\NormalTok{,}
        \StringTok{\textquotesingle{}headers\textquotesingle{}}\NormalTok{ : \{ }\StringTok{\textquotesingle{}Content{-}Type\textquotesingle{}}\NormalTok{ : }\StringTok{\textquotesingle{}text/plain\textquotesingle{}}\NormalTok{, }\StringTok{\textquotesingle{}Access{-}Control{-}Allow{-}Origin\textquotesingle{}}\NormalTok{ : }\StringTok{\textquotesingle{}*\textquotesingle{}}\NormalTok{ \},}
        \StringTok{\textquotesingle{}body\textquotesingle{}}\NormalTok{ : result}
\NormalTok{    \}}
\end{Highlighting}
\end{Shaded}

Once you have copy and pasted the code above into the Lambda code
editor, replace the \texttt{**ENDPOINT\ NAME\ HERE**} portion with the
name of the endpoint that we deployed earlier. You can determine the
name of the endpoint using the code cell below.

    \begin{tcolorbox}[breakable, size=fbox, boxrule=1pt, pad at break*=1mm,colback=cellbackground, colframe=cellborder]
\prompt{In}{incolor}{48}{\boxspacing}
\begin{Verbatim}[commandchars=\\\{\}]
\PY{n}{predictor}\PY{o}{.}\PY{n}{endpoint}
\end{Verbatim}
\end{tcolorbox}

            \begin{tcolorbox}[breakable, size=fbox, boxrule=.5pt, pad at break*=1mm, opacityfill=0]
\prompt{Out}{outcolor}{48}{\boxspacing}
\begin{Verbatim}[commandchars=\\\{\}]
'sagemaker-pytorch-2020-09-10-08-46-11-995'
\end{Verbatim}
\end{tcolorbox}
        
    Once you have added the endpoint name to the Lambda function, click on
\textbf{Save}. Your Lambda function is now up and running. Next we need
to create a way for our web app to execute the Lambda function.

\hypertarget{setting-up-api-gateway}{%
\subsubsection{Setting up API Gateway}\label{setting-up-api-gateway}}

Now that our Lambda function is set up, it is time to create a new API
using API Gateway that will trigger the Lambda function we have just
created.

Using AWS Console, navigate to \textbf{Amazon API Gateway} and then
click on \textbf{Get started}.

On the next page, make sure that \textbf{New API} is selected and give
the new api a name, for example, \texttt{sentiment\_analysis\_api}.
Then, click on \textbf{Create API}.

Now we have created an API, however it doesn't currently do anything.
What we want it to do is to trigger the Lambda function that we created
earlier.

Select the \textbf{Actions} dropdown menu and click \textbf{Create
Method}. A new blank method will be created, select its dropdown menu
and select \textbf{POST}, then click on the check mark beside it.

For the integration point, make sure that \textbf{Lambda Function} is
selected and click on the \textbf{Use Lambda Proxy integration}. This
option makes sure that the data that is sent to the API is then sent
directly to the Lambda function with no processing. It also means that
the return value must be a proper response object as it will also not be
processed by API Gateway.

Type the name of the Lambda function you created earlier into the
\textbf{Lambda Function} text entry box and then click on \textbf{Save}.
Click on \textbf{OK} in the pop-up box that then appears, giving
permission to API Gateway to invoke the Lambda function you created.

The last step in creating the API Gateway is to select the
\textbf{Actions} dropdown and click on \textbf{Deploy API}. You will
need to create a new Deployment stage and name it anything you like, for
example \texttt{prod}.

You have now successfully set up a public API to access your SageMaker
model. Make sure to copy or write down the URL provided to invoke your
newly created public API as this will be needed in the next step. This
URL can be found at the top of the page, highlighted in blue next to the
text \textbf{Invoke URL}.

    \hypertarget{step-4-deploying-our-web-app}{%
\subsection{Step 4: Deploying our web
app}\label{step-4-deploying-our-web-app}}

Now that we have a publicly available API, we can start using it in a
web app. For our purposes, we have provided a simple static html file
which can make use of the public api you created earlier.

In the \texttt{website} folder there should be a file called
\texttt{index.html}. Download the file to your computer and open that
file up in a text editor of your choice. There should be a line which
contains \textbf{**REPLACE WITH PUBLIC API URL**}. Replace this string
with the url that you wrote down in the last step and then save the
file.

Now, if you open \texttt{index.html} on your local computer, your
browser will behave as a local web server and you can use the provided
site to interact with your SageMaker model.

If you'd like to go further, you can host this html file anywhere you'd
like, for example using github or hosting a static site on Amazon's S3.
Once you have done this you can share the link with anyone you'd like
and have them play with it too!

\begin{quote}
\textbf{Important Note} In order for the web app to communicate with the
SageMaker endpoint, the endpoint has to actually be deployed and
running. This means that you are paying for it. Make sure that the
endpoint is running when you want to use the web app but that you shut
it down when you don't need it, otherwise you will end up with a
surprisingly large AWS bill.
\end{quote}

\textbf{TODO:} Make sure that you include the edited \texttt{index.html}
file in your project submission.

    Now that your web app is working, trying playing around with it and see
how well it works.

\textbf{Question}: Give an example of a review that you entered into
your web app. What was the predicted sentiment of your example review?

    \textbf{Answer:} The movie was well paced with amazing dialogue and
scenes. A definite must watch. API RESULT:- Positive.

    \hypertarget{delete-the-endpoint}{%
\subsubsection{Delete the endpoint}\label{delete-the-endpoint}}

Remember to always shut down your endpoint if you are no longer using
it. You are charged for the length of time that the endpoint is running
so if you forget and leave it on you could end up with an unexpectedly
large bill.

    \begin{tcolorbox}[breakable, size=fbox, boxrule=1pt, pad at break*=1mm,colback=cellbackground, colframe=cellborder]
\prompt{In}{incolor}{49}{\boxspacing}
\begin{Verbatim}[commandchars=\\\{\}]
\PY{n}{predictor}\PY{o}{.}\PY{n}{delete\PYZus{}endpoint}\PY{p}{(}\PY{p}{)}
\end{Verbatim}
\end{tcolorbox}

    \begin{tcolorbox}[breakable, size=fbox, boxrule=1pt, pad at break*=1mm,colback=cellbackground, colframe=cellborder]
\prompt{In}{incolor}{50}{\boxspacing}
\begin{Verbatim}[commandchars=\\\{\}]
\PY{o}{!}pip install jupyter\PYZhy{}cjk\PYZhy{}xelatex
\end{Verbatim}
\end{tcolorbox}

    \begin{Verbatim}[commandchars=\\\{\}]
Collecting jupyter-cjk-xelatex
  Downloading jupyter-cjk-xelatex-0.2.tar.gz (1.6 kB)
Requirement already satisfied: jupyter in
/home/ec2-user/anaconda3/envs/pytorch\_p36/lib/python3.6/site-packages (from
jupyter-cjk-xelatex) (1.0.0)
Requirement already satisfied: qtconsole in
/home/ec2-user/anaconda3/envs/pytorch\_p36/lib/python3.6/site-packages (from
jupyter->jupyter-cjk-xelatex) (4.7.5)
Requirement already satisfied: ipykernel in
/home/ec2-user/anaconda3/envs/pytorch\_p36/lib/python3.6/site-packages (from
jupyter->jupyter-cjk-xelatex) (5.3.2)
Requirement already satisfied: jupyter-console in
/home/ec2-user/anaconda3/envs/pytorch\_p36/lib/python3.6/site-packages (from
jupyter->jupyter-cjk-xelatex) (6.1.0)
Requirement already satisfied: nbconvert in
/home/ec2-user/anaconda3/envs/pytorch\_p36/lib/python3.6/site-packages (from
jupyter->jupyter-cjk-xelatex) (5.6.1)
Requirement already satisfied: notebook in
/home/ec2-user/anaconda3/envs/pytorch\_p36/lib/python3.6/site-packages (from
jupyter->jupyter-cjk-xelatex) (6.0.3)
Requirement already satisfied: ipywidgets in
/home/ec2-user/anaconda3/envs/pytorch\_p36/lib/python3.6/site-packages (from
jupyter->jupyter-cjk-xelatex) (7.5.1)
Requirement already satisfied: pygments in
/home/ec2-user/anaconda3/envs/pytorch\_p36/lib/python3.6/site-packages (from
qtconsole->jupyter->jupyter-cjk-xelatex) (2.6.1)
Requirement already satisfied: jupyter-client>=4.1 in
/home/ec2-user/anaconda3/envs/pytorch\_p36/lib/python3.6/site-packages (from
qtconsole->jupyter->jupyter-cjk-xelatex) (6.1.6)
Requirement already satisfied: ipython-genutils in
/home/ec2-user/anaconda3/envs/pytorch\_p36/lib/python3.6/site-packages (from
qtconsole->jupyter->jupyter-cjk-xelatex) (0.2.0)
Requirement already satisfied: traitlets in
/home/ec2-user/anaconda3/envs/pytorch\_p36/lib/python3.6/site-packages (from
qtconsole->jupyter->jupyter-cjk-xelatex) (4.3.3)
Requirement already satisfied: qtpy in
/home/ec2-user/anaconda3/envs/pytorch\_p36/lib/python3.6/site-packages (from
qtconsole->jupyter->jupyter-cjk-xelatex) (1.9.0)
Requirement already satisfied: pyzmq>=17.1 in
/home/ec2-user/anaconda3/envs/pytorch\_p36/lib/python3.6/site-packages (from
qtconsole->jupyter->jupyter-cjk-xelatex) (19.0.1)
Requirement already satisfied: jupyter-core in
/home/ec2-user/anaconda3/envs/pytorch\_p36/lib/python3.6/site-packages (from
qtconsole->jupyter->jupyter-cjk-xelatex) (4.6.3)
Requirement already satisfied: tornado>=4.2 in
/home/ec2-user/anaconda3/envs/pytorch\_p36/lib/python3.6/site-packages (from
ipykernel->jupyter->jupyter-cjk-xelatex) (6.0.4)
Requirement already satisfied: ipython>=5.0.0 in
/home/ec2-user/anaconda3/envs/pytorch\_p36/lib/python3.6/site-packages (from
ipykernel->jupyter->jupyter-cjk-xelatex) (7.16.1)
Requirement already satisfied: prompt-toolkit!=3.0.0,!=3.0.1,<3.1.0,>=2.0.0 in
/home/ec2-user/anaconda3/envs/pytorch\_p36/lib/python3.6/site-packages (from
jupyter-console->jupyter->jupyter-cjk-xelatex) (3.0.5)
Requirement already satisfied: entrypoints>=0.2.2 in
/home/ec2-user/anaconda3/envs/pytorch\_p36/lib/python3.6/site-packages (from
nbconvert->jupyter->jupyter-cjk-xelatex) (0.3)
Requirement already satisfied: defusedxml in
/home/ec2-user/anaconda3/envs/pytorch\_p36/lib/python3.6/site-packages (from
nbconvert->jupyter->jupyter-cjk-xelatex) (0.6.0)
Requirement already satisfied: nbformat>=4.4 in
/home/ec2-user/anaconda3/envs/pytorch\_p36/lib/python3.6/site-packages (from
nbconvert->jupyter->jupyter-cjk-xelatex) (5.0.7)
Requirement already satisfied: testpath in
/home/ec2-user/anaconda3/envs/pytorch\_p36/lib/python3.6/site-packages (from
nbconvert->jupyter->jupyter-cjk-xelatex) (0.4.4)
Requirement already satisfied: jinja2>=2.4 in
/home/ec2-user/anaconda3/envs/pytorch\_p36/lib/python3.6/site-packages (from
nbconvert->jupyter->jupyter-cjk-xelatex) (2.11.2)
Requirement already satisfied: mistune<2,>=0.8.1 in
/home/ec2-user/anaconda3/envs/pytorch\_p36/lib/python3.6/site-packages (from
nbconvert->jupyter->jupyter-cjk-xelatex) (0.8.4)
Requirement already satisfied: pandocfilters>=1.4.1 in
/home/ec2-user/anaconda3/envs/pytorch\_p36/lib/python3.6/site-packages (from
nbconvert->jupyter->jupyter-cjk-xelatex) (1.4.2)
Requirement already satisfied: bleach in
/home/ec2-user/anaconda3/envs/pytorch\_p36/lib/python3.6/site-packages (from
nbconvert->jupyter->jupyter-cjk-xelatex) (3.1.5)
Requirement already satisfied: prometheus-client in
/home/ec2-user/anaconda3/envs/pytorch\_p36/lib/python3.6/site-packages (from
notebook->jupyter->jupyter-cjk-xelatex) (0.8.0)
Requirement already satisfied: Send2Trash in
/home/ec2-user/anaconda3/envs/pytorch\_p36/lib/python3.6/site-packages (from
notebook->jupyter->jupyter-cjk-xelatex) (1.5.0)
Requirement already satisfied: terminado>=0.8.1 in
/home/ec2-user/anaconda3/envs/pytorch\_p36/lib/python3.6/site-packages (from
notebook->jupyter->jupyter-cjk-xelatex) (0.8.3)
Requirement already satisfied: widgetsnbextension\textasciitilde{}=3.5.0 in
/home/ec2-user/anaconda3/envs/pytorch\_p36/lib/python3.6/site-packages (from
ipywidgets->jupyter->jupyter-cjk-xelatex) (3.5.1)
Requirement already satisfied: python-dateutil>=2.1 in
/home/ec2-user/anaconda3/envs/pytorch\_p36/lib/python3.6/site-packages (from
jupyter-client>=4.1->qtconsole->jupyter->jupyter-cjk-xelatex) (2.8.1)
Requirement already satisfied: six in
/home/ec2-user/anaconda3/envs/pytorch\_p36/lib/python3.6/site-packages (from
traitlets->qtconsole->jupyter->jupyter-cjk-xelatex) (1.15.0)
Requirement already satisfied: decorator in
/home/ec2-user/anaconda3/envs/pytorch\_p36/lib/python3.6/site-packages (from
traitlets->qtconsole->jupyter->jupyter-cjk-xelatex) (4.4.2)
Requirement already satisfied: pickleshare in
/home/ec2-user/anaconda3/envs/pytorch\_p36/lib/python3.6/site-packages (from
ipython>=5.0.0->ipykernel->jupyter->jupyter-cjk-xelatex) (0.7.5)
Requirement already satisfied: pexpect; sys\_platform != "win32" in
/home/ec2-user/anaconda3/envs/pytorch\_p36/lib/python3.6/site-packages (from
ipython>=5.0.0->ipykernel->jupyter->jupyter-cjk-xelatex) (4.8.0)
Requirement already satisfied: backcall in
/home/ec2-user/anaconda3/envs/pytorch\_p36/lib/python3.6/site-packages (from
ipython>=5.0.0->ipykernel->jupyter->jupyter-cjk-xelatex) (0.2.0)
Requirement already satisfied: setuptools>=18.5 in
/home/ec2-user/anaconda3/envs/pytorch\_p36/lib/python3.6/site-packages (from
ipython>=5.0.0->ipykernel->jupyter->jupyter-cjk-xelatex) (49.2.0.post20200714)
Requirement already satisfied: jedi>=0.10 in
/home/ec2-user/anaconda3/envs/pytorch\_p36/lib/python3.6/site-packages (from
ipython>=5.0.0->ipykernel->jupyter->jupyter-cjk-xelatex) (0.17.1)
Requirement already satisfied: wcwidth in
/home/ec2-user/anaconda3/envs/pytorch\_p36/lib/python3.6/site-packages (from
prompt-toolkit!=3.0.0,!=3.0.1,<3.1.0,>=2.0.0->jupyter-console->jupyter->jupyter-
cjk-xelatex) (0.2.5)
Requirement already satisfied: jsonschema!=2.5.0,>=2.4 in
/home/ec2-user/anaconda3/envs/pytorch\_p36/lib/python3.6/site-packages (from
nbformat>=4.4->nbconvert->jupyter->jupyter-cjk-xelatex) (3.0.2)
Requirement already satisfied: MarkupSafe>=0.23 in
/home/ec2-user/anaconda3/envs/pytorch\_p36/lib/python3.6/site-packages (from
jinja2>=2.4->nbconvert->jupyter->jupyter-cjk-xelatex) (1.1.1)
Requirement already satisfied: webencodings in
/home/ec2-user/anaconda3/envs/pytorch\_p36/lib/python3.6/site-packages (from
bleach->nbconvert->jupyter->jupyter-cjk-xelatex) (0.5.1)
Requirement already satisfied: packaging in
/home/ec2-user/anaconda3/envs/pytorch\_p36/lib/python3.6/site-packages (from
bleach->nbconvert->jupyter->jupyter-cjk-xelatex) (20.4)
Requirement already satisfied: ptyprocess>=0.5 in
/home/ec2-user/anaconda3/envs/pytorch\_p36/lib/python3.6/site-packages (from
pexpect; sys\_platform != "win32"->ipython>=5.0.0->ipykernel->jupyter->jupyter-
cjk-xelatex) (0.6.0)
Requirement already satisfied: parso<0.8.0,>=0.7.0 in
/home/ec2-user/anaconda3/envs/pytorch\_p36/lib/python3.6/site-packages (from
jedi>=0.10->ipython>=5.0.0->ipykernel->jupyter->jupyter-cjk-xelatex) (0.7.0)
Requirement already satisfied: attrs>=17.4.0 in
/home/ec2-user/anaconda3/envs/pytorch\_p36/lib/python3.6/site-packages (from
jsonschema!=2.5.0,>=2.4->nbformat>=4.4->nbconvert->jupyter->jupyter-cjk-xelatex)
(19.3.0)
Requirement already satisfied: pyrsistent>=0.14.0 in
/home/ec2-user/anaconda3/envs/pytorch\_p36/lib/python3.6/site-packages (from
jsonschema!=2.5.0,>=2.4->nbformat>=4.4->nbconvert->jupyter->jupyter-cjk-xelatex)
(0.16.0)
Requirement already satisfied: pyparsing>=2.0.2 in
/home/ec2-user/anaconda3/envs/pytorch\_p36/lib/python3.6/site-packages (from
packaging->bleach->nbconvert->jupyter->jupyter-cjk-xelatex) (2.4.7)
Building wheels for collected packages: jupyter-cjk-xelatex
  Building wheel for jupyter-cjk-xelatex (setup.py) {\ldots} done
  Created wheel for jupyter-cjk-xelatex:
filename=jupyter\_cjk\_xelatex-0.2-py3-none-any.whl size=2076
sha256=7958ac18c463bb7317602ce83b18b6961573c844ee669d2095738ea2ccfaac11
  Stored in directory: /home/ec2-user/.cache/pip/wheels/5e/03/62/b6652316f429b43
ac12e266fb32593cb127d6d19ab1a2dff12
Successfully built jupyter-cjk-xelatex
Installing collected packages: jupyter-cjk-xelatex
Successfully installed jupyter-cjk-xelatex-0.2
\textcolor{ansi-yellow}{WARNING: You are using pip version 20.1.1; however, version 20.2.3 is
available.
You should consider upgrading via the
'/home/ec2-user/anaconda3/envs/pytorch\_p36/bin/python -m pip install --upgrade
pip' command.}
    \end{Verbatim}

    \begin{tcolorbox}[breakable, size=fbox, boxrule=1pt, pad at break*=1mm,colback=cellbackground, colframe=cellborder]
\prompt{In}{incolor}{51}{\boxspacing}
\begin{Verbatim}[commandchars=\\\{\}]
\PY{o}{!}pip install nbconvert
\end{Verbatim}
\end{tcolorbox}

    \begin{Verbatim}[commandchars=\\\{\}]
Requirement already satisfied: nbconvert in
/home/ec2-user/anaconda3/envs/pytorch\_p36/lib/python3.6/site-packages (5.6.1)
Requirement already satisfied: jupyter-core in
/home/ec2-user/anaconda3/envs/pytorch\_p36/lib/python3.6/site-packages (from
nbconvert) (4.6.3)
Requirement already satisfied: defusedxml in
/home/ec2-user/anaconda3/envs/pytorch\_p36/lib/python3.6/site-packages (from
nbconvert) (0.6.0)
Requirement already satisfied: traitlets>=4.2 in
/home/ec2-user/anaconda3/envs/pytorch\_p36/lib/python3.6/site-packages (from
nbconvert) (4.3.3)
Requirement already satisfied: bleach in
/home/ec2-user/anaconda3/envs/pytorch\_p36/lib/python3.6/site-packages (from
nbconvert) (3.1.5)
Requirement already satisfied: mistune<2,>=0.8.1 in
/home/ec2-user/anaconda3/envs/pytorch\_p36/lib/python3.6/site-packages (from
nbconvert) (0.8.4)
Requirement already satisfied: pygments in
/home/ec2-user/anaconda3/envs/pytorch\_p36/lib/python3.6/site-packages (from
nbconvert) (2.6.1)
Requirement already satisfied: jinja2>=2.4 in
/home/ec2-user/anaconda3/envs/pytorch\_p36/lib/python3.6/site-packages (from
nbconvert) (2.11.2)
Requirement already satisfied: nbformat>=4.4 in
/home/ec2-user/anaconda3/envs/pytorch\_p36/lib/python3.6/site-packages (from
nbconvert) (5.0.7)
Requirement already satisfied: entrypoints>=0.2.2 in
/home/ec2-user/anaconda3/envs/pytorch\_p36/lib/python3.6/site-packages (from
nbconvert) (0.3)
Requirement already satisfied: testpath in
/home/ec2-user/anaconda3/envs/pytorch\_p36/lib/python3.6/site-packages (from
nbconvert) (0.4.4)
Requirement already satisfied: pandocfilters>=1.4.1 in
/home/ec2-user/anaconda3/envs/pytorch\_p36/lib/python3.6/site-packages (from
nbconvert) (1.4.2)
Requirement already satisfied: six in
/home/ec2-user/anaconda3/envs/pytorch\_p36/lib/python3.6/site-packages (from
traitlets>=4.2->nbconvert) (1.15.0)
Requirement already satisfied: ipython-genutils in
/home/ec2-user/anaconda3/envs/pytorch\_p36/lib/python3.6/site-packages (from
traitlets>=4.2->nbconvert) (0.2.0)
Requirement already satisfied: decorator in
/home/ec2-user/anaconda3/envs/pytorch\_p36/lib/python3.6/site-packages (from
traitlets>=4.2->nbconvert) (4.4.2)
Requirement already satisfied: webencodings in
/home/ec2-user/anaconda3/envs/pytorch\_p36/lib/python3.6/site-packages (from
bleach->nbconvert) (0.5.1)
Requirement already satisfied: packaging in
/home/ec2-user/anaconda3/envs/pytorch\_p36/lib/python3.6/site-packages (from
bleach->nbconvert) (20.4)
Requirement already satisfied: MarkupSafe>=0.23 in
/home/ec2-user/anaconda3/envs/pytorch\_p36/lib/python3.6/site-packages (from
jinja2>=2.4->nbconvert) (1.1.1)
Requirement already satisfied: jsonschema!=2.5.0,>=2.4 in
/home/ec2-user/anaconda3/envs/pytorch\_p36/lib/python3.6/site-packages (from
nbformat>=4.4->nbconvert) (3.0.2)
Requirement already satisfied: pyparsing>=2.0.2 in
/home/ec2-user/anaconda3/envs/pytorch\_p36/lib/python3.6/site-packages (from
packaging->bleach->nbconvert) (2.4.7)
Requirement already satisfied: setuptools in
/home/ec2-user/anaconda3/envs/pytorch\_p36/lib/python3.6/site-packages (from
jsonschema!=2.5.0,>=2.4->nbformat>=4.4->nbconvert) (49.2.0.post20200714)
Requirement already satisfied: pyrsistent>=0.14.0 in
/home/ec2-user/anaconda3/envs/pytorch\_p36/lib/python3.6/site-packages (from
jsonschema!=2.5.0,>=2.4->nbformat>=4.4->nbconvert) (0.16.0)
Requirement already satisfied: attrs>=17.4.0 in
/home/ec2-user/anaconda3/envs/pytorch\_p36/lib/python3.6/site-packages (from
jsonschema!=2.5.0,>=2.4->nbformat>=4.4->nbconvert) (19.3.0)
\textcolor{ansi-yellow}{WARNING: You are using pip version 20.1.1; however, version 20.2.3 is
available.
You should consider upgrading via the
'/home/ec2-user/anaconda3/envs/pytorch\_p36/bin/python -m pip install --upgrade
pip' command.}
    \end{Verbatim}

    \begin{tcolorbox}[breakable, size=fbox, boxrule=1pt, pad at break*=1mm,colback=cellbackground, colframe=cellborder]
\prompt{In}{incolor}{ }{\boxspacing}
\begin{Verbatim}[commandchars=\\\{\}]

\end{Verbatim}
\end{tcolorbox}


    % Add a bibliography block to the postdoc
    
    
    
\end{document}
